%\section{Reacting Mixing Layer Equations}


\begin{figure}
   \centerline{\includegraphics[width=1\textwidth]{reacting.jpeg}} 
\label{f1}
\caption{Reacting Mixing Layer}
\end{figure}


\subsection{Conservation's Equations}
 
In a compressible, conservative form and in a perfect gas the conservation's
equations are:
\\
\\
%\vspace{5cm}
\textbf{Mass conservation:}

\begin{equation} 
\frac{\partial\rho}{\partial t}+
\frac{\partial}{\partial x_i}(\rho u_i)=0
\quad \text{for} \quad i=1,2,3
,
\label{mass}
\end{equation} 
\\
\\
%\vspace{5cm}
\textbf{Species conservation:}

\begin{equation} 
\frac{\partial}{\partial t}(\rho Y^m)+
\frac{\partial}{\partial x_i}(\rho Y^m(u_i+V^m))=s^m\Omega
\quad \text{for} \quad i=1,2,3
\quad \text{for} \quad m=\text{Oxidizer, Fuel}
%\frac{\partial\rho D^m \tfrac{\partial\rho Y^m u_i}{\partial x_i}u_i}{\partial x_i}=
\end{equation} 
\\
\\
Using a constitutive equation for $V^m$ 

\begin{equation} 
V^m=-\rho D^m\frac{\partial Y^m}{\partial x_i} 
%\vspace{5cm}
\end{equation} 
\\
where $D^m$ is the diffusion coefficient of specie 
respect to the most abundant specie.

{\color{red} who is $\Omega$?}
\\
\\
\textbf{Momemtum conservation:}
\\
\\
\begin{equation} 
\frac{\partial}{\partial t}(\rho u_j)+
\frac{\partial}{\partial x_i}(\rho u_iu_j)=-
\frac{\partial p}{\partial x_j}+
\frac{\partial \tau_{ij}}{\partial x_i}+
\rho f_j
\quad \text{for} \quad i,j=1,2,3
%\frac{\partial\rho D^m \tfrac{\partial\rho Y^m u_i}{\partial x_i}u_i}{\partial x_i}=
\end{equation} 
\\
\\
where 
\\
\\
\begin{equation} 
\tau_{ij}=\lambda \delta_{ij}\frac{\partial u_k}{\partial x_k}+
\mu\left(\frac{\partial u_i}{\partial x_j}+\frac{\partial u_j}{\partial x_i}\right)
\end{equation} 
\\
\\
and using the Stoke's relation: 
\\
\\
\begin{equation} 
\lambda=-\frac{2}{3}\mu
\end{equation} 

\begin{equation} 
\tau_{ij}=
\mu\left(
-\frac{2}{3} \delta_{ij}\frac{\partial u_k}{\partial x_k}+
\left(\frac{\partial u_i}{\partial x_j}+\frac{\partial u_j}{\partial x_i}\right)
\right)
\end{equation} 
\\
\\
%\vspace{5cm}
\textbf{Energy conservation:}
\\
\\
\begin{equation} 
\frac{\partial}{\partial t}(\rho h)+
\frac{\partial}{\partial x_i}(\rho u_i h)=
\frac{\partial p}{\partial t}+
u_i\frac{\partial p}{\partial x_i}+
\frac{\partial u_j}{\partial x_i}\tau_{ij}
+
\frac{\partial}{\partial x_i}\left(k\frac{\partial T}{\partial x_i}\right)
-
\frac{\partial}{\partial x_i}
\sum_i\rho D_i\frac{\partial Y_i}{\partial x_i}
+
Q\Omega
\end{equation} 
\\
where 
$\dfrac{\partial}{\partial x_i}
\sum_i\rho D_i\dfrac{\partial Y_i}{\partial y}$
is the 
enthalpy transported by species diffusion.
\\
\\
\textbf{Perfect gas equation:}
\\
\begin{equation} 
p=\rho T R
\end{equation} 
\\
\begin{equation}
h=c_pT
\end{equation} 
\\
\\
\\
In the above equations $\rho$ is the density, $u_i=(u,v,w)$ 
are the different velocities in the three different dimensions $x$, $y$ and $z$.
$p$ is the pressure of the flow, $T$ is the temperature of the flow, $f_j$ 
is the body force acting on the flow.
$Y^m$ are the different mass fraction for the two different species, oxygen and fuel.
$\mu$, $D^{m}$ and $k$ are the viscosity, diffusion coefficient for the different 
species and the gas thermal conductivity.  $Q$ is the heat release for the chemical reaction and
$\Omega$ which is the mass reaction rate. 
\\
\\
\textbf{Base Flow equations:}
\\
\\
We are interesting in the a two dimensional compressible mixing layer, because the 
tridimensional effect can be neglected for the study of instabilities in the first stages, 
 it states that for {\it the onset of linear instability in parallel shear flows, the
 least stable (i.e., first to become unstable) disturbances are two-dimensiona} \cite{Squire}.
In the incompressible case, the process of transition to turbulence in a mixing layer 
is dictated by: the growth of two dimensional coherent structures and the development
of secondary instability,  the merging of the large-structures and finally the 
breakdown into small-scales three dimensional turbulence \cite{planche1992numerical}.

The mean flow for two dimensional mixing layer
which separates two fluids, the oxidizer and the fuel, 
at different speeds and temperature with zero pressure gradient can be assumed 
that is governed by boundary layer equations.
This mean that gradients for the different properties are in the $y$ direction, in Cartesian 
coordinates.

To applied the boundary layer equations is necessary  to defined a characteristic 
length of the flow. In this problem exist two characteristics lengths, one
can be defined as the thickness  of the mixing layer $\delta$ and other can be 
defined as the distance that the mixing layer needs to grow $L_c$.

Assume that $\delta$  is sufficient small compared  with the $x$ length
where the mixing layer is development, it mean that $\delta/L_c<<1$.  
The mean velocity scale can be approximate, in the $x$ direction is the 
velocity of order of $U_c$,
and $\partial/\partial x$ is of order of $1/L_c$ 
and assuming that $\rho$ is of order of $\rho_c$ then:

\begin{equation}
\frac{\partial}{\partial x}(\rho u)\sim \frac{\rho_cU_c}{L_c}
\end{equation}

Using the conservation of mass equation \ref{mass} in two dimensions $(x,y)$, 
the order of 
\\
\\
\begin{equation}
\frac{\partial}{\partial y}(\rho v)\sim \frac{\rho_cU_c}{L_c}
\end{equation}
\\
\\
We know that in the mixing layer the velocity $v$ is smaller than
$u$, then we can assume that 
\\
\\
\begin{equation}
v\sim\frac{\delta}{L_c}U_c\qquad
\text{ being }\qquad
\frac{\partial}{\partial y}\sim \frac{1}{\delta}.
\end{equation}
\\
\\
With this assumptions the terms in the conservation equations of order: 
\\
\\
\begin{equation}
\frac{\partial^2}{\partial x^2}\sim\frac{1}{L_c^2},
\end{equation}
\\
\\
are smaller that term of order   
\\
\\
\begin{equation}
\frac{\partial^2}{\partial y^2}\sim\frac{1}{\delta^2}.
\end{equation}
\\
\\
Thereby, using the different length scales in the others conservations equations
and assuming steady state to the mean flow we get: 
\\
\\
\textbf{Mass conservation:}
\\
\\
\begin{equation} 
\frac{\partial}{\partial x}(\rho u)=
\frac{\partial}{\partial y}(\rho v)
\label{mass1}
\end{equation} 
\\
\\
%Using the characteristic lenght scales of the problem we obtein:
%\\
%\\
%\begin{equation} 
%\frac{\rho_cU}{}\frac{\partial}{\partial x}(\rho u)=
%\frac{}{}\frac{\partial}{\partial y}(\rho v)
%\label{mass1}
%\end{equation} 
\\
\\
\textbf{Species conservation:}
\\
\\
\begin{equation} 
\frac{\partial}{\partial x}(\rho Y^{o}(u+V^{o}))+
\frac{\partial}{\partial y}(\rho Y^{o}(v+V^{o}))
=s^{o}\Omega
%\frac{\partial\rho D^m \tfrac{\partial\rho Y^m u_i}{\partial x_i}u_i}{\partial x_i}=
\end{equation} 
\\
\\
\begin{equation} 
\frac{\partial}{\partial x}(\rho Y^{F}(u+V^{F}))+
\frac{\partial}{\partial y}(\rho Y^{F}(v+V^{F}))
=s^{F}\Omega
%\frac{\partial\rho D^m \tfrac{\partial\rho Y^m u_i}{\partial x_i}u_i}{\partial x_i}=
\end{equation} 
\\
\\
Using a constitutive equation for $V^m$ 
\\
\\
\begin{equation} 
V^m=-\rho D^m\frac{\partial }{\partial x_i}(ln(Y^m))
\quad \text{for} \quad m=1,2
%\vspace{5cm}
\end{equation} 
\\
\\
Hence
\\
\\
\begin{equation} 
\frac{\partial}{\partial x_i}(\rho Y^{m}V^m)=
-\frac{\partial}{\partial x_i}\left(\rho D^m\frac{\partial Y^m}{\partial x_i}\right)
\quad \text{for} \quad m=1,2
%\vspace{5cm}
\end{equation} 
\\
\\
\begin{equation} 
\frac{\partial}{\partial x_i}(\rho Y^{m}V^m)=
-\frac{\partial}{\partial x}\left(\rho D^m\frac{\partial Y^m}{\partial x}\right)
-\frac{\partial}{\partial y}\left(\rho D^m\frac{\partial Y^m}{\partial y}\right)
\quad \text{for} \quad m=1,2
%\vspace{5cm}
\end{equation} 
\\
\\
Using the characteristic length scales:
\\
\\
\begin{equation} 
\frac{\partial}{\partial x_i}(\rho Y^{m}V^m)=
-\frac{1}{ L_c}\frac{\rho_c D_c^mY_c^m}{L_c}
-\frac{1}{ \delta}\frac{\rho_c D_c^mY_c^m}{\delta}
\quad \text{for} \quad m=1,2
%\vspace{5cm}
\end{equation} 
\\
\\
\begin{equation} 
\frac{\partial}{\partial x_i}(\rho Y^{m}V^m)=
-\frac{\rho_c D_c^mY_c^m}{ L_c^2}
-\frac{\rho_c D_c^mY_c^m}{ \delta^2}
\quad \text{for} \quad m=1,2
%\vspace{5cm}
\end{equation} 
\\
\\
\begin{equation} 
\frac{\partial}{\partial x_i}(\rho Y^{m}V^m)=
\cancelto{\approx~0}{
-\frac{\rho_c D_c^mY_c^m}{ L_c^2}
}
-\frac{\rho_c D_c^mY_c^m}{ \delta^2}
\quad \text{for} \quad m=1,2
%\vspace{5cm}
\end{equation} 
\\
\\
The Species conservation can be wrote as:
\\
\\
\begin{equation} 
\frac{\partial}{\partial x}(\rho Y^{o}u)+
\frac{\partial}{\partial y}(\rho Y^{o}v)=
\frac{\partial}{\partial y}\left(\rho D^o\frac{\partial Y^o}{\partial y}\right)+
s^{o}\Omega
%\frac{\partial\rho D^m \tfrac{\partial\rho Y^m u_i}{\partial x_i}u_i}{\partial x_i}=
\end{equation} 
\\
\begin{equation} 
\frac{\partial}{\partial x}(\rho Y^{F}u)+
\frac{\partial}{\partial y}(\rho Y^{F}v)=
\frac{\partial}{\partial y}\left(\rho D^F\frac{\partial Y^F}{\partial y}\right)+
s^{F}\Omega
%\frac{\partial\rho D^m \tfrac{\partial\rho Y^m u_i}{\partial x_i}u_i}{\partial x_i}=
\end{equation} 
\\
\\ 
Using the mass conservation equation \ref{mass}, we can write 
the Species conservation as:
\\
\\
\begin{equation} 
\rho u\frac{\partial Y^{o}}{\partial x}+
\rho v\frac{\partial Y^{o}}{\partial y}=
\frac{\partial}{\partial y}\left(\rho D^o\frac{\partial Y^o}{\partial y}\right)+
s^{o}\Omega
%\frac{\partial\rho D^m \tfrac{\partial\rho Y^m u_i}{\partial x_i}u_i}{\partial x_i}=
\end{equation} 
\\
\begin{equation} 
\rho u\frac{\partial Y^{F}}{\partial x}+
\rho v\frac{\partial Y^{F}}{\partial y}=
\frac{\partial}{\partial y}\left(\rho D^F\frac{\partial Y^F}{\partial y}\right)+
s^{F}\Omega
%\frac{\partial\rho D^m \tfrac{\partial\rho Y^m u_i}{\partial x_i}u_i}{\partial x_i}=
\end{equation} 
\\
\\ 

%\vspace{5cm}
\textbf{Momemtum conservation:}
\\
\\
\begin{equation} 
%\frac{\partial}{\partial t}(\rho u_j)+
\frac{\partial}{\partial x_i}(\rho u_iu_j)=-
\frac{\partial p}{\partial x_j}+
\frac{\partial \tau_{ij}}{\partial x_i}+
\rho f_j
\quad \text{for} \quad i,j=1,2,3
%\frac{\partial\rho D^m \tfrac{\partial\rho Y^m u_i}{\partial x_i}u_i}{\partial x_i}=
\end{equation} 
\\
\\
\begin{equation} 
\tau_{ij}=
\lambda \delta_{ij}\frac{\partial u_k}{\partial x_k}+
\mu\left(\frac{\partial u_i}{\partial x_j}+\frac{\partial u_j}{\partial x_i}\right)
\end{equation} 
\\
\\
\begin{equation} 
\frac{\partial\tau_{ij}}{\partial x_i}=
\frac{\partial}{\partial x_i}\left(\lambda \delta_{ij}\frac{\partial u_k}{\partial x_k}\right)+
\frac{\partial}{\partial x_i}\left(\mu\left(\frac{\partial u_i}{\partial x_j}
+\frac{\partial u_j}{\partial x_i}\right)\right)
\end{equation} 
\\
\\
\begin{equation} 
\frac{\partial\tau_{ij}}{\partial x_i}=
\frac{\partial}{\partial x_j}\left(\lambda \frac{\partial u_k}{\partial x_k}\right)+
\frac{\partial}{\partial x_i}\left(\mu\frac{\partial u_i}{\partial x_j}\right)+
\frac{\partial}{\partial x_i}\left(\mu\frac{\partial u_j}{\partial x_i}\right)
\end{equation} 
\\
\\
\begin{equation} 
\frac{\partial\tau_{i1}}{\partial x_i}=
\frac{\partial}{\partial x}\left(\lambda\frac{\partial u}{\partial x}\right)+
\frac{\partial}{\partial x}\left(\lambda\frac{\partial v}{\partial x}\right)+
\frac{\partial}{\partial x}\left(\mu\frac{\partial u}{\partial x}\right)+
\frac{\partial}{\partial y}\left(\mu\frac{\partial v}{\partial x}\right)+
\frac{\partial}{\partial x}\left(\mu\frac{\partial u}{\partial x}\right)+
\frac{\partial}{\partial y}\left(\mu\frac{\partial u}{\partial y}\right)
\end{equation} 
\\
\\
\begin{equation} 
\frac{\partial\tau_{i1}}{\partial x_i}=
\frac{1}{L_c^2}(\lambda_c U_c)+
\frac{1}{L_c^3}(\lambda_c \delta U_c)+
\frac{1}{L_c^2}(\mu_c U_c)+
\frac{1}{L_c^2}(\mu_c U_c)+
\frac{1}{L_c^2}(\mu_c U_c)+
\frac{1}{\delta^2}(\mu_c U_c)+
\end{equation} 
\\
\\
\begin{equation} 
\frac{\partial\tau_{i1}}{\partial x_i}=
	\cancelto{\approx0}{\frac{1}{L_c^2}(\lambda_c U_c)}+
\cancelto{\approx0}{\frac{1}{L_c^3}(\lambda_c \delta U_c)}+
\cancelto{\approx0}{\frac{1}{L_c^2}(\mu_c U_c)}+
\cancelto{\approx0}{\frac{1}{L_c^2}(\mu_c U_c)}+
\cancelto{\approx0}{\frac{1}{L_c^2}(\mu_c U_c)}+
\frac{1}{\delta^2}(\mu_c U_c)+
\end{equation} 
\\
\\
\begin{equation} 
\frac{\partial\tau_{i1}}{\partial x_i}=
\frac{\partial}{\partial y}\left(\mu\frac{\partial u}{\partial y}\right)
\end{equation} 
\\
\\
\begin{equation} 
\frac{\partial\tau_{i2}}{\partial x_i}=
\frac{\partial}{\partial y}\left(\lambda\frac{\partial u}{\partial x}\right)+
\frac{\partial}{\partial y}\left(\lambda\frac{\partial v}{\partial y}\right)+
\frac{\partial}{\partial x}\left(\mu\frac{\partial u}{\partial y}\right)+
\frac{\partial}{\partial y}\left(\mu\frac{\partial v}{\partial y}\right)+
\frac{\partial}{\partial x}\left(\mu\frac{\partial v}{\partial x}\right)+
\frac{\partial}{\partial y}\left(\mu\frac{\partial v}{\partial y}\right)
\end{equation} 
\\
\\
Now using the characteristic length of the problem we found:
\\
\\
\begin{equation} 
\frac{\partial\tau_{i2}}{\partial x_i}=
\frac{1}{\delta L_c}(\lambda_c U_c)+
\frac{1}{\delta L_c}(\lambda_c U_c)+
\frac{1}{\delta L_c}(\mu_c U_c)+
\frac{1}{\delta L_c}(\mu_c U_c)+
\frac{1}{L_c^3}(\mu_c \delta U_c)+
\frac{1}{\delta L_c}(\mu_c U_c)
\label{my}
\end{equation} 
\\
Multiply \ref{my} by $\delta^2$ 
\\
\\
\begin{equation} 
\frac{\partial\tau_{i2}}{\partial x_i}=
\left(\frac{\delta}{ L_c}\right)(\lambda_c U_c)+
\left(\frac{\delta}{L_c}\right)(\lambda_c U_c)+
\left(\frac{\delta}{ L_c}\right)(\mu_c U_c)+
\left(\frac{\delta}{L_c}\right)(\mu_c U_c)+
\left(\frac{\delta}{L_c}\right)^3(\mu_c \delta U_c)+
\left(\frac{\delta}{L_c}\right)(\mu_c U_c)
\end{equation} 
\\
\\
\begin{equation} 
\frac{\partial\tau_{i2}}{\partial x_i}=
\cancelto{\approx0}{\left(\frac{\delta}{ L_c}\right)(\lambda_c U_c)}+
\cancelto{\approx0}{\left(\frac{\delta}{L_c}\right)(\lambda_c U_c)}+
\cancelto{\approx0}{\left(\frac{\delta}{ L_c}\right)(\mu_c U_c)}+
\cancelto{\approx0}{\left(\frac{\delta}{L_c}\right)(\mu_c U_c)}+
\cancelto{\approx0}{\left(\frac{\delta}{L_c}\right)^3(\mu_c \delta U_c)}+
\cancelto{\approx0}{\left(\frac{\delta}{L_c}\right)(\mu_c U_c)}
\end{equation} 
\\
\\
\begin{equation} 
\frac{\partial\tau_{i2}}{\partial x_i}\approx0
\end{equation} 
\\
\\
Now Momentum conservation equation in $x$ direction can be wrote:
\\
\\
\begin{equation} 
\frac{\partial}{\partial x}(\rho uu)+
\frac{\partial}{\partial y}(\rho vu)=-
\frac{\partial p}{\partial x}+
\frac{\partial}{\partial y}\left(\mu\frac{\partial u}{\partial y}\right)+
\rho f_x
\end{equation} 
\\
\\
Using the characteristic length of the problem, we can see that 
the others terms are important:
\\
\\
\begin{equation} 
\frac{1}{L_c}(\rho_c U_c^2)+
\frac{1}{L_c}(\rho_c U_c^2)=
\frac{p_c}{L_c}+
\frac{1}{\delta^2}(\mu_c U_c)+
\rho_c f_{xc}.
\label{mx}
\end{equation} 
%
In the viscous term multiple by $\dfrac{L_c}{U_c^2\rho_c}$:
%
\begin{equation} 
\frac{L_c}{U_c^2\rho_c}
\frac{1}{\delta^2}(\mu_c U_c)=
\frac{\mu_c }{\rho_c\delta}
\frac{L_c}{\delta U_c}=
\frac{\mu_c }{\rho_cV_c\delta}
\end{equation} 
% 
In order to keep the approximation $\delta_c<<L_c$
this term:

\begin{equation} 
\frac{\mu_c }{\rho_cV_c\delta}\approx1=Re
\end{equation} 
%
\\
\\
Using the mass conservation equation \ref{mass1} in \ref{mx}: 
\\
\\
\begin{equation} 
\rho u\frac{\partial u}{\partial x}+
\rho v\frac{\partial u}{\partial y}=-
\frac{\partial p}{\partial x}+
\frac{\partial}{\partial y}\left(\mu\frac{\partial u}{\partial y}\right)+
\rho f_x
\end{equation} 
\\
\\
The Momentum equation in $y$ direction can be wrote as:
\\
\\

\begin{equation} 
\frac{\partial}{\partial x}(\rho uv)+
\frac{\partial}{\partial y}(\rho vv)=-
\frac{\partial p}{\partial y}+
\rho f_y
\end{equation} 
\\
\\
Using the characteristic length of the problem, we can see that 
the others terms are important:
\\
\\
\begin{equation} 
\frac{1}{L_c}\frac{\delta}{L_c}(\rho_c U_c^2)+
\frac{1}{L_c}\frac{\delta}{L_c}(\rho_c U_c^2)=
-\frac{p_c}{L_c}+
\rho_c f_{yc}.
\end{equation} 
\\
\\
\begin{equation} 
\cancelto{\approx0}{\frac{1}{L_c}\frac{\delta}{L_c}(\rho_c U_c^2)}+
\cancelto{\approx0}{\frac{1}{L_c}\frac{\delta}{L_c}(\rho_c U_c^2)}=
-\frac{p_c}{L_c}+
\rho_c f_{yc}.
\end{equation} 
\\
\\
Then the $y$ momentum equation using the boundary layer assumption can be wrote as:
\\
\\
\begin{equation} 
\frac{\partial p}{\partial y}=
\rho f_y.
\end{equation} 
\\
\\
This equation means that the only variations of the pressure in $y$ is determined 
by the body forces in that direction.
%\vspace{5cm}
\\
\\
\textbf{Energy conservation:}
\\
\begin{equation} 
\frac{\partial}{\partial x_i}(\rho u_i h)=
u_i\frac{\partial p}{\partial x_i}+
\frac{\partial u_j}{\partial x_i}\tau_{ij}-
\frac{\partial}{\partial x_i}\left(k\frac{\partial T}{\partial x_i}\right)
-
\frac{\partial}{\partial x_i}
\sum_i\rho D_i\frac{\partial Y_i}{\partial x_i}+
+
Q\Omega
\end{equation} 
\\
\begin{equation} 
\tau_{ij}=
\lambda \delta_{ij}\frac{\partial u_k}{\partial x_k}+
\mu\left(\frac{\partial u_i}{\partial x_j}
+\frac{\partial u_j}{\partial x_i}\right)
\end{equation} 
\\
\\
\begin{equation} 
\frac{\partial u}{\partial x_i}\tau_{i1}=
\frac{\partial u}{\partial x_i}\lambda \delta_{i1}\frac{\partial u_k}{\partial x_k}+
\frac{\partial u}{\partial x_i}\mu\left(\frac{\partial u_i}{\partial x}
+\frac{\partial u}{\partial x_i}\right)
\end{equation} 
\\
\\
\begin{equation} 
\frac{\partial u}{\partial x_i}\tau_{i1}=
\frac{\partial u}{\partial x}\lambda \frac{\partial u_k}{\partial x_k}+
\frac{\partial u}{\partial x_i}\mu\left(\frac{\partial u_i}{\partial x}
+\frac{\partial u}{\partial x_i}\right)
\end{equation} 
\\
\\
\begin{equation} 
\frac{\partial u}{\partial x_i}\tau_{i1}=
\frac{\partial u}{\partial x}
\lambda 
\left(
     \frac{\partial u}{\partial x}+
     \frac{\partial v}{\partial y}
\right)+
\mu\frac{\partial u}{\partial x_i}
\frac{\partial u_i}{\partial x}+
\mu\left(\frac{\partial u}{\partial x_i}\right)^2
\end{equation} 
\\
\\
\begin{equation} 
\frac{\partial u}{\partial x_i}\tau_{i1}=
\lambda 
\left(\frac{\partial u}{\partial x}\right)^2
+
\lambda 
\frac{\partial u }{\partial x}
\frac{\partial v }{\partial y}
+
\mu\frac{\partial u}{\partial x}
\frac{\partial u}{\partial x}+
\mu\frac{\partial u}{\partial y}
\frac{\partial v}{\partial x}+
\mu\left(\frac{\partial u}{\partial x}\right)^2
+
\mu\left(\frac{\partial u}{\partial y}\right)^2
\end{equation} 
\\
\\
\begin{equation} 
\frac{\partial u}{\partial x_i}\tau_{i1}=
\cancelto{\approx0}{
\lambda 
\left(\frac{\partial u}{\partial x}\right)^2
}
+
\cancelto{\approx0}{
\lambda 
\frac{\partial u }{\partial x}
\frac{\partial v }{\partial y}
}
+
\cancelto{\approx0}{
\mu\frac{\partial u}{\partial x}
\frac{\partial u}{\partial x}
}
+
\cancelto{\approx0}{
\mu\frac{\partial u}{\partial y}
\frac{\partial v}{\partial x}
}
+
\cancelto{\approx0}{
\mu\left(\frac{\partial u}{\partial x}\right)^2
}
+
\mu\left(\frac{\partial u}{\partial y}\right)^2
\end{equation} 
\\
\\
\begin{equation} 
\frac{\partial u}{\partial x_i}\tau_{i1}\approx
\mu\left(\frac{\partial u}{\partial y}\right)^2
\end{equation} 
\\
\\
\begin{equation} 
\frac{\partial v}{\partial x_i}\tau_{i2}=
\frac{\partial v}{\partial x_i}\lambda \delta_{i2}\frac{\partial u_k}{\partial x_k}+
\frac{\partial v}{\partial x_i}\mu\left(\frac{\partial u_i}{\partial y}
+\frac{\partial v}{\partial x_i}\right)
\end{equation} 
\\
\\
\begin{equation} 
\frac{\partial v}{\partial x_i}\tau_{i2}=
\lambda\frac{\partial v}{\partial y}
\left(
\frac{\partial u}{\partial x}+
\frac{\partial v}{\partial y}
\right)
+\mu\frac{\partial v}{\partial x_i}
\frac{\partial u_i}{\partial y}+
\left(
\mu\frac{\partial v}{\partial x_i}\right)^2
\end{equation} 
\\
\\
\begin{equation} 
\frac{\partial v}{\partial x_i}\tau_{i2}=
\lambda\frac{\partial v}{\partial y}
\frac{\partial u}{\partial x}+
\lambda
\left(
\frac{\partial v}{\partial y}\right)^2+
\mu\frac{\partial v}{\partial x}
\frac{\partial u}{\partial y}+
\mu
\left(
\frac{\partial v}{\partial y}
\right)^2+
\mu
\left(
\frac{\partial v}{\partial x}
\right)^2+
\mu
\left(
\frac{\partial v}{\partial y}
\right)^2
\end{equation} 
\\
\\
\begin{equation} 
\frac{\partial v}{\partial x_i}\tau_{i2}=
\cancelto{\approx0}{
\lambda\frac{\partial v}{\partial y}
\frac{\partial u}{\partial x}
}
+
\cancelto{\approx0}{
\left(
\frac{\partial v}{\partial y}
\right)^2
}
+
\cancelto{\approx0}{
\mu\frac{\partial v}{\partial x}
\frac{\partial u}{\partial y}
}
+
\cancelto{\approx0}{
\mu
\left(
\frac{\partial v}{\partial y}
\right)^2
}
+
\mu
\cancelto{\approx0}{
\left(
\frac{\partial v}{\partial x}
\right)^2
}
+
\cancelto{\approx0}{
\left(
\frac{\partial v}{\partial y}
\right)^2
}
\end{equation} 
\\
\\
\\
\\
\begin{equation} 
\frac{\partial v}{\partial x_i}\tau_{i2}\approx0
\end{equation} 

Finally for the heat diffusion:

\begin{equation} 
\frac{\partial kT}{\partial x_i}
=
k
\frac{\partial T}{\partial x}+
k
\frac{\partial T}{\partial y}
\end{equation} 
%
Using the characteristic lengths: 
%
\begin{equation} 
\frac{\partial kT}{\partial x_i}
=
k
\frac{ T_c}{L_c}
+
k
\frac{ T_c}{\delta_c}
\end{equation} 
%
\\
\\
Hence, the energy equation for steady flow become:
\\
\begin{equation} 
\frac{\partial}{\partial x_i}(\rho u_i h)=
u_i\frac{\partial p}{\partial x_i}+
\frac{\partial u_j}{\partial x_i}\tau_{ij}
+
\frac{\partial}{\partial x_i}\left(k\frac{\partial T}{\partial x_i}\right)
-
\frac{\partial }{\partial x_i}
\left(
\sum_i\rho D_i\frac{\partial Y_i}{\partial x_i}+
\right)
+
Q\Omega
\end{equation} 
\\
\begin{equation} 
\frac{\rho_c U_c h_c}{L_c}
+
\frac{\rho_c U_c\delta_c h_c}{L_c\delta_c}
=
U_c\frac{p_c}{L_c}
+
U_c\frac{p_c}{L_c}
+
\frac{U_c\delta_c}{L_c}
\frac{p_c}{\delta_c}
+
\mu\frac{U_c^2}{\delta^2}
+
k
\frac{ T_c}{L_c^2}
+
k
\frac{ T_c}{\delta_c^2}
+
\frac{\rho_cD_i}{\delta_c^2}
+...
\end{equation} 
%
\begin{equation} 
\frac{1}{L_c}
+
\frac{1}{L_c}
=
\frac{p_c}{\rho_c h_c}
\frac{1}{L_c}
+
\frac{p_c}{\rho_c  h_c}
\frac{1}{L_c}
+
\mu
\frac{U_c}{\rho_c  h_c}
\frac{1}{\delta^2}
+
\cancelto{\approx0}{
k
\frac{1}{\rho_c U_c h_c}
\frac{ T_c}{L_c^2}
}
+
k
\frac{1}{\rho_c U_c h_c}
\frac{ T_c}{\delta_c^2}
+
\frac{D_i}{\delta_c^2h_c}
+...
\end{equation} 
%%
%
\\
Then the Energy conservation gets:
\\
\begin{equation} 
\frac{\partial}{\partial x}(\rho u h)
+
\frac{\partial}{\partial y}(\rho v h)
=
u_i\frac{\partial p}{\partial x_i}
+
\mu
\left(\frac{\partial u}{\partial y}\right)^2
+
\frac{\partial }{\partial y}
\left(
k
\frac{\partial T}{\partial y}
\right)
-
\frac{\partial}{\partial y}
\left(
\sum_i\rho D_i\frac{\partial Y_i}{\partial y}
\right)
+
Q\Omega
\end{equation} 
\\
Using the mass conservation equation:
%
\begin{equation} 
h
\cancelto{\approx0}{
\left(
\frac{\partial}{\partial x}(\rho u )
+
\frac{\partial}{\partial y}(\rho v )
\right)
}
+
\rho u
\frac{\partial h}{\partial x}
+
\rho v
\frac{\partial h}{\partial y}
=
u_i\frac{\partial p}{\partial x_i}
+
\mu
\left(\frac{\partial u}{\partial y}\right)^2
+
\frac{\partial }{\partial y}
\left(
k
\frac{\partial T}{\partial y}
\right)
-
\sum_i\rho D_i\frac{\partial Y_i}{\partial y}
+
Q\Omega
\end{equation} 
%
\begin{equation} 
\rho u
\frac{\partial h}{\partial x}
+
\rho v
\frac{\partial h}{\partial y}
=
u_i\frac{\partial p}{\partial x_i}
+
\mu
\left(\frac{\partial u}{\partial y}\right)^2
+
\frac{\partial }{\partial y}
\left(
k
\frac{\partial T}{\partial y}
\right)
-
\frac{\partial }{\partial y}
\left(
\sum_i\rho D_i\frac{\partial Y_i}{\partial y}
\right)
+
Q\Omega
\end{equation} 
\\
The equations that govern the Steady Reacting Mixing Layer are: 
\\
\\
\begin{equation} 
\rho u\frac{\partial Y^{o}}{\partial x}+
\rho v\frac{\partial Y^{o}}{\partial y}
=
\frac{\partial}{\partial y}\left(\rho D^o\frac{\partial Y^o}{\partial y}\right)+
s^{o}\Omega
%\frac{\partial\rho D^m \tfrac{\partial\rho Y^m u_i}{\partial x_i}u_i}{\partial x_i}=
\end{equation} 
\\
\begin{equation} 
\rho u\frac{\partial Y^{F}}{\partial x}+
\rho v\frac{\partial Y^{F}}{\partial y}
=
\frac{\partial}{\partial y}\left(\rho D^F\frac{\partial Y^F}{\partial y}\right)+
s^{F}\Omega
%\frac{\partial\rho D^m \tfrac{\partial\rho Y^m u_i}{\partial x_i}u_i}{\partial x_i}=
\end{equation} 
\\
\\
\begin{equation} 
\rho u\frac{\partial u}{\partial x}+
\rho v\frac{\partial u}{\partial y}=-
\frac{\partial p}{\partial x}+
\frac{\partial}{\partial y}\left(\mu\frac{\partial u}{\partial y}\right)+
\rho f_x
\end{equation} 
\\
\\
\begin{equation} 
\frac{\partial p}{\partial y}=
\rho f_y.
\end{equation} 
\\
\\
\begin{equation}
%\frac{\partial}{\partial t}(\rho h)+
\rho u \frac{\partial h}{\partial x}+
\rho v \frac{\partial h}{\partial y}=
%\frac{\partial p}{\partial t}+
u\frac{\partial p}{\partial x}+
v\frac{\partial p}{\partial y}+
%\frac{\partial u_j}{\partial x_i}\tau_{ij}-
\mu\frac{\partial^2 u}{\partial y^2}+
\frac{\partial}{\partial y}
\left(
   k\frac{\partial T}{\partial y}\right)
-
\frac{\partial}{\partial y}
\left(
\sum_i\rho D_i\frac{\partial Y_i}{\partial y}
\right)
+
Q\Omega
\end{equation} 
\\
\\
The above equations can be more simplified if we omit the body force
in all direction and  with a zero pressure gradient,  became: 
\\
\\
\begin{equation} 
\rho u\frac{\partial Y^{o}}{\partial x}+
\rho v\frac{\partial Y^{o}}{\partial y}
=
\frac{\partial}{\partial y}\left(\rho D^o\frac{\partial Y^o}{\partial y}\right)+
s^{o}\Omega
%\frac{\partial\rho D^m \tfrac{\partial\rho Y^m u_i}{\partial x_i}u_i}{\partial x_i}=
\end{equation} 
\\
\\
\begin{equation} 
\rho u\frac{\partial Y^{F}}{\partial x}+
\rho v\frac{\partial Y^{F}}{\partial y}
=
\frac{\partial}{\partial y}\left(\rho D^F\frac{\partial Y^F}{\partial y}\right)+
s^{F}\Omega
%\frac{\partial\rho D^m \tfrac{\partial\rho Y^m u_i}{\partial x_i}u_i}{\partial x_i}=
\end{equation} 
\\
\\
\begin{equation} 
\rho u\frac{\partial u}{\partial x}+
\rho v\frac{\partial u}{\partial y}=
%-\frac{\partial p}{\partial x}+
\frac{\partial}{\partial y}\left(\mu\frac{\partial u}{\partial y}\right)
%\rho f_x
\end{equation} 
%\\
%\\
%\begin{equation} 
%\frac{\partial p}{\partial y}=0
%\end{equation} 
\\
\\
\begin{equation}
%\frac{\partial}{\partial t}(\rho h)+
\rho u \frac{\partial h}{\partial x}+
\rho v \frac{\partial h}{\partial y}=
%\frac{\partial p}{\partial t}+
%u\frac{\partial p}{\partial x}+
%v\frac{\partial p}{\partial y}+
%\frac{\partial u_j}{\partial x_i}\tau_{ij}-
\mu\left(\frac{\partial u}{\partial y}\right)^2+
\frac{\partial}{\partial y}
\left(
   k\frac{\partial T}{\partial y}
   \right)
-
\frac{\partial}{\partial y}
\left(
\sum_i\rho D_i\frac{\partial Y_i}{\partial y}
\right)
+
Q\Omega
\end{equation} 
\\
\\
\begin{equation}
p=\rho R T
\end{equation} 
\\
\begin{equation}
h=c_pT
\end{equation} 
\\
\begin{equation}
\Omega=\beta Y_{F}Y_{O}\exp\left(\frac{E}{RT}\right)
\end{equation} 

\clearpage 

\textbf{Boundary conditions}

\textbf{x $\longrightarrow +\infty$}
\\
$u=u_{+\infty}$
\\
$v=v_{+\infty}=0$
\\
$p=p_{+\infty}=constant=C$
\\
$\rho=\rho_{+\infty}=0$
\\
$Y_F=Y_{F+\infty}=1$
\\
$Y_O=Y_{O+\infty}=0$
\\

\textbf{x $\longrightarrow -\infty$}
\\
$u=u_{-\infty}$
\\
$v=v_{-\infty}=0$
\\
$p=p_{-\infty}=constant=C$
\\
$\rho=\rho_{-\infty}=0$
\\
$Y_F=Y_{F+\infty}=0$
\\
$Y_O=Y_{O+\infty}=1$
 


