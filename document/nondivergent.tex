%\section{Self-Similarity solution for Reacting Mixing Layer equations}
%\\
To find a similarity solutions for a reacting mixing layer equations, firstly we 
need to use the Howarth-Dorodnitsyn transformation to find am 
incompressible form of the compressibles equations. 


Definig the stream function for compressible flow as: 
\\
\\
\begin{equation}
\frac{\partial \Psi}{\partial \overline{x}}=-\varrho V \qquad
\frac{\partial \Psi}{\partial \overline{y}}=\varrho U,
\end{equation} 
\\
\\
The mass conservation equation is identically satisfied by 
the stream function.
%
%
\begin{equation}
\frac{\partial{(\varrho U)}}{\partial \overline{x}}
+
\frac{\partial{(\varrho V)}}{\partial \overline{y}}
=0
\end{equation}

\begin{equation}
-\frac{\partial^2 \Psi }{\partial  \overline{x}\partial \overline{y}}
+
\frac{\partial^2 \Psi }{\partial \overline{x}\partial \overline{y}}
=0
\end{equation}

Now defining a vertical stretching as:

\begin{equation}
x=\overline{x}\qquad
y=\int_0^{\overline{y}}\varrho d\overline{y}%\qquad 
%\hat{V}=\varrho V+U\int_o^{\overline{y}}\varrho_{\overline{x}}\qquad
\label{h-d}
\end{equation} 
\\
\\
and using the chain rule:
\\
\\

\begin{equation}
f
=
f(x,y)=
f(x(\overline{x},\overline{y}),
y(\overline{x},\overline{y}))
\end{equation} 
\\
\begin{equation}
\frac{\partial f}{\partial \overline{x}}
=
\frac{\partial f}{\partial x}
\frac{\partial x}{\partial \overline{x}}
+
\frac{\partial f}{\partial y}
\frac{\partial y}{\partial \overline{x}}
\end{equation} 


\begin{equation}
\frac{\partial f}{\partial \overline{y}}=
\frac{\partial f}{\partial x}
\frac{\partial x}{\partial \overline{y}}+
\frac{\partial f}{\partial y}
\frac{\partial y}{\partial \overline{y}}
\end{equation} 

Then,

\begin{equation}
\frac{\partial f}{\partial \overline{x}}=
\frac{\partial f}{\partial x}+
%\frac{\partial x}{\partial \overline{x}}+
\frac{\partial f}{\partial y}
\int_0^{\overline{y}}\frac{\partial \varrho }{\partial \overline{x}}dy
%=
%\frac{\partial f}{\partial x}+
%\frac{\partial f}{\partial y}
%\frac{\partial }{\partial \overline{x}}
%\int_0^{\overline{y}} \varrho dy
%=
%\frac{\partial f}{\partial x}
%+
%\frac{\partial f}{\partial y}
%\frac{\partial y}{\partial \overline{x}}
,
\label{fx}
\end{equation} 

and 

\begin{equation}
\frac{\partial f}{\partial \overline{y}}=
%\frac{\partial f}{\partial x}
%\frac{\partial x}{\partial \overline{y}}+
\frac{\partial f}{\partial y}
%\frac{\partial y}{\partial \overline{y}}
\varrho
,
\label{fy}
\end{equation} 
\\
\\
so that, using the stream function definition, we can defined a new velocity $\hat{U}$ as:
\\
\\
\begin{equation}
\frac{\partial \Psi}{\partial \overline{y}}
=
\varrho U
\end{equation} 

\begin{equation}
\varrho
\frac{\partial \Psi}{\partial y}
=
\varrho
U
\end{equation} 

\begin{equation}
\frac{\partial \Psi}{\partial y}
=
U
=
\hat{U}
\end{equation} 
\\
that have the same form as incompresible flow. 
In the same way we can defined:
\\
\\
\begin{equation}
\hat{V}=-\frac{\partial \Psi}{\partial x},
\end{equation} 
\\
\\
Using the definition of $V$:
\\
\\
\begin{equation}
\varrho V=-
\frac{\partial \Psi}{\partial \overline{x}}
\end{equation} 
\\
\\
with  the chain rule \ref{fx} we can write: 
\\
\\
\begin{equation}
  \varrho V=-
  \left(
       \frac{\partial \Psi}{\partial x}+
       \frac{\partial \Psi}{\partial y}
       \frac{\partial y}{\partial \overline{x}}
  \right)
\end{equation} 
\\
\\
\begin{equation}
  \varrho V=
  \left(
       \hat{V}
       -
       \hat{U}
       \frac{\partial y}{\partial \overline{x}}
  \right)
\end{equation} 
\\
\\
\begin{equation}
  \hat{V}=
  \varrho V
  +\hat{U}
  \frac{\partial y}{\partial \overline{x}}
\label {hatv}
\end{equation} 
\\
\\
This transformation implies a physical areal equals mass-weighted area. 
\\
\\
Defining, a mass element is 
\\
\\
$dm=\varrho d\overline{x} d\overline{y}$, 
\\
\\
but 
\\
\\
$dx=d\overline{x}$ and 
\\
\\
$dy=\varrho d\overline{y}$,
\\
\\
then 
\\
\\
$dm=dx dy$.
\\
\\
The vertical coordinate is a mass coordinate ($dy=\varrho$d$\overline{y}$, mass per unit horizontal length)
and density is absorbed  in the geometry of the transformation. 
\\
\\
Matemathically:  by the change of variable theorem:
\\
\\
$dxdy=detJd\overline{x}d\overline{y}=\varrho d\overline{x}d\overline{y}$
\\
\\
with $J$ the Jacobian of the transformation:
\\
\\
\begin{equation}
J =
\begin{pmatrix}
\dfrac{\partial x}{\partial \overline{x}} &
\dfrac{\partial x}{\partial \overline{y}} \\[8pt]
\dfrac{\partial y}{\partial \overline{x}} &
\dfrac{\partial y}{\partial \overline{y}}
\end{pmatrix}
=
\begin{pmatrix}
1 & 0 \\[8pt]
\displaystyle \int_{0}^{\overline{y}}
\frac{\partial \varrho}{\partial \overline{x}}(\overline{x},\eta)\, d\eta
&
\varrho(\overline{x},\overline{y})
\end{pmatrix}.
\end{equation}
\\\\
and 
\\\\
\begin{equation}
\det J = \varrho(\overline{x},\overline{y}).
\end{equation}
\\\\
$\overline{y}$ labels equal mass layers. 
\\\\
and mass consevation in ($\overline{x}\overline{y}$) 
implies geometric conservation in the new coordinates ($x,y$), 
but the volume is not preserved. 
\\\\
$dV=\frac{1}{\varrho}dxdy= \frac{1}{\varrho}\text{mass-area}$
\\\\
High density  implies 
small physical volume per mass cell and consequently, 
Low density 
large physical volume per mass cell.
So: Equal mass cells occupy different physical volumes.
This is exactly where compressibility lives.
\\\\
Now we can used the equation \ref{h-d}, \ref{fx}, \ref{fy} and \ref{hatv} in the 
Mixing layer equations, \ref{mass2} to  
\ref{g2}, obtain:
\\
\\
\textbf{Mass conservation}
\\
\\
\begin{equation} 
\frac{\partial}{\partial \overline{x}}(\varrho U)+
\frac{\partial}{\partial \overline{y}}(\varrho V)=0
\end{equation} 
\\
\\
\begin{equation} 
\frac{\partial}{\partial x}(\varrho U)+
\frac{\partial}{\partial y}(\varrho U)
\frac{\partial y}{\partial \overline{x}}+
\varrho\frac{\partial}{\partial y}(\varrho V)=0
\end{equation} 
\\
\\
\begin{equation} 
\frac{\partial}{\partial x}(\varrho U)+
\frac{\partial}{\partial y}(\varrho U)
\frac{\partial y}{\partial \overline{x}}+
\varrho\frac{\partial}{\partial y}
  \left(
       \hat{V}
       -
       \hat{U}
       \frac{\partial y}{\partial \overline{x}}
  \right)
=0
\end{equation} 
\\
\\
\begin{equation} 
U\frac{\partial \varrho }{\partial x}+
\varrho\frac{\partial U}{\partial x}+
U\frac{\partial \varrho }{\partial y}
\frac{\partial y}{\partial \overline{x}}+
\varrho\frac{\partial U}{\partial y}
\frac{\partial y}{\partial \overline{x}}+
\varrho\frac{\partial \hat{V}}{\partial y}
-\varrho\frac{\partial }{\partial \overline{y}}
       \left(
       \hat{U}\frac{\partial y}{\partial \overline{x}}
       \right)
=0
\end{equation} 
\\
\\
\begin{equation} 
U\frac{\partial \varrho }{\partial x}+
\varrho\frac{\partial U}{\partial x}+
U\frac{\partial \varrho }{\partial y}
\frac{\partial y}{\partial \overline{x}}+
\varrho\frac{\partial U}{\partial y}
\frac{\partial y}{\partial \overline{x}}+
\varrho\frac{\partial \hat{V}}{\partial \overline{y}}
-\varrho\frac{\partial \hat{U}}{\partial \overline{y}}
       \frac{\partial y}{\partial \overline{x}}
-\varrho \hat{U}\frac{\partial }{\partial y}
       \left(
       \frac{\partial y}{\partial \overline{x}}
       \right)
=0
\end{equation} 
\\
As $U=\hat{U}$, the fouth and the sixth terms of the above equation are canceling each other, hence:
\\
\begin{equation} 
\hat{U}\frac{\partial \varrho }{\partial x}+
\varrho\frac{\partial \hat{U}}{\partial x}+
\hat{U}\frac{\partial \varrho }{\partial y}
\frac{\partial y}{\partial \overline{x}}+
\cancel{
\varrho\frac{\partial \hat{U}}{\partial y}
\frac{\partial y}{\partial \overline{x}}}+
\varrho\frac{\partial \hat{V}}{\partial y}
-\cancel{
         \varrho\frac{\partial \hat{U}}{\partial \overline{y}}
         \frac{\partial y}{\partial \overline{x}}
        }
-\varrho \hat{U}\frac{\partial }{\partial y}
       \left(
       \frac{\partial y}{\partial \overline{x}}
       \right)
=0
\end{equation} 
\\
\\
\begin{equation} 
\hat{U}\frac{\partial \varrho }{\partial x}+
\varrho\frac{\partial \hat{U}}{\partial x}+
\hat{U}\frac{\partial \varrho }{\partial y}
\frac{\partial y}{\partial \overline{x}}+
\varrho\frac{\partial \hat{V}}{\partial y}
-\varrho \hat{U}\frac{\partial }{\partial y}
       \left(
       \frac{\partial y}{\partial \overline{x}}
       \right)
=0
\end{equation} 
\\
\\
the last term of the above equation  can be write as:
\\
\\
\begin{equation} 
\varrho \hat{U}\frac{\partial }{\partial y}
       \left(
       \frac{\partial y}{\partial \overline{x}}
       \right)
=
\hat{U}\varrho \frac{\partial }{\partial y}
\left(
      \frac{\partial y}{\partial \overline{x}}
\right)
=
\hat{U}\frac{\partial }{\partial \overline{y}}
\left(
      \frac{\partial y}{\partial \overline{x}}
\right)
=
\hat{U}\frac{\partial \varrho}{\partial \overline{x}}
=
\hat{U}
\left(
\frac{\partial \varrho}{\partial x}+
\frac{\partial \varrho}{\partial y}
\frac{\partial y}{\partial \overline{x}}
\right)
\end{equation} 
\\
\\
hence:
\\
\\
\begin{equation} 
\varrho\frac{\partial \hat{U}}{\partial x}+
\varrho\frac{\partial \hat{V}}{\partial y}
=0
\end{equation} 
\\
\\
\begin{equation} 
\frac{\partial \hat{U}}{\partial x}+
\frac{\partial \hat{V}}{\partial y}
=0
\end{equation} 
\\
\\
\textbf{Species Conservation}
\\
\\
Oxidizer conservation
\\
\\
\begin{equation} 
\varrho U\frac{\partial \psi_{o}}{\partial\overline{x}}+
\varrho V\frac{\partial \psi_{o}}{\partial\overline{y}}=
\frac{1}{PrLe_o}\frac{\partial}{\partial \overline{y}}\left(\rho D^o\frac{\partial \psi_o}{\partial \overline{y}}\right)+
s^{o}\Omega
%\frac{\partial\rho D^m \tfrac{\partial\rho Y^m u_i}{\partial x_i}u_i}{\partial x_i}=
\end{equation} 
\\
\\
\begin{equation} 
\varrho U\frac{\partial \psi_{o}}{\partial x}+
\varrho U\frac{\partial \psi_{o}}{\partial y}
\frac{\partial y}{\partial \overline{x}}+
\varrho^2 V\frac{\partial \psi_{o}}{\partial y}=
\frac{\varrho}{PrLe_o}\frac{\partial}{\partial y}\left(\varrho^2 D^o\frac{\partial \psi_o}{\partial y}\right)+
s^{o}\Omega
%\frac{\partial\rho D^m \tfrac{\partial\rho Y^m u_i}{\partial x_i}u_i}{\partial x_i}=
\end{equation} 
\\
\\
\begin{equation} 
\varrho U\frac{\partial \psi_{o}}{\partial x}+
\varrho U\frac{\partial \psi_{o}}{\partial y}
\frac{\partial y}{\partial \overline{x}}+
\varrho^2 \frac{\partial \psi_{o}}{\partial y}
\left(
       \frac{1}{\varrho}\hat{V}-
       \frac{1}{\varrho}U\frac{\partial y}{\partial \overline{x}}
\right)
=
\frac{\varrho}{PrLe_o}\frac{\partial}{\partial y}\left(\varrho^2 D^o\frac{\partial \psi_o}{\partial y}\right)+
s^{o}\Omega
%\frac{\partial\rho D^m \tfrac{\partial\rho Y^m u_i}{\partial x_i}u_i}{\partial x_i}=
\end{equation} 
\\
\\
\begin{equation} 
\varrho U\frac{\partial \psi_{o}}{\partial x}+
\varrho U\frac{\partial \psi_{o}}{\partial y}
\frac{\partial y}{\partial \overline{x}}+
\varrho^2 \frac{\partial \psi_{o}}{\partial y}
\left(
       \frac{1}{\varrho}\hat{V}-
       \frac{1}{\varrho}U\frac{\partial y}{\partial \overline{x}}
\right)
=
\frac{\varrho}{PrLe_o}\frac{\partial}{\partial y}\left(\varrho^2 D^o\frac{\partial \psi_o}{\partial y}\right)+
s^{o}\Omega
%\frac{\partial\rho D^m \tfrac{\partial\rho Y^m u_i}{\partial x_i}u_i}{\partial x_i}=
\end{equation} 
\\
\\
\begin{equation} 
\varrho U\frac{\partial \psi_{o}}{\partial x}+
\varrho \hat{V}\frac{\partial \psi_{o}}{\partial y}
=
\frac{\varrho}{PrLe_o}\frac{\partial}{\partial y}\left(\varrho^2 D^o\frac{\partial \psi_o}{\partial y}\right)+
s^{o}\Omega
%\frac{\partial\rho D^m \tfrac{\partial\rho Y^m u_i}{\partial x_i}u_i}{\partial x_i}=
\end{equation} 
\\
\\
\begin{equation} 
 \hat{U}\frac{\partial \psi_{o}}{\partial x}+
 \hat{V}\frac{\partial \psi_{o}}{\partial y}
=
\frac{1}{PrLe_o}\frac{\partial}{\partial y}\left(\varrho^2 D^o\frac{\partial \psi_o}{\partial y}\right)+
\frac{s^{o}\Omega}{\varrho}
%\frac{\partial\rho D^m \tfrac{\partial\rho Y^m u_i}{\partial x_i}u_i}{\partial x_i}=
\end{equation} 
\\
\\
Fuel conservation
\\
\\
In the same way as in oxidizer conservartion, using the transform coordinates $x, y, \hat{U}$ and $\hat{V}$ in the Fuel
conservations species, we get: 
\\
\\
\begin{equation} 
 \hat{U}\frac{\partial \psi_{F}}{\partial x}+
 \hat{V}\frac{\partial \psi_{F}}{\partial y}
=
\frac{1}{PrLe_F}\frac{\partial}{\partial y}\left(\varrho^2 D^F\frac{\partial \psi_F}{\partial y}\right)+
\frac{s^{F}\Omega}{\varrho}
%\frac{\partial\rho D^m \tfrac{\partial\rho Y^m u_i}{\partial x_i}u_i}{\partial x_i}=
\end{equation} 
\\
\\
\textbf{Momemtum conservation}
\\
\\
\begin{equation} 
\varrho U\frac{\partial U}{\partial \overline{x}}+
\varrho V\frac{\partial U}{\partial \overline{y}}=
\frac{\partial}{\partial \overline{y}}\left(\mu\frac{\partial U}{\partial  \overline{y}}\right)
\end{equation} 
\\
\\
\begin{equation} 
\varrho U\frac{\partial U}{\partial x}+
\varrho U\frac{\partial U}{\partial y}
\frac{\partial y}{\partial \overline{x}}+
\varrho^2 V\frac{\partial U}{\partial y}=
\varrho\frac{\partial}{\partial y}\left(\mu\varrho\frac{\partial U}{\partial  y}\right)
\end{equation} 
\\
\\
\begin{equation} 
\varrho U\frac{\partial U}{\partial x}+
\varrho U\frac{\partial U}{\partial y}
\frac{\partial y}{\partial \overline{x}}+
\varrho^2 
\left(
       \frac{1}{\varrho}\hat{V}-
       \frac{1}{\varrho}U\frac{\partial y}{\partial \overline{x}}
\right)
\frac{\partial U}{\partial y}=
\varrho\frac{\partial}{\partial y}\left(\mu\varrho\frac{\partial U}{\partial  y}\right)
\end{equation} 
\\
\\
\begin{equation} 
\varrho U\frac{\partial U}{\partial x}+
\varrho \hat{V}\frac{\partial U}{\partial y}=
\varrho\frac{\partial}{\partial y}\left(\mu\varrho\frac{\partial U}{\partial  y}\right)
\end{equation} 
\\
\\
\begin{equation} 
 \hat{U}\frac{\partial \hat{U}}{\partial x}+
 \hat{V}\frac{\partial \hat{U}}{\partial y}=
\frac{\partial}{\partial y}\left(\mu\varrho\frac{\partial \hat{U}}{\partial  y}\right)
\end{equation} 
\\
\\
\textbf{Energy Equation}
\\
\\
\begin{equation}
\varrho U\frac{\partial }{\partial \overline{x}}(c_p \overline{T})+
\varrho V\frac{\partial }{\partial \overline{y}}(c_p \overline{T})=
(\gamma-1)M^2\mu
\left(
\frac{\partial U}{\partial \overline{y}}
\right)^2
+
\frac{1}{Pr}\frac{\partial}{\partial \overline{y}}\left(k\frac{\partial \overline{T}}{\partial \overline{y}}\right)
-
\frac{Y_{i\infty}}{Pr}
\frac{\partial }{\partial \overline{y}}
\left(
\sum_i\frac{\varrho}{Le_i}\overline{h}_i D_i\frac{\partial \psi_i}{\partial \overline{y}}
\right)
+
Q\Omega
\end{equation} 

{\color{red}Does not necessary because is not a second order derivative in the viscous term.}
Using the chain rule to change the second order derivative of $\overline{y}$ in the energy equation. 

\begin{equation}
\frac{\partial^2f}{\partial \overline{y}^2}=
\frac{\partial}{\partial y}
\left(
\frac{\partial f}{\partial y}
\frac{\partial y}{\partial \overline{y}}
\right)
\frac{\partial y}{\partial \overline{y}}
\end{equation} 

\begin{equation}
\frac{\partial^2f}{\partial \overline{y}^2}
=
\frac{\partial ^2 f}{\partial y^2}
\left(
\frac{\partial y}{\partial \overline{y}}
\right)^2+
\frac{\partial f}{\partial y}
\frac{\partial ^2 y}{\partial \overline{y}^2}
\end{equation} 



\begin{equation}
\frac{\partial^2f}{\partial \overline{y}^2}=
\frac{\partial ^2 f}{\partial y^2}
\left(
\frac{\partial y}{\partial \overline{y}}
\right)^2
+
\frac{\partial f}{\partial y}
\frac{\partial  }{\partial \overline{y}}
\left(
\frac{\partial y}{\partial \overline{y}}
\right)
\end{equation} 

$
\dfrac{\partial  }{\partial \overline{y}}
=\varrho
\dfrac{\partial  }{\partial y}
$

\begin{equation}
\frac{\partial^2f}{\partial \overline{y}^2}=
\frac{\partial ^2 f}{\partial y^2}
\varrho^2+
\frac{\partial f}{\partial y}
\varrho
\frac{\partial \varrho}{\partial y}
\end{equation} 


{\color{red}correction the second derivative that is no present}

\begin{equation}
\begin{split}
\varrho U\frac{\partial }{\partial x}(c_p \overline{T})+
\varrho U\frac{\partial }{\partial y}(c_p \overline{T})
\frac{\partial y}{\partial \overline{x}}+
\varrho^2 V
\frac{\partial }{\partial y}(c_p \overline{T})=
\\
\varrho^2(\gamma-1)M^2\mu
\left(
\frac{\partial U}{\partial y}
\right)^2
+
\frac{\varrho}{Pr}\frac{\partial}{\partial y}\left(k\varrho\frac{\partial \overline{T}}{\partial y}\right)
-
\frac{Y_{i\infty}\varrho}{Pr}
\frac{\partial }{\partial y}
\left(
\sum_i\frac{\varrho^2}{Le_i}\overline{h}_i D_i\frac{\partial \psi_i}{\partial y}
\right)
+
Q\Omega
\end{split}
\end{equation} 
\\
\\
\begin{equation}
\begin{split}
\varrho U\frac{\partial }{\partial x}(c_p \overline{T})+
\varrho U\frac{\partial }{\partial y}(c_p \overline{T})
\frac{\partial y}{\partial \overline{x}}+
\varrho^2 
\left(
       \frac{1}{\varrho}\hat{V}-
       \frac{1}{\varrho}U\frac{\partial y}{\partial \overline{x}}
\right)
\frac{\partial }{\partial y}(c_p \overline{T})=
\\
\varrho^2(\gamma-1)M^2\mu
\left(
\frac{\partial U}{\partial \overline{y}}
\right)^2
+
\frac{\varrho}{Pr}\frac{\partial}{\partial y}\left(k\varrho\frac{\partial \overline{T}}{\partial y}\right)
-
\frac{Y_{i\infty}\varrho}{Pr}
\frac{\partial }{\partial y}
\left(
\sum_i\frac{\varrho^2}{Le_i}\overline{h}_i D_i\frac{\partial \psi_i}{\partial y}
\right)
+
Q\Omega
\end{split}
\end{equation} 
\\
\\
\begin{equation}
\begin{split}
\hat{U}\frac{\partial }{\partial x}(c_p \overline{T})+
\hat{V}
\frac{\partial }{\partial y}(c_p \overline{T})=
\varrho(\gamma-1)M^2\mu
\left(
\frac{\partial U}{\partial \overline{y}}
\right)^2
+
\\
\frac{1}{Pr}\frac{\partial}{\partial y}\left(k\varrho\frac{\partial \overline{T}}{\partial y}\right)
-
\frac{Y_{i\infty}}{Pr}
\frac{\partial }{\partial y}
\left(
\sum_i\frac{\varrho^2}{Le_i}\overline{h}_i D_i\frac{\partial \psi_i}{\partial y}
\right)
+
\frac{Q\Omega}{\varrho}
\end{split}
\end{equation} 
\\
\\
\begin{equation}
\hat{U}\frac{\partial }{\partial x}(c_p \overline{T})+
\hat{V}
\frac{\partial }{\partial y}(c_p \overline{T})=
\varrho
(\gamma-1)M^2\mu
\left(
\frac{\partial U}{\partial \overline{y}}
\right)^2
+
\frac{1}{Pr}\frac{\partial}{\partial y}\left(k\varrho\frac{\partial \overline{T}}{\partial y}\right)
-
\frac{Y_{i\infty}}{Pr}
\frac{\partial }{\partial y}
\left(
\sum_i\frac{\varrho^2}{Le_i}\overline{h}_i D_i\frac{\partial \psi_i}{\partial y}
\right)
+
\frac{Q\Omega}{\varrho}
\end{equation} 
\\
\\
To transform completely the compressible equations to incompressible form we need to used the
Chapman's approximate viscosity law:
\\
\\
\begin{equation} 
\mu=C_w T
\end{equation} 
\\
\\
Multiply by $\varrho$ and using the gas perfect equation, we get:
\\
\\
\begin{equation} 
\mu\varrho=C_w
\end{equation} 
\\
\\
where $C_w$  is a function of the temperature $C_w=C_w(T)$ . 
\\
\\
\begin{equation} 
\frac{\partial \hat{U}}{\partial x}+
\frac{\partial \hat{V}}{\partial y}
=0
\end{equation} 
\\
\\
\begin{equation} 
 \hat{U}\frac{\partial \psi_{o}}{\partial x}+
 \hat{V}\frac{\partial \psi_{o}}{\partial y}
=
\frac{1}{PrLe_o}\frac{\partial}{\partial y}\left(\varrho^2 D^o\frac{\partial \psi_o}{\partial y}\right)+
\frac{s^{o}\Omega}{\varrho}
%\frac{\partial\rho D^m \tfrac{\partial\rho Y^m u_i}{\partial x_i}u_i}{\partial x_i}=
\end{equation} 
\\
\\
\begin{equation} 
 \hat{U}\frac{\partial \psi_{F}}{\partial x}+
 \hat{V}\frac{\partial \psi_{F}}{\partial y}
=
\frac{1}{PrLe_F}\frac{\partial}{\partial y}\left(\varrho^2 D^F\frac{\partial \psi_F}{\partial y}\right)+
\frac{s^{F}\Omega}{\varrho}
%\frac{\partial\rho D^m \tfrac{\partial\rho Y^m u_i}{\partial x_i}u_i}{\partial x_i}=
\end{equation} 
\\
\\
\begin{equation} 
 \hat{U}\frac{\partial \hat{U}}{\partial x}+
 \hat{V}\frac{\partial \hat{U}}{\partial y}=
\frac{\partial}{\partial y}\left(\mu\varrho\frac{\partial \hat{U}}{\partial  y}\right)
\end{equation} 
\\
\\
\begin{equation}
\hat{U}\frac{\partial }{\partial x}(c_p \overline{T})+
\hat{V}
\frac{\partial }{\partial y}(c_p \overline{T})=
\varrho
(\gamma-1)M^2\mu
\left(
\frac{\partial \hat{U}}{\partial y}
\right)
^2
+
\frac{1}{Pr}\frac{\partial}{\partial y}\left(k\varrho\frac{\partial \overline{T}}{\partial y}\right)
-
\frac{Y_{i\infty}}{Pr}
\frac{\partial }{\partial y}
\left(
\sum_i\frac{\varrho^2}{Le_i}\overline{h}_i D_i\frac{\partial \psi_i}{\partial y}
\right)
+
\frac{Q\Omega}{\varrho}
\end{equation} 
