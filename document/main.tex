%% This is file `elsarticle-template-1-num.tex',
%%
%% Copyright 2009 Elsevier Ltd
%%
%% This file is part of the 'Elsarticle Bundle'.
%% ---------------------------------------------
%%
%% It may be distributed under the conditions of the LaTeX Project Public
%% License, either version 1.2 of this license or (at your option) any
%% later version.  The latest version of this license is in
%%    http://www.latex-project.org/lppl.txt
%% and version 1.2 or later is part of all distributions of LaTeX
%% version 1999/12/01 or later.
%%
%% Template article for Elsevier's document class `elsarticle'
%% with numbered style bibliographic references
%%
%% $Id: elsarticle-template-1-num.tex 149 2009-10-08 05:01:15Z rishi $
%% $URL: http://lenova.river-valley.com/svn/elsbst/trunk/elsarticle-template-1-num.tex $
%%
\documentclass[preprint,12pt]{elsarticle}

%% Use the option review to obtain double line spacing
%% \documentclass[preprint,review,12pt]{elsarticle}

%% Use the options 1p,twocolumn; 3p; 3p,twocolumn; 5p; or 5p,twocolumn
%% for a journal layout:
%% \documentclass[final,1p,times]{elsarticle}
%% \documentclass[final,1p,times,twocolumn]{elsarticle}
%% \documentclass[final,3p,times]{elsarticle}
%% \documentclass[final,3p,times,twocolumn]{elsarticle}
%% \documentclass[final,5p,times]{elsarticle}
%% \documentclass[final,5p,times,twocolumn]{elsarticle}

\usepackage{amsmath}

%% The graphicx package provides the includegraphics command.
\usepackage{graphicx}
%% The amssymb package provides various useful mathematical symbols
\usepackage{amssymb}
%% The amsthm package provides extended theorem environments
%% \usepackage{amsthm}

%% The lineno packages adds line numbers. Start line numbering with
%% \begin{linenumbers}, end it with \end{linenumbers}. Or switch it on
%% for the whole article with \linenumbers after \end{frontmatter}.
\usepackage{lineno}
\usepackage{mathtools,cancel}%cancel in the equations 
\usepackage[usenames,dvipsnames,svgnames,table]{xcolor}
\usepackage{color}%using fond color 
\usepackage[utf8x]{inputenc}%using portugues letter
\usepackage[T1]{fontenc} % Use 8-bit encoding that has 256 glyphs
%\usepackage{fourier} % Use the Adobe Utopia font for the document - comment this line to return to the LaTeX default
\usepackage[english]{babel} % English language/hyphenation
\usepackage{amsmath,amsfonts,amsthm} % Math packages

\usepackage{lipsum} % Used for inserting dummy 'Lorem ipsum' text into the template

\usepackage{sectsty} % Allows customizing section commands
\allsectionsfont{\centering \normalfont\scshape} % Make all sections centered, the default font and small caps

\usepackage{fancyhdr}


%% natbib.sty is loaded by default. However, natbib options can be
%% provided with \biboptions{...} command. Following options are
%% valid:

%%   round  -  round parentheses are used (default)
%%   square -  square brackets are used   [option]
%%   curly  -  curly braces are used      {option}
%%   angle  -  angle brackets are used    <option>
%%   semicolon  -  multiple citations separated by semi-colon
%%   colon  - same as semicolon, an earlier confusion
%%   comma  -  separated by comma
%%   numbers-  selects numerical citations
%%   super  -  numerical citations as superscripts
%%   sort   -  sorts multiple citations according to order in ref. list
%%   sort&compress   -  like sort, but also compresses numerical citations
%%   compress - compresses without sorting
%%
%% \biboptions{comma,round}

% \biboptions{}

\journal{Journal Name}

\begin{document}

\begin{frontmatter}

%% Title, authors and addresses

\title{Reacting Mixing Layer}

%% use the tnoteref command within \title for footnotes;
%% use the tnotetext command for the associated footnote;
%% use the fnref command within \author or \address for footnotes;
%% use the fntext command for the associated footnote;
%% use the corref command within \author for corresponding author footnotes;
%% use the cortext command for the associated footnote;
%% use the ead command for the email address,
%% and the form \ead[url] for the home page:
%%
%% \title{Title\tnoteref{label1}}
%% \tnotetext[label1]{}
%% \author{Name\corref{cor1}\fnref{label2}}
%% \ead{email address}
%% \ead[url]{home page}
%% \fntext[label2]{}
%% \cortext[cor1]{}
%% \address{Address\fnref{label3}}
%% \fntext[label3]{}


%% use optional labels to link authors explicitly to addresses:
%% \author[label1,label2]{<author name>}
%% \address[label1]{<address>}
%% \address[label2]{<address>}

\author{Jhonatan Andrés Aguirre Manco}

\address{Cachoeira Paulista, Brasil}

\begin{abstract}
%% Text of abstract
\end{abstract}

\begin{keyword}
Science \sep Publication \sep Complicated
%% keywords here, in the form: keyword \sep keyword

%% MSC codes here, in the form: \MSC code \sep code
%% or \MSC[2008] code \sep code (2000 is the default)

\end{keyword}

\end{frontmatter}

%%
%% Start line numbering here if you want
%%
\linenumbers


\maketitle % Print the title

\section{Reacting Mixing Layer Equations}
%\section{Reacting Mixing Layer Equations}


\begin{figure}
   \centerline{\includegraphics[width=1\textwidth]{reacting.jpeg}} 
\label{f1}
\caption{Reacting Mixing Layer}
\end{figure}


\subsection{Conservation's Equations}
 
In a compressible, conservative form and in a perfect gas the conservation's
equations are:
\\
\\
%\vspace{5cm}
\textbf{Mass conservation:}

\begin{equation} 
\frac{\partial\rho}{\partial t}+
\frac{\partial}{\partial x_i}(\rho u_i)=0
\quad \text{for} \quad i=1,2,3
,
\label{mass}
\end{equation} 
\\
\\
%\vspace{5cm}
\textbf{Species conservation:}

\begin{equation} 
\frac{\partial}{\partial t}(\rho Y^m)+
\frac{\partial}{\partial x_i}(\rho Y^m(u_i+V^m))=s^m\Omega
\quad \text{for} \quad i=1,2,3
\quad \text{for} \quad m=\text{Oxidizer, Fuel}
%\frac{\partial\rho D^m \tfrac{\partial\rho Y^m u_i}{\partial x_i}u_i}{\partial x_i}=
\end{equation} 
\\
\\
Using a constitutive equation for $V^m$ 

\begin{equation} 
V^m=-\rho D^m\frac{\partial Y^m}{\partial x_i} 
%\vspace{5cm}
\end{equation} 
\\
where $D^m$ is the diffusion coefficient of specie 
respect to the most abundant specie.

{\color{red} who is $\Omega$?}
\\
\\
\textbf{Momemtum conservation:}
\\
\\
\begin{equation} 
\frac{\partial}{\partial t}(\rho u_j)+
\frac{\partial}{\partial x_i}(\rho u_iu_j)=-
\frac{\partial p}{\partial x_j}+
\frac{\partial \tau_{ij}}{\partial x_i}+
\rho f_j
\quad \text{for} \quad i,j=1,2,3
%\frac{\partial\rho D^m \tfrac{\partial\rho Y^m u_i}{\partial x_i}u_i}{\partial x_i}=
\end{equation} 
\\
\\
where 
\\
\\
\begin{equation} 
\tau_{ij}=\lambda \delta_{ij}\frac{\partial u_k}{\partial x_k}+
\mu\left(\frac{\partial u_i}{\partial x_j}+\frac{\partial u_j}{\partial x_i}\right)
\end{equation} 
\\
\\
and using the Stoke's relation: 
\\
\\
\begin{equation} 
\lambda=-\frac{2}{3}\mu
\end{equation} 

\begin{equation} 
\tau_{ij}=
\mu\left(
-\frac{2}{3} \delta_{ij}\frac{\partial u_k}{\partial x_k}+
\left(\frac{\partial u_i}{\partial x_j}+\frac{\partial u_j}{\partial x_i}\right)
\right)
\end{equation} 
\\
\\
%\vspace{5cm}
\textbf{Energy conservation:}
\\
\\
\begin{equation} 
\frac{\partial}{\partial t}(\rho h)+
\frac{\partial}{\partial x_i}(\rho u_i h)=
\frac{\partial p}{\partial t}+
u_i\frac{\partial p}{\partial x_i}+
\frac{\partial u_j}{\partial x_i}\tau_{ij}
+
\frac{\partial}{\partial x_i}\left(k\frac{\partial T}{\partial x_i}\right)
-
\frac{\partial}{\partial x_i}
\sum_i\rho D_i\frac{\partial Y_i}{\partial x_i}
+
Q\Omega
\end{equation} 
\\
where 
$\dfrac{\partial}{\partial x_i}
\sum_i\rho D_i\dfrac{\partial Y_i}{\partial y}$
is the 
enthalpy transported by species diffusion.
\\
\\
\textbf{Perfect gas equation:}
\\
\begin{equation} 
p=\rho T R
\end{equation} 
\\
\begin{equation}
h=c_pT
\end{equation} 
\\
\\
\\
In the above equations $\rho$ is the density, $u_i=(u,v,w)$ 
are the different velocities in the three different dimensions $x$, $y$ and $z$.
$p$ is the pressure of the flow, $T$ is the temperature of the flow, $f_j$ 
is the body force acting on the flow.
$Y^m$ are the different mass fraction for the two different species, oxygen and fuel.
$\mu$, $D^{m}$ and $k$ are the viscosity, diffusion coefficient for the different 
species and the gas thermal conductivity.  $Q$ is the heat release for the chemical reaction and
$\Omega$ which is the mass reaction rate. 
\\
\\
\textbf{Base Flow equations:}
\\
\\
We are interesting in the a two dimensional compressible mixing layer, because the 
tridimensional effect can be neglected for the study of instabilities in the first stages, 
 it states that for {\it the onset of linear instability in parallel shear flows, the
 least stable (i.e., first to become unstable) disturbances are two-dimensiona} \cite{Squire}.
In the incompressible case, the process of transition to turbulence in a mixing layer 
is dictated by: the growth of two dimensional coherent structures and the development
of secondary instability,  the merging of the large-structures and finally the 
breakdown into small-scales three dimensional turbulence \cite{planche1992numerical}.

The mean flow for two dimensional mixing layer
which separates two fluids, the oxidizer and the fuel, 
at different speeds and temperature with zero pressure gradient can be assumed 
that is governed by boundary layer equations.
This mean that gradients for the different properties are in the $y$ direction, in Cartesian 
coordinates.

To applied the boundary layer equations is necessary  to defined a characteristic 
length of the flow. In this problem exist two characteristics lengths, one
can be defined as the thickness  of the mixing layer $\delta$ and other can be 
defined as the distance that the mixing layer needs to grow $L_c$.

Assume that $\delta$  is sufficient small compared  with the $x$ length
where the mixing layer is development, it mean that $\delta/L_c<<1$.  
The mean velocity scale can be approximate, in the $x$ direction is the 
velocity of order of $U_c$,
and $\partial/\partial x$ is of order of $1/L_c$ 
and assuming that $\rho$ is of order of $\rho_c$ then:

\begin{equation}
\frac{\partial}{\partial x}(\rho u)\sim \frac{\rho_cU_c}{L_c}
\end{equation}

Using the conservation of mass equation \ref{mass} in two dimensions $(x,y)$, 
the order of 
\\
\\
\begin{equation}
\frac{\partial}{\partial y}(\rho v)\sim \frac{\rho_cU_c}{L_c}
\end{equation}
\\
\\
We know that in the mixing layer the velocity $v$ is smaller than
$u$, then we can assume that 
\\
\\
\begin{equation}
v\sim\frac{\delta}{L_c}U_c\qquad
\text{ being }\qquad
\frac{\partial}{\partial y}\sim \frac{1}{\delta}.
\end{equation}
\\
\\
With this assumptions the terms in the conservation equations of order: 
\\
\\
\begin{equation}
\frac{\partial^2}{\partial x^2}\sim\frac{1}{L_c^2},
\end{equation}
\\
\\
are smaller that term of order   
\\
\\
\begin{equation}
\frac{\partial^2}{\partial y^2}\sim\frac{1}{\delta^2}.
\end{equation}
\\
\\
Thereby, using the different length scales in the others conservations equations
and assuming steady state to the mean flow we get: 
\\
\\
\textbf{Mass conservation:}
\\
\\
\begin{equation} 
\frac{\partial}{\partial x}(\rho u)=
\frac{\partial}{\partial y}(\rho v)
\label{mass1}
\end{equation} 
\\
\\
%Using the characteristic lenght scales of the problem we obtein:
%\\
%\\
%\begin{equation} 
%\frac{\rho_cU}{}\frac{\partial}{\partial x}(\rho u)=
%\frac{}{}\frac{\partial}{\partial y}(\rho v)
%\label{mass1}
%\end{equation} 
\\
\\
\textbf{Species conservation:}
\\
\\
\begin{equation} 
\frac{\partial}{\partial x}(\rho Y^{o}(u+V^{o}))+
\frac{\partial}{\partial y}(\rho Y^{o}(v+V^{o}))
=s^{o}\Omega
%\frac{\partial\rho D^m \tfrac{\partial\rho Y^m u_i}{\partial x_i}u_i}{\partial x_i}=
\end{equation} 
\\
\\
\begin{equation} 
\frac{\partial}{\partial x}(\rho Y^{F}(u+V^{F}))+
\frac{\partial}{\partial y}(\rho Y^{F}(v+V^{F}))
=s^{F}\Omega
%\frac{\partial\rho D^m \tfrac{\partial\rho Y^m u_i}{\partial x_i}u_i}{\partial x_i}=
\end{equation} 
\\
\\
Using a constitutive equation for $V^m$ 
\\
\\
\begin{equation} 
V^m=-\rho D^m\frac{\partial }{\partial x_i}(ln(Y^m))
\quad \text{for} \quad m=1,2
%\vspace{5cm}
\end{equation} 
\\
\\
Hence
\\
\\
\begin{equation} 
\frac{\partial}{\partial x_i}(\rho Y^{m}V^m)=
-\frac{\partial}{\partial x_i}\left(\rho D^m\frac{\partial Y^m}{\partial x_i}\right)
\quad \text{for} \quad m=1,2
%\vspace{5cm}
\end{equation} 
\\
\\
\begin{equation} 
\frac{\partial}{\partial x_i}(\rho Y^{m}V^m)=
-\frac{\partial}{\partial x}\left(\rho D^m\frac{\partial Y^m}{\partial x}\right)
-\frac{\partial}{\partial y}\left(\rho D^m\frac{\partial Y^m}{\partial y}\right)
\quad \text{for} \quad m=1,2
%\vspace{5cm}
\end{equation} 
\\
\\
Using the characteristic length scales:
\\
\\
\begin{equation} 
\frac{\partial}{\partial x_i}(\rho Y^{m}V^m)=
-\frac{1}{ L_c}\frac{\rho_c D_c^mY_c^m}{L_c}
-\frac{1}{ \delta}\frac{\rho_c D_c^mY_c^m}{\delta}
\quad \text{for} \quad m=1,2
%\vspace{5cm}
\end{equation} 
\\
\\
\begin{equation} 
\frac{\partial}{\partial x_i}(\rho Y^{m}V^m)=
-\frac{\rho_c D_c^mY_c^m}{ L_c^2}
-\frac{\rho_c D_c^mY_c^m}{ \delta^2}
\quad \text{for} \quad m=1,2
%\vspace{5cm}
\end{equation} 
\\
\\
\begin{equation} 
\frac{\partial}{\partial x_i}(\rho Y^{m}V^m)=
\cancelto{\approx~0}{
-\frac{\rho_c D_c^mY_c^m}{ L_c^2}
}
-\frac{\rho_c D_c^mY_c^m}{ \delta^2}
\quad \text{for} \quad m=1,2
%\vspace{5cm}
\end{equation} 
\\
\\
The Species conservation can be wrote as:
\\
\\
\begin{equation} 
\frac{\partial}{\partial x}(\rho Y^{o}u)+
\frac{\partial}{\partial y}(\rho Y^{o}v)=
\frac{\partial}{\partial y}\left(\rho D^o\frac{\partial Y^o}{\partial y}\right)+
s^{o}\Omega
%\frac{\partial\rho D^m \tfrac{\partial\rho Y^m u_i}{\partial x_i}u_i}{\partial x_i}=
\end{equation} 
\\
\begin{equation} 
\frac{\partial}{\partial x}(\rho Y^{F}u)+
\frac{\partial}{\partial y}(\rho Y^{F}v)=
\frac{\partial}{\partial y}\left(\rho D^F\frac{\partial Y^F}{\partial y}\right)+
s^{F}\Omega
%\frac{\partial\rho D^m \tfrac{\partial\rho Y^m u_i}{\partial x_i}u_i}{\partial x_i}=
\end{equation} 
\\
\\ 
Using the mass conservation equation \ref{mass}, we can write 
the Species conservation as:
\\
\\
\begin{equation} 
\rho u\frac{\partial Y^{o}}{\partial x}+
\rho v\frac{\partial Y^{o}}{\partial y}=
\frac{\partial}{\partial y}\left(\rho D^o\frac{\partial Y^o}{\partial y}\right)+
s^{o}\Omega
%\frac{\partial\rho D^m \tfrac{\partial\rho Y^m u_i}{\partial x_i}u_i}{\partial x_i}=
\end{equation} 
\\
\begin{equation} 
\rho u\frac{\partial Y^{F}}{\partial x}+
\rho v\frac{\partial Y^{F}}{\partial y}=
\frac{\partial}{\partial y}\left(\rho D^F\frac{\partial Y^F}{\partial y}\right)+
s^{F}\Omega
%\frac{\partial\rho D^m \tfrac{\partial\rho Y^m u_i}{\partial x_i}u_i}{\partial x_i}=
\end{equation} 
\\
\\ 

%\vspace{5cm}
\textbf{Momemtum conservation:}
\\
\\
\begin{equation} 
%\frac{\partial}{\partial t}(\rho u_j)+
\frac{\partial}{\partial x_i}(\rho u_iu_j)=-
\frac{\partial p}{\partial x_j}+
\frac{\partial \tau_{ij}}{\partial x_i}+
\rho f_j
\quad \text{for} \quad i,j=1,2,3
%\frac{\partial\rho D^m \tfrac{\partial\rho Y^m u_i}{\partial x_i}u_i}{\partial x_i}=
\end{equation} 
\\
\\
\begin{equation} 
\tau_{ij}=
\lambda \delta_{ij}\frac{\partial u_k}{\partial x_k}+
\mu\left(\frac{\partial u_i}{\partial x_j}+\frac{\partial u_j}{\partial x_i}\right)
\end{equation} 
\\
\\
\begin{equation} 
\frac{\partial\tau_{ij}}{\partial x_i}=
\frac{\partial}{\partial x_i}\left(\lambda \delta_{ij}\frac{\partial u_k}{\partial x_k}\right)+
\frac{\partial}{\partial x_i}\left(\mu\left(\frac{\partial u_i}{\partial x_j}
+\frac{\partial u_j}{\partial x_i}\right)\right)
\end{equation} 
\\
\\
\begin{equation} 
\frac{\partial\tau_{ij}}{\partial x_i}=
\frac{\partial}{\partial x_j}\left(\lambda \frac{\partial u_k}{\partial x_k}\right)+
\frac{\partial}{\partial x_i}\left(\mu\frac{\partial u_i}{\partial x_j}\right)+
\frac{\partial}{\partial x_i}\left(\mu\frac{\partial u_j}{\partial x_i}\right)
\end{equation} 
\\
\\
\begin{equation} 
\frac{\partial\tau_{i1}}{\partial x_i}=
\frac{\partial}{\partial x}\left(\lambda\frac{\partial u}{\partial x}\right)+
\frac{\partial}{\partial x}\left(\lambda\frac{\partial v}{\partial x}\right)+
\frac{\partial}{\partial x}\left(\mu\frac{\partial u}{\partial x}\right)+
\frac{\partial}{\partial y}\left(\mu\frac{\partial v}{\partial x}\right)+
\frac{\partial}{\partial x}\left(\mu\frac{\partial u}{\partial x}\right)+
\frac{\partial}{\partial y}\left(\mu\frac{\partial u}{\partial y}\right)
\end{equation} 
\\
\\
\begin{equation} 
\frac{\partial\tau_{i1}}{\partial x_i}=
\frac{1}{L_c^2}(\lambda_c U_c)+
\frac{1}{L_c^3}(\lambda_c \delta U_c)+
\frac{1}{L_c^2}(\mu_c U_c)+
\frac{1}{L_c^2}(\mu_c U_c)+
\frac{1}{L_c^2}(\mu_c U_c)+
\frac{1}{\delta^2}(\mu_c U_c)+
\end{equation} 
\\
\\
\begin{equation} 
\frac{\partial\tau_{i1}}{\partial x_i}=
	\cancelto{\approx0}{\frac{1}{L_c^2}(\lambda_c U_c)}+
\cancelto{\approx0}{\frac{1}{L_c^3}(\lambda_c \delta U_c)}+
\cancelto{\approx0}{\frac{1}{L_c^2}(\mu_c U_c)}+
\cancelto{\approx0}{\frac{1}{L_c^2}(\mu_c U_c)}+
\cancelto{\approx0}{\frac{1}{L_c^2}(\mu_c U_c)}+
\frac{1}{\delta^2}(\mu_c U_c)+
\end{equation} 
\\
\\
\begin{equation} 
\frac{\partial\tau_{i1}}{\partial x_i}=
\frac{\partial}{\partial y}\left(\mu\frac{\partial u}{\partial y}\right)
\end{equation} 
\\
\\
\begin{equation} 
\frac{\partial\tau_{i2}}{\partial x_i}=
\frac{\partial}{\partial y}\left(\lambda\frac{\partial u}{\partial x}\right)+
\frac{\partial}{\partial y}\left(\lambda\frac{\partial v}{\partial y}\right)+
\frac{\partial}{\partial x}\left(\mu\frac{\partial u}{\partial y}\right)+
\frac{\partial}{\partial y}\left(\mu\frac{\partial v}{\partial y}\right)+
\frac{\partial}{\partial x}\left(\mu\frac{\partial v}{\partial x}\right)+
\frac{\partial}{\partial y}\left(\mu\frac{\partial v}{\partial y}\right)
\end{equation} 
\\
\\
Now using the characteristic length of the problem we found:
\\
\\
\begin{equation} 
\frac{\partial\tau_{i2}}{\partial x_i}=
\frac{1}{\delta L_c}(\lambda_c U_c)+
\frac{1}{\delta L_c}(\lambda_c U_c)+
\frac{1}{\delta L_c}(\mu_c U_c)+
\frac{1}{\delta L_c}(\mu_c U_c)+
\frac{1}{L_c^3}(\mu_c \delta U_c)+
\frac{1}{\delta L_c}(\mu_c U_c)
\label{my}
\end{equation} 
\\
Multiply \ref{my} by $\delta^2$ 
\\
\\
\begin{equation} 
\frac{\partial\tau_{i2}}{\partial x_i}=
\left(\frac{\delta}{ L_c}\right)(\lambda_c U_c)+
\left(\frac{\delta}{L_c}\right)(\lambda_c U_c)+
\left(\frac{\delta}{ L_c}\right)(\mu_c U_c)+
\left(\frac{\delta}{L_c}\right)(\mu_c U_c)+
\left(\frac{\delta}{L_c}\right)^3(\mu_c \delta U_c)+
\left(\frac{\delta}{L_c}\right)(\mu_c U_c)
\end{equation} 
\\
\\
\begin{equation} 
\frac{\partial\tau_{i2}}{\partial x_i}=
\cancelto{\approx0}{\left(\frac{\delta}{ L_c}\right)(\lambda_c U_c)}+
\cancelto{\approx0}{\left(\frac{\delta}{L_c}\right)(\lambda_c U_c)}+
\cancelto{\approx0}{\left(\frac{\delta}{ L_c}\right)(\mu_c U_c)}+
\cancelto{\approx0}{\left(\frac{\delta}{L_c}\right)(\mu_c U_c)}+
\cancelto{\approx0}{\left(\frac{\delta}{L_c}\right)^3(\mu_c \delta U_c)}+
\cancelto{\approx0}{\left(\frac{\delta}{L_c}\right)(\mu_c U_c)}
\end{equation} 
\\
\\
\begin{equation} 
\frac{\partial\tau_{i2}}{\partial x_i}\approx0
\end{equation} 
\\
\\
Now Momentum conservation equation in $x$ direction can be wrote:
\\
\\
\begin{equation} 
\frac{\partial}{\partial x}(\rho uu)+
\frac{\partial}{\partial y}(\rho vu)=-
\frac{\partial p}{\partial x}+
\frac{\partial}{\partial y}\left(\mu\frac{\partial u}{\partial y}\right)+
\rho f_x
\end{equation} 
\\
\\
Using the characteristic length of the problem, we can see that 
the others terms are important:
\\
\\
\begin{equation} 
\frac{1}{L_c}(\rho_c U_c^2)+
\frac{1}{L_c}(\rho_c U_c^2)=
\frac{p_c}{L_c}+
\frac{1}{\delta^2}(\mu_c U_c)+
\rho_c f_{xc}.
\label{mx}
\end{equation} 
%
In the viscous term multiple by $\dfrac{L_c}{U_c^2\rho_c}$:
%
\begin{equation} 
\frac{L_c}{U_c^2\rho_c}
\frac{1}{\delta^2}(\mu_c U_c)=
\frac{\mu_c }{\rho_c\delta}
\frac{L_c}{\delta U_c}=
\frac{\mu_c }{\rho_cV_c\delta}
\end{equation} 
% 
In order to keep the approximation $\delta_c<<L_c$
this term:

\begin{equation} 
\frac{\mu_c }{\rho_cV_c\delta}\approx1=Re
\end{equation} 
%
\\
\\
Using the mass conservation equation \ref{mass1} in \ref{mx}: 
\\
\\
\begin{equation} 
\rho u\frac{\partial u}{\partial x}+
\rho v\frac{\partial u}{\partial y}=-
\frac{\partial p}{\partial x}+
\frac{\partial}{\partial y}\left(\mu\frac{\partial u}{\partial y}\right)+
\rho f_x
\end{equation} 
\\
\\
The Momentum equation in $y$ direction can be wrote as:
\\
\\

\begin{equation} 
\frac{\partial}{\partial x}(\rho uv)+
\frac{\partial}{\partial y}(\rho vv)=-
\frac{\partial p}{\partial y}+
\rho f_y
\end{equation} 
\\
\\
Using the characteristic length of the problem, we can see that 
the others terms are important:
\\
\\
\begin{equation} 
\frac{1}{L_c}\frac{\delta}{L_c}(\rho_c U_c^2)+
\frac{1}{L_c}\frac{\delta}{L_c}(\rho_c U_c^2)=
-\frac{p_c}{L_c}+
\rho_c f_{yc}.
\end{equation} 
\\
\\
\begin{equation} 
\cancelto{\approx0}{\frac{1}{L_c}\frac{\delta}{L_c}(\rho_c U_c^2)}+
\cancelto{\approx0}{\frac{1}{L_c}\frac{\delta}{L_c}(\rho_c U_c^2)}=
-\frac{p_c}{L_c}+
\rho_c f_{yc}.
\end{equation} 
\\
\\
Then the $y$ momentum equation using the boundary layer assumption can be wrote as:
\\
\\
\begin{equation} 
\frac{\partial p}{\partial y}=
\rho f_y.
\end{equation} 
\\
\\
This equation means that the only variations of the pressure in $y$ is determined 
by the body forces in that direction.
%\vspace{5cm}
\\
\\
\textbf{Energy conservation:}
\\
\begin{equation} 
\frac{\partial}{\partial x_i}(\rho u_i h)=
u_i\frac{\partial p}{\partial x_i}+
\frac{\partial u_j}{\partial x_i}\tau_{ij}-
\frac{\partial}{\partial x_i}\left(k\frac{\partial T}{\partial x_i}\right)
-
\frac{\partial}{\partial x_i}
\sum_i\rho D_i\frac{\partial Y_i}{\partial x_i}+
+
Q\Omega
\end{equation} 
\\
\begin{equation} 
\tau_{ij}=
\lambda \delta_{ij}\frac{\partial u_k}{\partial x_k}+
\mu\left(\frac{\partial u_i}{\partial x_j}
+\frac{\partial u_j}{\partial x_i}\right)
\end{equation} 
\\
\\
\begin{equation} 
\frac{\partial u}{\partial x_i}\tau_{i1}=
\frac{\partial u}{\partial x_i}\lambda \delta_{i1}\frac{\partial u_k}{\partial x_k}+
\frac{\partial u}{\partial x_i}\mu\left(\frac{\partial u_i}{\partial x}
+\frac{\partial u}{\partial x_i}\right)
\end{equation} 
\\
\\
\begin{equation} 
\frac{\partial u}{\partial x_i}\tau_{i1}=
\frac{\partial u}{\partial x}\lambda \frac{\partial u_k}{\partial x_k}+
\frac{\partial u}{\partial x_i}\mu\left(\frac{\partial u_i}{\partial x}
+\frac{\partial u}{\partial x_i}\right)
\end{equation} 
\\
\\
\begin{equation} 
\frac{\partial u}{\partial x_i}\tau_{i1}=
\frac{\partial u}{\partial x}
\lambda 
\left(
     \frac{\partial u}{\partial x}+
     \frac{\partial v}{\partial y}
\right)+
\mu\frac{\partial u}{\partial x_i}
\frac{\partial u_i}{\partial x}+
\mu\left(\frac{\partial u}{\partial x_i}\right)^2
\end{equation} 
\\
\\
\begin{equation} 
\frac{\partial u}{\partial x_i}\tau_{i1}=
\lambda 
\left(\frac{\partial u}{\partial x}\right)^2
+
\lambda 
\frac{\partial u }{\partial x}
\frac{\partial v }{\partial y}
+
\mu\frac{\partial u}{\partial x}
\frac{\partial u}{\partial x}+
\mu\frac{\partial u}{\partial y}
\frac{\partial v}{\partial x}+
\mu\left(\frac{\partial u}{\partial x}\right)^2
+
\mu\left(\frac{\partial u}{\partial y}\right)^2
\end{equation} 
\\
\\
\begin{equation} 
\frac{\partial u}{\partial x_i}\tau_{i1}=
\cancelto{\approx0}{
\lambda 
\left(\frac{\partial u}{\partial x}\right)^2
}
+
\cancelto{\approx0}{
\lambda 
\frac{\partial u }{\partial x}
\frac{\partial v }{\partial y}
}
+
\cancelto{\approx0}{
\mu\frac{\partial u}{\partial x}
\frac{\partial u}{\partial x}
}
+
\cancelto{\approx0}{
\mu\frac{\partial u}{\partial y}
\frac{\partial v}{\partial x}
}
+
\cancelto{\approx0}{
\mu\left(\frac{\partial u}{\partial x}\right)^2
}
+
\mu\left(\frac{\partial u}{\partial y}\right)^2
\end{equation} 
\\
\\
\begin{equation} 
\frac{\partial u}{\partial x_i}\tau_{i1}\approx
\mu\left(\frac{\partial u}{\partial y}\right)^2
\end{equation} 
\\
\\
\begin{equation} 
\frac{\partial v}{\partial x_i}\tau_{i2}=
\frac{\partial v}{\partial x_i}\lambda \delta_{i2}\frac{\partial u_k}{\partial x_k}+
\frac{\partial v}{\partial x_i}\mu\left(\frac{\partial u_i}{\partial y}
+\frac{\partial v}{\partial x_i}\right)
\end{equation} 
\\
\\
\begin{equation} 
\frac{\partial v}{\partial x_i}\tau_{i2}=
\lambda\frac{\partial v}{\partial y}
\left(
\frac{\partial u}{\partial x}+
\frac{\partial v}{\partial y}
\right)
+\mu\frac{\partial v}{\partial x_i}
\frac{\partial u_i}{\partial y}+
\left(
\mu\frac{\partial v}{\partial x_i}\right)^2
\end{equation} 
\\
\\
\begin{equation} 
\frac{\partial v}{\partial x_i}\tau_{i2}=
\lambda\frac{\partial v}{\partial y}
\frac{\partial u}{\partial x}+
\lambda
\left(
\frac{\partial v}{\partial y}\right)^2+
\mu\frac{\partial v}{\partial x}
\frac{\partial u}{\partial y}+
\mu
\left(
\frac{\partial v}{\partial y}
\right)^2+
\mu
\left(
\frac{\partial v}{\partial x}
\right)^2+
\mu
\left(
\frac{\partial v}{\partial y}
\right)^2
\end{equation} 
\\
\\
\begin{equation} 
\frac{\partial v}{\partial x_i}\tau_{i2}=
\cancelto{\approx0}{
\lambda\frac{\partial v}{\partial y}
\frac{\partial u}{\partial x}
}
+
\cancelto{\approx0}{
\left(
\frac{\partial v}{\partial y}
\right)^2
}
+
\cancelto{\approx0}{
\mu\frac{\partial v}{\partial x}
\frac{\partial u}{\partial y}
}
+
\cancelto{\approx0}{
\mu
\left(
\frac{\partial v}{\partial y}
\right)^2
}
+
\mu
\cancelto{\approx0}{
\left(
\frac{\partial v}{\partial x}
\right)^2
}
+
\cancelto{\approx0}{
\left(
\frac{\partial v}{\partial y}
\right)^2
}
\end{equation} 
\\
\\
\\
\\
\begin{equation} 
\frac{\partial v}{\partial x_i}\tau_{i2}\approx0
\end{equation} 

Finally for the heat diffusion:

\begin{equation} 
\frac{\partial kT}{\partial x_i}
=
k
\frac{\partial T}{\partial x}+
k
\frac{\partial T}{\partial y}
\end{equation} 
%
Using the characteristic lengths: 
%
\begin{equation} 
\frac{\partial kT}{\partial x_i}
=
k
\frac{ T_c}{L_c}
+
k
\frac{ T_c}{\delta_c}
\end{equation} 
%
\\
\\
Hence, the energy equation for steady flow become:
\\
\begin{equation} 
\frac{\partial}{\partial x_i}(\rho u_i h)=
u_i\frac{\partial p}{\partial x_i}+
\frac{\partial u_j}{\partial x_i}\tau_{ij}
+
\frac{\partial}{\partial x_i}\left(k\frac{\partial T}{\partial x_i}\right)
-
\frac{\partial }{\partial x_i}
\left(
\sum_i\rho D_i\frac{\partial Y_i}{\partial x_i}+
\right)
+
Q\Omega
\end{equation} 
\\
\begin{equation} 
\frac{\rho_c U_c h_c}{L_c}
+
\frac{\rho_c U_c\delta_c h_c}{L_c\delta_c}
=
U_c\frac{p_c}{L_c}
+
U_c\frac{p_c}{L_c}
+
\frac{U_c\delta_c}{L_c}
\frac{p_c}{\delta_c}
+
\mu\frac{U_c^2}{\delta^2}
+
k
\frac{ T_c}{L_c^2}
+
k
\frac{ T_c}{\delta_c^2}
+
\frac{\rho_cD_i}{\delta_c^2}
+...
\end{equation} 
%
\begin{equation} 
\frac{1}{L_c}
+
\frac{1}{L_c}
=
\frac{p_c}{\rho_c h_c}
\frac{1}{L_c}
+
\frac{p_c}{\rho_c  h_c}
\frac{1}{L_c}
+
\mu
\frac{U_c}{\rho_c  h_c}
\frac{1}{\delta^2}
+
\cancelto{\approx0}{
k
\frac{1}{\rho_c U_c h_c}
\frac{ T_c}{L_c^2}
}
+
k
\frac{1}{\rho_c U_c h_c}
\frac{ T_c}{\delta_c^2}
+
\frac{D_i}{\delta_c^2h_c}
+...
\end{equation} 
%%
%
\\
Then the Energy conservation gets:
\\
\begin{equation} 
\frac{\partial}{\partial x}(\rho u h)
+
\frac{\partial}{\partial y}(\rho v h)
=
u_i\frac{\partial p}{\partial x_i}
+
\mu
\left(\frac{\partial u}{\partial y}\right)^2
+
\frac{\partial }{\partial y}
\left(
k
\frac{\partial T}{\partial y}
\right)
-
\frac{\partial}{\partial y}
\left(
\sum_i\rho D_i\frac{\partial Y_i}{\partial y}
\right)
+
Q\Omega
\end{equation} 
\\
Using the mass conservation equation:
%
\begin{equation} 
h
\cancelto{\approx0}{
\left(
\frac{\partial}{\partial x}(\rho u )
+
\frac{\partial}{\partial y}(\rho v )
\right)
}
+
\rho u
\frac{\partial h}{\partial x}
+
\rho v
\frac{\partial h}{\partial y}
=
u_i\frac{\partial p}{\partial x_i}
+
\mu
\left(\frac{\partial u}{\partial y}\right)^2
+
\frac{\partial }{\partial y}
\left(
k
\frac{\partial T}{\partial y}
\right)
-
\sum_i\rho D_i\frac{\partial Y_i}{\partial y}
+
Q\Omega
\end{equation} 
%
\begin{equation} 
\rho u
\frac{\partial h}{\partial x}
+
\rho v
\frac{\partial h}{\partial y}
=
u_i\frac{\partial p}{\partial x_i}
+
\mu
\left(\frac{\partial u}{\partial y}\right)^2
+
\frac{\partial }{\partial y}
\left(
k
\frac{\partial T}{\partial y}
\right)
-
\frac{\partial }{\partial y}
\left(
\sum_i\rho D_i\frac{\partial Y_i}{\partial y}
\right)
+
Q\Omega
\end{equation} 
\\
The equations that govern the Steady Reacting Mixing Layer are: 
\\
\\
\begin{equation} 
\rho u\frac{\partial Y^{o}}{\partial x}+
\rho v\frac{\partial Y^{o}}{\partial y}
=
\frac{\partial}{\partial y}\left(\rho D^o\frac{\partial Y^o}{\partial y}\right)+
s^{o}\Omega
%\frac{\partial\rho D^m \tfrac{\partial\rho Y^m u_i}{\partial x_i}u_i}{\partial x_i}=
\end{equation} 
\\
\begin{equation} 
\rho u\frac{\partial Y^{F}}{\partial x}+
\rho v\frac{\partial Y^{F}}{\partial y}
=
\frac{\partial}{\partial y}\left(\rho D^F\frac{\partial Y^F}{\partial y}\right)+
s^{F}\Omega
%\frac{\partial\rho D^m \tfrac{\partial\rho Y^m u_i}{\partial x_i}u_i}{\partial x_i}=
\end{equation} 
\\
\\
\begin{equation} 
\rho u\frac{\partial u}{\partial x}+
\rho v\frac{\partial u}{\partial y}=-
\frac{\partial p}{\partial x}+
\frac{\partial}{\partial y}\left(\mu\frac{\partial u}{\partial y}\right)+
\rho f_x
\end{equation} 
\\
\\
\begin{equation} 
\frac{\partial p}{\partial y}=
\rho f_y.
\end{equation} 
\\
\\
\begin{equation}
%\frac{\partial}{\partial t}(\rho h)+
\rho u \frac{\partial h}{\partial x}+
\rho v \frac{\partial h}{\partial y}=
%\frac{\partial p}{\partial t}+
u\frac{\partial p}{\partial x}+
v\frac{\partial p}{\partial y}+
%\frac{\partial u_j}{\partial x_i}\tau_{ij}-
\mu\frac{\partial^2 u}{\partial y^2}+
\frac{\partial}{\partial y}
\left(
   k\frac{\partial T}{\partial y}\right)
-
\frac{\partial}{\partial y}
\left(
\sum_i\rho D_i\frac{\partial Y_i}{\partial y}
\right)
+
Q\Omega
\end{equation} 
\\
\\
The above equations can be more simplified if we omit the body force
in all direction and  with a zero pressure gradient,  became: 
\\
\\
\begin{equation} 
\rho u\frac{\partial Y^{o}}{\partial x}+
\rho v\frac{\partial Y^{o}}{\partial y}
=
\frac{\partial}{\partial y}\left(\rho D^o\frac{\partial Y^o}{\partial y}\right)+
s^{o}\Omega
%\frac{\partial\rho D^m \tfrac{\partial\rho Y^m u_i}{\partial x_i}u_i}{\partial x_i}=
\end{equation} 
\\
\\
\begin{equation} 
\rho u\frac{\partial Y^{F}}{\partial x}+
\rho v\frac{\partial Y^{F}}{\partial y}
=
\frac{\partial}{\partial y}\left(\rho D^F\frac{\partial Y^F}{\partial y}\right)+
s^{F}\Omega
%\frac{\partial\rho D^m \tfrac{\partial\rho Y^m u_i}{\partial x_i}u_i}{\partial x_i}=
\end{equation} 
\\
\\
\begin{equation} 
\rho u\frac{\partial u}{\partial x}+
\rho v\frac{\partial u}{\partial y}=
%-\frac{\partial p}{\partial x}+
\frac{\partial}{\partial y}\left(\mu\frac{\partial u}{\partial y}\right)
%\rho f_x
\end{equation} 
%\\
%\\
%\begin{equation} 
%\frac{\partial p}{\partial y}=0
%\end{equation} 
\\
\\
\begin{equation}
%\frac{\partial}{\partial t}(\rho h)+
\rho u \frac{\partial h}{\partial x}+
\rho v \frac{\partial h}{\partial y}=
%\frac{\partial p}{\partial t}+
%u\frac{\partial p}{\partial x}+
%v\frac{\partial p}{\partial y}+
%\frac{\partial u_j}{\partial x_i}\tau_{ij}-
\mu\left(\frac{\partial u}{\partial y}\right)^2+
\frac{\partial}{\partial y}
\left(
   k\frac{\partial T}{\partial y}
   \right)
-
\frac{\partial}{\partial y}
\left(
\sum_i\rho D_i\frac{\partial Y_i}{\partial y}
\right)
+
Q\Omega
\end{equation} 
\\
\\
\begin{equation}
p=\rho R T
\end{equation} 
\\
\begin{equation}
h=c_pT
\end{equation} 
\\
\begin{equation}
\Omega=\beta Y_{F}Y_{O}\exp\left(\frac{E}{RT}\right)
\end{equation} 

\clearpage 

\textbf{Boundary conditions}

\textbf{x $\longrightarrow +\infty$}
\\
$u=u_{+\infty}$
\\
$v=v_{+\infty}=0$
\\
$p=p_{+\infty}=constant=C$
\\
$\rho=\rho_{+\infty}=0$
\\
$Y_F=Y_{F+\infty}=1$
\\
$Y_O=Y_{O+\infty}=0$
\\

\textbf{x $\longrightarrow -\infty$}
\\
$u=u_{-\infty}$
\\
$v=v_{-\infty}=0$
\\
$p=p_{-\infty}=constant=C$
\\
$\rho=\rho_{-\infty}=0$
\\
$Y_F=Y_{F+\infty}=0$
\\
$Y_O=Y_{O+\infty}=1$
 




\section{Dimensionless Equations}
%%
The dimesionless equations for a reacting mixing layer are obtined using
the positive freestream variables,$L_c$ characteritic lenghts:
\\
\\
\begin{equation}
%\overline{t}=t\frac{L_c}{u_{\infty}} \qquad 
\overline{x}=\frac{x}{L_c}, \qquad 
\overline{y}=\frac{y}{\delta}, \qquad 
\end{equation} 
\\
\begin{equation}
U=\frac{u}{u_{\infty}}, \qquad 
V=\frac{v}{v_{\infty}},\qquad\text{but}\qquad
v_{\infty}=\frac{u_{\infty}\delta_c}{L_c},\qquad\text{then}\qquad
V=\frac{L_c}{\delta}\frac{v}{u_{\infty}},\qquad
\end{equation} 
\\
\begin{equation}
\overline{\rho}=\frac{\rho}{\rho_{\infty}}, \qquad
\overline{T}=\frac{T}{T_{\infty}}, \qquad
\overline{h}=\frac{h}{h_{\infty}}, \qquad
P=\frac{p}{P_{\infty}}, 
\end{equation} 
\\
\begin{equation}
\psi_F=\frac{Y^F}{Y^{F\infty}}, \qquad
\psi_O=\frac{Y^O}{Y^{O\infty}}, 
\end{equation} 
\\
\begin{equation}
\frac{\mu}{\mu_{\infty}} =
\frac{k}{k_{\infty}} =
\frac{c_v}{c_{v\infty}} =
\frac{c_p}{c_{p\infty}} =
\frac{D^F}{D^{F\infty}}= 
\frac{D^0}{D^{O\infty}}= 
\frac{R}{R_\infty}= 
\overline{T}^n,
\end{equation} 
\\
\\
\textbf{Mass conservation:}
\\
\\
\begin{equation} 
\frac{\partial}{\partial x}(\rho u)+
\frac{\partial}{\partial y}(\rho v)=0
\label{mass2}
\end{equation} 
\\
\\
\begin{equation} 
\frac{\rho_{\infty}U_{\infty}}{L_c}\frac{\partial}{\partial \overline{x}}(\varrho U)+
\frac{\rho_{\infty}U_{\infty}}{L_c}\frac{\partial}{\partial \overline{y}}(\varrho V)=0
\label{mass2}
\end{equation} 
\\
\\
\begin{equation} 
\frac{\partial}{\partial \overline{x}}(\varrho U)+
\frac{\partial}{\partial \overline{y}}(\varrho V)=0
\label{mass2}
\end{equation} 
\\
\\
\textbf{Momemtum equation:}
\\
\\
\begin{equation} 
\frac{\rho_{\infty}U_{\infty}^2}{L_c}\varrho U\frac{\partial U}{\partial \overline{x}}+
\frac{\rho_{\infty}U_{\infty}^2}{L_c}\varrho V\frac{\partial U}{\partial \overline{y}}=
\frac{\mu_{\infty}U_{\infty}}{\delta^2}\frac{\partial}{\partial \overline{y}}\left(\mu\frac{\partial U}{\partial  \overline{y}}\right)
\label{mnn}
\end{equation} 
\\
\\
We do not known the relation between characteristics lenghts  of  the problem, $\delta$ and $L_c$. 
For the above equation have the same same order, we can defined:
\\
\\
\begin{equation} 
\frac{\rho_{\infty}U_{\infty}}{L_c}=\frac{\mu_{\infty}}{\delta^2}
\end{equation} 
\\
\\
\begin{equation} 
\frac{\delta^2}{L_c}=\frac{\mu_{\infty}}{\rho_{\infty}U_{\infty}}
\end{equation} 
\\
\\
\begin{equation} 
\frac{\delta^2}{L_c}=L_c\frac{\mu_{\infty}}{\rho_{\infty}L_c U_{\infty}}
\end{equation} 
\\
\\
Defining:
\\
\\
\begin{equation} 
Re_{L_c}=\frac{\mu_{\infty}}{\rho_{\infty}L_c U_{\infty}}
\end{equation} 
\\
\\
as the Reynolds number respect to $L_c$, we get:
\\
\\
\begin{equation} 
\frac{\delta^2}{L_c}    =L_c\frac{1}{Re_{L_c}}
\end{equation} 
\\
\\
then we obtain:
\\
\\
\begin{equation} 
\delta=L_c\sqrt{\frac{1}{Re_{L_c}}}
\end{equation} 
\\
%%At the beginning we assume  
%%\\
%%\\
%%\begin{equation} 
%%\frac{\delta}{L_c}<<1
%%,
%%\end{equation} 
%%
%%or in others words 
%%
%%\begin{equation} 
%%\sqrt{\frac{1}{Re}}<<1,\qquad
%%{Re}>>1.
%%\end{equation} 

Inserting this relation in the dimensionless Momemtum equation \ref{mnn}: 
\\
\\
\begin{equation} 
\frac{\rho_{\infty}U_{\infty}^2}{L_c}\varrho U\frac{\partial U}{\partial \overline{x}}+
\frac{\rho_{\infty}U_{\infty}^2}{L_c}\varrho V\frac{\partial U}{\partial \overline{y}}=
\frac{\mu_{\infty}U_{\infty}}{\delta^2}\frac{\partial}{\partial \overline{y}}\left(\mu\frac{\partial U}{\partial  \overline{y}}\right)
\label{mn1}
\end{equation} 
\\
\\
we obtain:
\\
\\
\begin{equation} 
\varrho U\frac{\partial U}{\partial \overline{x}}+
\varrho V\frac{\partial U}{\partial \overline{y}}=
\frac{\partial}{\partial \overline{y}}\left(\mu\frac{\partial U}{\partial  \overline{y}}\right)
\end{equation} 
\\
\textbf{Species conservation:}
\\
\\
\begin{equation} 
\frac{\rho_{\infty}U_{\infty}Y^{O\infty}}{L_c}\varrho U\frac{\partial \psi_{o}}{\partial\overline{x}}+
\frac{\rho_{\infty}U_{\infty}Y^{O\infty}}{L_c}\varrho V\frac{\partial \psi_{o}}{\partial\overline{y}}+
\frac{\rho_{\infty}D^{O\infty}Y^{O\infty}}{\delta_c^2}\frac{\partial}{\partial \overline{y}}\left(\rho D^o\frac{\partial \psi_o}{\partial \overline{y}}\right)+
s^{o}\Omega
%\frac{\partial\rho D^m \tfrac{\partial\rho Y^m u_i}{\partial x_i}u_i}{\partial x_i}=
\end{equation} 
\\
\begin{equation} 
\frac{\rho_{\infty}U_{\infty}Y^{F\infty}}{L_c}\varrho U\frac{\partial \psi_{F}}{\partial\overline{x}}+
\frac{\rho_{\infty}U_{\infty}Y^{F\infty}}{L_c}\varrho V\frac{\partial \psi_{F}}{\partial\overline{y}}+
\frac{\rho_{\infty}D^{F\infty}Y^{F\infty}}{\delta^2}\frac{\partial}{\partial \overline{y}}\left(\rho D^F\frac{\partial \psi_F}{\partial \overline{y}}\right)+
s^{F}\Omega
%\frac{\partial\rho D^m \tfrac{\partial\rho Y^m u_i}{\partial x_i}u_i}{\partial x_i}=
\end{equation} 
\\
\\
\begin{equation} 
\varrho U\frac{\partial \psi_{o}}{\partial\overline{x}}+
\varrho V\frac{\partial \psi_{o}}{\partial\overline{y}}+
%\frac{D^{O\infty}}{L_cU}\frac{\partial}{\partial \overline{y}}
\frac{L_c}{\delta^2_c}
\frac{D^{O\infty}}{U_\infty}\frac{\partial}{\partial \overline{y}}
\left(\rho D^o\frac{\partial \psi_o}{\partial \overline{y}}\right)+
s^{o}\Omega
%\frac{\partial\rho D^m \tfrac{\partial\rho Y^m u_i}{\partial x_i}u_i}{\partial x_i}=
\end{equation} 
\\
\\
\begin{equation} 
\varrho U\frac{\partial \psi_{F}}{\partial\overline{x}}+
\varrho V\frac{\partial \psi_{F}}{\partial\overline{y}}+
\frac{L_c}{\delta_c}
\frac{1}{\delta_c}
\frac{D^{F\infty}}{U_\infty}\frac{\partial}{\partial \overline{y}}\left(\rho D^F\frac{\partial \psi_F}{\partial \overline{y}}\right)+
s^{F}\Omega
%\frac{\partial\rho D^m \tfrac{\partial\rho Y^m u_i}{\partial x_i}u_i}{\partial x_i}=
\end{equation} 
%
Using the definition of the Prandtl, Lewis numbers  and 
the characteristic lenghts relation, we can write:
%
\begin{equation} 
Pr=\frac{\nu}{\alpha}=\frac{\mu_{\infty}c_{p\infty}}{k_{\infty}}
,
\qquad
Le_i=\frac{\alpha}{D^{i\infty}}=\frac{k_{\infty}}{\rho_{\infty}c_{p\infty}}
\frac{1}{D^{i\infty}}
,
\quad
\alpha=\frac{k_{\infty}}{\rho_{\infty}c_{p\infty}}
\end{equation} 

where $i$ is the specie, fuel (F) or oxidizer (O).

\begin{equation} 
\frac{L_c}{\delta_c^2}
\frac{D^{O\infty}}{U_\infty}
%\frac{\delta^2}{L_c}=L_c\frac{1}{Re_{L_c}}
%\frac{\delta^2}{L_c}
=
\frac{Re_{L_c}}{L_c}
\frac{\mu_{\infty}}{U_\infty\rho_{\infty}}
\frac{\rho_{\infty}D^{O\infty}}{\mu_{\infty}}
=
\frac{\rho_{\infty}D^{O\infty}c_{p\infty}}{k_{\infty}}
\frac{k_{\infty}}{\mu_{\infty}c_{p\infty}}
=
\frac{1}{Le_O}
\frac{1}{Pr}
%\frac{1}{Le_O},\qquad \text{$Le_O$ is the Lewis number for the oxidizer.}
\end{equation} 
\\
\begin{equation} 
\frac{L_c}{\delta_c}
\frac{D^{F\infty}}{\delta_cU_\infty}=\frac{1}{Le_FPr},
\end{equation} 
\\
\\
Then we get:
\\
\\
\begin{equation} 
\varrho U\frac{\partial \psi_{o}}{\partial\overline{x}}+
\varrho V\frac{\partial \psi_{o}}{\partial\overline{y}}+
\frac{1}{Le_oPr}\frac{\partial}{\partial \overline{y}}\left(\rho D^o\frac{\partial \psi_o}{\partial \overline{y}}\right)+
s^{o}\Omega
%\frac{\partial\rho D^m \tfrac{\partial\rho Y^m u_i}{\partial x_i}u_i}{\partial x_i}=
\end{equation} 
\\
\\
\begin{equation} 
\varrho U\frac{\partial \psi_{F}}{\partial\overline{x}}+
\varrho V\frac{\partial \psi_{F}}{\partial\overline{y}}+
\frac{1}{Le_FPr}\frac{\partial}{\partial \overline{y}}\left(\rho D^F\frac{\partial \psi_F}{\partial \overline{y}}\right)+
s^{F}\Omega.
%\frac{\partial\rho D^m \tfrac{\partial\rho Y^m u_i}{\partial x_i}u_i}{\partial x_i}=
\end{equation} 
\\
\\
\\
\\
Before non-dimensionalization the energy equation, we use the equation that relate the entaphy and 
the temperature for a perfect gas, we get: 
\\
\\
\begin{equation}
%\frac{\partial}{\partial t}(\rho h)+
\rho u \frac{\partial }{\partial x}(c_p T)+
\rho v \frac{\partial }{\partial y}(c_p T)=
%\frac{\partial p}{\partial t}+
%u\frac{\partial p}{\partial x}+
%v\frac{\partial p}{\partial y}+
%\frac{\partial u_j}{\partial x_i}\tau_{ij}-
\mu
\left(
\frac{\partial u}{\partial y}
\right)^2
+
\frac{\partial}{\partial y}\left(k\frac{\partial T}{\partial y}\right)
-
\frac{\partial }{\partial y}
\left(
\sum_i\rho \overline{h}_iD_i\frac{\partial \psi_i}{\partial y}
\right)
+
Q\Omega
\end{equation} 
\\
\\
Using the non-dimensional variables:
\\
\\
\begin{equation}
\begin{split}
%\frac{\partial}{\partial t}(\rho h)+
\frac{\rho_{\infty}U_{\infty}T_{\infty}c_{p\infty}}{L_c}\varrho U\frac{\partial }{\partial \overline{x}}(c_p \overline{T})+
\frac{\rho_{\infty}U_{\infty}T_{\infty}c_{p\infty}}{L_c}\varrho V\frac{\partial }{\partial \overline{y}}(c_p \overline{T})=
%\frac{\partial p}{\partial t}+
%u\frac{\partial p}{\partial x}+
%v\frac{\partial p}{\partial y}+
%\frac{\partial u_j}{\partial x_i}\tau_{ij}-
\frac{\mu_{\infty}U_{\infty}^2}{\delta^2}
\mu
\left(
\frac{\partial U}{\partial \overline{y}}
\right)^2
-
\\
%\frac{k_{\infty}T_{\infty}}{\delta^2}\frac{\partial}{\partial \overline{y}}\left(k\frac{\partial \overline{T}}{\partial \overline{y}}\right)
%+
\frac{\rho_{\infty}T_{\infty}c_{p\infty}D^{i\infty}Y^{i\infty}}{\delta^2}
\frac{\partial }{\partial \overline{y}}
\left(
\sum_i\rho \overline{h}_i D_i\frac{\partial \psi_i}{\partial \overline{y}}
\right)
+
Q\Omega
\end{split}
\end{equation} 
\\
\begin{equation}
\begin{split}
\frac{\rho_{\infty}D^{i\infty}Y^{i\infty}T_{\infty}c_{p\infty}}{\delta^2}
\frac{L_c}{\rho_{\infty}U_{\infty}T_{\infty}c_{p\infty}}
=
\frac{L_c^2}{\delta^2}
\frac{D^{i\infty}}{\alpha}
\frac{\alpha}{\nu}
\frac{\nu Y^{i\infty}}{U_{\infty}L_c}
=
\\
\frac{L_c^2}{\delta^2}
\frac{D^{i\infty}}{\alpha}
\frac{\alpha}{\nu}
\frac{\mu }{\rho U_{\infty}L_c}
Y^{i\infty}
=
\frac{L_c^2}{\delta^2}
\frac{1}{Le_iPrRe}Y^{i\infty}
=
\frac{Y^{i\infty}}{Le_iPr}
\end{split}
\end{equation} 
\\
\begin{equation}
\begin{split}
%\frac{\partial}{\partial t}(\rho h)+
\varrho U\frac{\partial }{\partial \overline{x}}(c_p \overline{T})+
\varrho V\frac{\partial }{\partial \overline{y}}(c_p \overline{T})
=
(\gamma-1)M^2\mu
\left(
\frac{\partial U}{\partial \overline{y}}
\right)^2
+
\\
\frac{1}{Pr}\frac{\partial}{\partial \overline{y}}\left(k\frac{\partial \overline{T}}{\partial \overline{y}}\right)
+
\frac{1}{Pr}\frac{\partial}{\partial \overline{y}}\left(k\frac{\partial \overline{T}}{\partial \overline{y}}\right)
-
\frac{Y^{i\infty}}{Pr}
\frac{\partial }{\partial \overline{y}}
\left(
\sum_i \frac{\varrho}{Le_i}\overline{h}_i D_i\frac{\partial \psi_i}{\partial \overline{y}}
\right)
+
Q\Omega
\end{split}
\end{equation} 
\\
\\
where:
\\
\\
\begin{equation}
M=\frac{U_{\infty}}{a_{0\infty}}\qquad\text{Mach Number with}\qquad a_{0\infty}=\sqrt{\gamma_{\infty}R_{\infty}T_{\infty}}
\end{equation} 
\\
\\
\begin{equation}
Pr=\frac{\mu_{\infty}c_{p\infty}}{k_{\infty}}
\end{equation} 

\begin{equation}
Le_n=\frac{\alpha}{D^{i\infty}}.
\end{equation} 

Finaly the equation for a perfect gas became:  
\\
\\
\begin{equation}
1=\varrho\overline{T}
\end{equation} 
\\
\\
\clearpage
\textbf{Non-dimensional Reacting Mixing layer equation:}
\\
\\
\\
\begin{equation} 
\frac{\partial}{\partial \overline{x}}(\varrho U)+
\frac{\partial}{\partial \overline{y}}(\varrho V)=0
\label{mass2}
\end{equation} 
\\
\\
\begin{equation} 
\varrho U\frac{\partial \psi_{o}}{\partial\overline{x}}+
\varrho V\frac{\partial \psi_{o}}{\partial\overline{y}}=
\frac{1}{Le_o}\frac{\partial}{\partial \overline{y}}\left(\rho D^o\frac{\partial \psi_o}{\partial \overline{y}}\right)+
s^{o}\Omega
%\frac{\partial\rho D^m \tfrac{\partial\rho Y^m u_i}{\partial x_i}u_i}{\partial x_i}=
\end{equation} 
\\
\\
\begin{equation} 
\varrho U\frac{\partial \psi_{F}}{\partial\overline{x}}+
\varrho V\frac{\partial \psi_{F}}{\partial\overline{y}}=
\frac{1}{Le_F}\frac{\partial}{\partial \overline{y}}\left(\rho D^F\frac{\partial \psi_F}{\partial \overline{y}}\right)+
s^{F}\Omega.
%\frac{\partial\rho D^m \tfrac{\partial\rho Y^m u_i}{\partial x_i}u_i}{\partial x_i}=
\end{equation} 
\\
\\
\begin{equation} 
\varrho U\frac{\partial U}{\partial \overline{x}}+
\varrho V\frac{\partial U}{\partial \overline{y}}=
\frac{\partial}{\partial \overline{y}}\left(\mu\frac{\partial U}{\partial  \overline{y}}\right)
\end{equation} 
\\
\\
\begin{equation}
%\frac{\partial}{\partial t}(\rho h)+
\varrho U\frac{\partial }{\partial \overline{x}}(c_p \overline{T})+
\varrho V\frac{\partial }{\partial \overline{y}}(c_p \overline{T})=
(\gamma-1)M^2\mu
\left(
\frac{\partial U}{\partial y}
\right)^2
\frac{1}{Pr}\frac{\partial}{\partial \overline{y}}\left(k\frac{\partial \overline{T}}{\partial \overline{y}}\right)
-
\frac{Y^{i\infty}}{Pr}
\frac{\partial }{\partial \overline{y}}
\left(
\sum_i\frac{\varrho}{Le_i}\overline{h}_i D_i\frac{\partial \psi_i}{\partial \overline{y}}
\right)
+
Q\Omega
\end{equation} 
\\
\\
\begin{equation}
1=\varrho\overline{T}
\label{g2}
\end{equation} 


\section{Self-Similarity solution for Reacting Mixing Layer equations}
Begins with the non-divergent boundary layer form of the compressible equations, 

\begin{equation} 
\frac{\partial \hat{U}}{\partial x}+
\frac{\partial \hat{V}}{\partial y}
=0
,
\end{equation} 

for two  species, the oxidizer (O) and 

\begin{equation} 
 \hat{U}\frac{\partial \psi_{o}}{\partial x}+
 \hat{V}\frac{\partial \psi_{o}}{\partial y}
=
\frac{1}{PrLe_o}\frac{\partial}{\partial y}\left(\varrho^2 D^o\frac{\partial \psi_o}{\partial y}\right)+
\frac{s^{o}\Omega}{\varrho}
%\frac{\partial\rho D^m \tfrac{\partial\rho Y^m u_i}{\partial x_i}u_i}{\partial x_i}=
,
\end{equation} 
  
the fuel (F), 

\begin{equation} 
 \hat{U}\frac{\partial \psi_{F}}{\partial x}+
 \hat{V}\frac{\partial \psi_{F}}{\partial y}
=
\frac{1}{PrLe_F}\frac{\partial}{\partial y}\left(\varrho^2 D^F\frac{\partial \psi_F}{\partial y}\right)+
\frac{s^{F}\Omega}{\varrho}
%\frac{\partial\rho D^m \tfrac{\partial\rho Y^m u_i}{\partial x_i}u_i}{\partial x_i}=
\end{equation} 

momentum equation in the $x$ direction:
  
\begin{equation} 
 \hat{U}\frac{\partial \hat{U}}{\partial x}+
 \hat{V}\frac{\partial \hat{U}}{\partial y}=
\frac{\partial}{\partial y}\left(\mu\varrho\frac{\partial \hat{U}}{\partial  y}\right)
\end{equation} 
  
and energy equation 
  
\begin{equation}
\begin{split}
\hat{U}\frac{\partial }{\partial x}(c_p \overline{T})+
\hat{V}
\frac{\partial }{\partial y}(c_p \overline{T})=
\varrho
(\gamma-1)M^2\mu
\left(
\frac{\partial \hat{U}}{\partial y}
\right)
^2
+
\\
\frac{1}{Pr}\frac{\partial}{\partial y}\left(k\varrho\frac{\partial \overline{T}}{\partial y}\right)
-
\frac{Y^{i\infty}}{Pr}
\frac{\partial}{\partial y}
       \left(
              \frac{1}{Le_i}
              \sum_i\varrho^2 D_i\overline{h}_i\frac{\partial \psi_i}{\partial y}
       \right)
+
\frac{Q\Omega}{\varrho}
\end{split}
\end{equation} 

with $c_p=\sum_i \psi_i c_{p,i}$.

The energy equation can be rewritten in terms of the total enthalpy, 

\begin{equation}
H=\overline{h}+\frac{1}{2}\hat{U}^2
=c_p \overline{T}+\frac{1}{2}\hat{U}^2. 
\end{equation}

Multiplying the momentum equation by $\hat{U}$ 

\begin{equation} 
\hat{U}
 \hat{U}\frac{\partial \hat{U}}{\partial x}
 +
\hat{U}
 \hat{V}\frac{\partial \hat{U}}{\partial y}
 =
\hat{U}
\frac{\partial}{\partial y}\left(\mu\varrho\frac{\partial \hat{U}}{\partial  y}\right)
\end{equation} 

and adding to the energy
equation, we obtain

\begin{equation} 
\begin{split}
\hat{U}
 \hat{U}\frac{\partial \hat{U}}{\partial x}
 +
\hat{U}
 \hat{V}\frac{\partial \hat{U}}{\partial y}
+
\hat{U}\frac{\partial }{\partial x}(c_p \overline{T})
+
\hat{V}
\frac{\partial }{\partial y}(c_p \overline{T})
 =
 \\
\hat{U}
\frac{\partial}{\partial y}\left(\mu\varrho\frac{\partial \hat{U}}{\partial  y}\right)
+
\varrho
(\gamma-1)M^2\mu
\left(
\frac{\partial \hat{U}}{\partial y}
\right)
^2
+
\\
\frac{1}{Pr}\frac{\partial}{\partial y}\left(k\varrho\frac{\partial \overline{T}}{\partial y}\right)
-
\frac{Y^{i\infty}}{Pr}
\frac{\partial}{\partial y}
       \left(
              \frac{1}{Le_i}
              \sum_i\varrho^2 D_i\overline{h}_i\frac{\partial \psi_i}{\partial y}
       \right)
+
\frac{Q\Omega}{\varrho}
\end{split}
\end{equation}       

\begin{equation} 
\begin{split}
\hat{U}
 \hat{U}\frac{\partial \hat{U}}{\partial x}
 +
\hat{U}
 \hat{V}\frac{\partial \hat{U}}{\partial y}
+
\hat{U}\frac{\partial }{\partial x}(c_p \overline{T})
+
\hat{V}
\frac{\partial }{\partial y}(c_p \overline{T})
 =
 \\
\frac{\partial}{\partial y}\left(\mu \varrho\hat{U}\frac{\partial \hat{U}}{\partial  y}\right)
-
\mu 
\varrho
\left(
\frac{\partial \hat{U}}{\partial y}
\right)
^2
+
\varrho
(\gamma-1)M^2\mu
\left(
\frac{\partial \hat{U}}{\partial y}
\right)
^2
+
\\
\frac{1}{Pr}\frac{\partial}{\partial y}
\left(kc_p\varrho\frac{\partial \overline{h}}{\partial y}\right)
-
\frac{Y^{i\infty}}{Pr}
\frac{\partial}{\partial y}
       \left(
              \frac{1}{Le_i}
              \sum_i\varrho^2 D_i\overline{h}_i\frac{\partial \psi_i}{\partial y}
       \right)
+
\frac{Q\Omega}{\varrho}
\end{split}
\end{equation}       

Note that  

\begin{equation}
\overline{h}=\sum_{i} \psi_i \overline{h}_i
\end{equation}

then, 

\begin{equation}
\frac{\partial \overline{h}}{\partial y}
=
\sum_{i} \psi_i \frac{\partial \overline{h}_i}{\partial y}
+
\sum_{i} 
\frac{\partial \psi_i}{\partial y} \overline{h}_i
.
\end{equation}

Therefore,

\begin{equation} 
\begin{split}
(\rho u)\frac{\partial H}{\partial x}    
+
(\rho v)\frac{\partial H}{\partial y}    
=
 \\
\frac{\partial}{\partial y}
\left(
       \mu  \varrho\hat{U}\frac{\partial \hat{U}}{\partial  y}
 \right)
-
\mu 
\varrho
\left(
1-
(\gamma-1)M^2
\right)
\left(
\frac{\partial \hat{U}}{\partial y}
\right)
^2
\\
\frac{1}{Pr}
\frac{\partial}{\partial y}
\left[
       k\varrho c_p
       \left(
       \sum_{i} \psi_i \frac{\partial \overline{h}_i}{\partial y}
       +
       \sum_{i} 
       \frac{\partial \psi_i}{\partial y} \overline{h}_i
       \right)
\right]
-
\frac{Y^{i\infty}}{Pr}
\frac{\partial}{\partial y}
       \left(
              \frac{1}{Le_i}
              \sum_i\varrho^2 D_i\overline{h}_i\frac{\partial \psi_i}{\partial y}
       \right)
+
\frac{Q\Omega}{\varrho}
\end{split}
\end{equation}       

\begin{equation} 
\begin{split}
(\rho u)\frac{\partial H}{\partial x}    
+
(\rho v)\frac{\partial H}{\partial y}    
=
 \\
\frac{\partial}{\partial y}\left(\mu \varrho\hat{U}\frac{\partial \hat{U}}{\partial  y}\right)
-
\mu 
\varrho
\left(
1-
(\gamma-1)M^2
\right)
\left(
\frac{\partial \hat{U}}{\partial y}
\right)
^2
\\
\frac{1}{Pr}
\frac{\partial}{\partial y}
\left[
       k\varrho c_p
       \sum_{i} \psi_i \frac{\partial \overline{h}_i}{\partial y}
\right]
+
\frac{1}{Pr}
\frac{\partial}{\partial y}
\left[
       k\varrho c_p
       \sum_{i} 
       \frac{\partial \psi_i}{\partial y} \overline{h}_i
\right]
-
\frac{Y^{i\infty}}{Pr}
\frac{\partial}{\partial y}
\left(\sum_i\frac{1}{Le_i}\varrho^2 D_i\frac{\partial \psi_i}{\partial y}\overline{h}_i\right)
+
\frac{Q\Omega}{\varrho}
\end{split}
\end{equation}       

\begin{equation} 
\begin{split}
(\rho u)\frac{\partial H}{\partial x}    
+
(\rho v)\frac{\partial H}{\partial y}    
=
 \\
\frac{\partial}{\partial y}\left(\mu \varrho\hat{U}\frac{\partial \hat{U}}{\partial  y}\right)
-
\mu 
\varrho
\left(
1-
(\gamma-1)M^2
\right)
\left(
\frac{\partial \hat{U}}{\partial y}
\right)
^2
\\
\frac{1}{Pr}
\frac{\partial}{\partial y}
\left[
       k\varrho c_p
       \sum_{i} \psi_i \frac{\partial \overline{h}_i}{\partial y}
\right]
+
\frac{1}{Pr}
\left\{
\frac{\partial}{\partial y}
\left[
kc_p
\varrho
\left(
       \sum_{i} 
       \left(
       1
       -
       \frac{1}{Le_i}
       \frac{Y^{i\infty}
       \varrho D_i}{kc_p}
       \right)
       \frac{\partial \psi_i}{\partial y} \overline{h}_i
\right)
\right]
\right\}
+
\frac{Q\Omega}{\varrho}
\end{split}
\end{equation}       


\section{Crocco Busemann} %% %%
%%\section{Crocco Busemann} %% %%

In this section a special solution to boundary layer equation is found using the definition of the total enthalpy.

From the boundary layer equation in two dimensional flow 

\begin{equation}
\frac{\partial }{\partial x}(\rho u)    
+
\frac{\partial }{\partial y}(\rho v)    
=
0
\end{equation}

\begin{equation}
(\rho u)\frac{\partial u}{\partial x}    
+
(\rho v)\frac{\partial u}{\partial y}    
=
-
\frac{\partial  P}{\partial x}
+
\frac{\partial  }{\partial y}\left( \mu\frac{\partial  u }{\partial y}\right)
\end{equation}

\begin{equation}
(\rho u)\frac{\partial h}{\partial x}    
+
(\rho v)\frac{\partial h}{\partial y}    
=
u
\frac{\partial  P}{\partial x}
+
\frac{\partial  }{\partial y}
\left( \frac{\mu}{Pr}\frac{\partial  h }{\partial y}\right)
+
\mu
\left(
    \frac{\partial u }{\partial y}
\right)^2 
\end{equation}

where the Prantdl number is defined as $Pr=\mu c_p/k$.
Since the total enthalpy is defined as 

\begin{equation}
H=h+\frac{1}{2}u^2
\end{equation}

the total enthalpy equation using both 
the energy and momentum equation. 

Multiplying the momentum equation by $u$ 

\begin{equation}
    u
(\rho u)\frac{\partial u}{\partial x}    
+
u
(\rho v)\frac{\partial u}{\partial y}    
=
-
u
\frac{\partial  P}{\partial x}
+
u
\frac{\partial  }{\partial y}\left( \mu\frac{\partial  u }{\partial y}\right)
\end{equation}

\begin{equation}
\frac{1}{2}  
\rho u\frac{\partial u^2}{\partial x}    
+
\frac{1}{2}  
\rho v\frac{\partial u^2}{\partial y}    
=
-
u
\frac{\partial  P}{\partial x}
+
u
\frac{\partial  }{\partial y}\left( \mu \frac{\partial  u }{\partial y}\right)
\end{equation}

The last term can be written as

\begin{equation}
u
\frac{\partial  }{\partial y}\left( \mu \frac{\partial  u }{\partial y}\right)
=
\frac{\partial  }{\partial y}\left( \mu u \frac{\partial  u }{\partial y}\right)
-
\mu
\left(\frac{\partial u }{\partial y}
\right)^2
,
\end{equation}


and adding it to the energy equation

\begin{equation}
\begin{split}
\frac{1}{2}  
(\rho u)\frac{\partial u^2}{\partial x}    
+
(\rho u)\frac{\partial h}{\partial x}    
+
\frac{1}{2}  
(\rho v)\frac{\partial u^2}{\partial y}    
+
(\rho v)\frac{\partial h}{\partial y}    
=
\\
\frac{\partial  }{\partial y}\left( \mu u \frac{\partial  u }{\partial y}\right)
-
\cancel{
\mu
\left(\frac{\partial u }{\partial y}
\right)^2
}
+
\frac{\partial  }{\partial y}
\left( \frac{\mu}{Pr}\frac{\partial  h }{\partial y}\right)
+
\cancel{
    \mu
\left(
    \frac{\partial u }{\partial y}
\right)^2 
}
\end{split}
\end{equation}

\begin{equation}
(\rho u)\frac{\partial H}{\partial x}    
+
(\rho v)\frac{\partial H}{\partial y}    
=
\frac{\partial  }{\partial y}
\left[
\mu
\left(  
    u
    \frac{\partial  u }{\partial y}
    +
    \frac{1}{Pr}
    \frac{\partial  h }{\partial y}
\right)
\right]
\end{equation}

\begin{equation}
\frac{\partial h}{\partial y}
=
\frac{\partial  H}{\partial y}
-
u
\frac{\partial  u }{\partial y}
\end{equation}

\begin{equation}
(\rho u)\frac{\partial H}{\partial x}    
+
(\rho v)\frac{\partial H}{\partial y}    
=
\frac{\partial  }{\partial y}
\left\{
\mu
    \left[
    u
    \frac{\partial  u }{\partial y}
    +
    \frac{1}{Pr}
    \left(
    \frac{\partial  H}{\partial y}
    -
    u
    \frac{\partial  u }{\partial y}
    \right)
    \right]
\right\}
\end{equation}

\begin{equation}
(\rho u)\frac{\partial H}{\partial x}    
+
(\rho v)\frac{\partial H}{\partial y}    
=
\frac{\partial  }{\partial y}
\left[
\frac{\mu}{Pr}
\left(  
    \frac{\partial  H}{\partial y}
\right)
\right]
+
\frac{\partial  }{\partial y}
\left[
\mu
\left(  
    1-
    \frac{1}{Pr}
\right)
u
\frac{\partial  u }{\partial y}
\right]
\end{equation}


This equation allows a simplest solution  
$H=h+\frac{1}{2}u^2=$ constant as long as $Pr=1$ of a adiabatic wall,
which leads to  

\begin{equation}
\left.\frac{\partial T}{\partial y}\right|_{w}
=
\left.\frac{\partial h}{\partial y}\right|_{w}
=
\left.\frac{\partial H}{\partial y}\right|_{w}
=
0
\end{equation}

Representing the not heat transfer at the wall.

$Pr=1$ implies in a perfect balance between viscous 
dissipation and heat conduction so as keep the 
the stagnation enthalpy constant in adiabatic boundary layer 
and also is a good approximation for gases.
So velocity and temperature are smoothed by the same mechanism.


Another solution can be obtained if 
the pressure gradient is neglected
in the boundary layer equation.  
The momentum equation and the total enthalpy equation 
are identical.  Total enthalpy behaves like a passive scalar.
This implies a direct functional relationship between 
$H$ and $u$, $H=H(u)$.
With this, the momentum equation and the energy equation
get very similar,
it seems as if $u$ and $h$ could be 
interchanged except for the dissipation term. 


\begin{equation}
(\rho u)\frac{\partial u}{\partial x}    
+
(\rho v)\frac{\partial u}{\partial y}    
=
\frac{\partial  }{\partial y}\left( \mu\frac{\partial  u }{\partial y}\right)
\end{equation}

\begin{equation}
(\rho u)\frac{\partial h}{\partial x}    
+
(\rho v)\frac{\partial h}{\partial y}    
=
\frac{\partial  }{\partial y}
\left( \frac{\mu}{Pr}\frac{\partial  h }{\partial y}\right)
+
\mu
\left(
    \frac{\partial u }{\partial y}
\right)^2 
\end{equation}



Then, $h=h(u)$ and $u=u(x,y)$, a solution of the form 

\begin{equation}
\frac{\partial h }{\partial y}
=
\frac{d h} { du}
\frac{\partial u }{\partial y}
\end{equation}


\begin{equation}
\frac{\partial  }{\partial y}
\left(
    \frac{\partial h }{\partial y}
\right)
=
\frac{\partial  }{\partial y}
\left(
\frac{d h} { du}
\frac{\partial u }{\partial y}
\right)
\end{equation}

\begin{equation}
\frac{\partial  }{\partial y}
\left(
    \frac{\partial h }{\partial y}
\right)
=
\frac{\partial  }{\partial y}
\left(
\frac{d h} { du}
\right)
\frac{\partial u }{\partial y}
+
\frac{d h} { du}
\frac{\partial  }{\partial y}
\left(
\frac{\partial u }{\partial y}
\right)
\end{equation}

\begin{equation}
\frac{\partial  }{\partial y}
\left(
    \frac{\partial h }{\partial y}
\right)
=
\frac{\partial  }{\partial u}
\left(
\frac{d h} { du}
\right)
\frac{\partial u}{\partial y}
\frac{\partial u }{\partial y}
+
\frac{d h} { du}
\left(
\frac{\partial^2 u }{\partial y^2}
\right)
\end{equation}

\begin{equation}
\frac{\partial  }{\partial y}
\left(
    \frac{\partial h }{\partial y}
\right)
=
\frac{d^2 h} { d u^2}
\left(
\frac{\partial u}{\partial y}
\right)^2
+
\frac{d h} { du}
\left(
\frac{\partial^2 u }{\partial y^2}
\right)
\end{equation}

Then, in the momemtum and energy equation, neglecting the pressure grandient
and assuming $Pr=1$,  

\begin{equation}
(\rho u)\frac{\partial u}{\partial x}    
+
(\rho v)\frac{\partial u}{\partial y}    
=
\frac{\partial  }{\partial y}\left( \mu\frac{\partial  u }{\partial y}\right)
\end{equation}


\begin{equation}
(\rho u)\frac{\partial h}{\partial x}    
+
(\rho v)
\frac{d h} { dy}
=
\frac{\partial  }{\partial y}
\left( 
\mu
\frac{d h} { dy}
\right)
+
\mu
\left(
    \frac{\partial u }{\partial y}
\right)^2 
\end{equation}

Opening the energy dissipation term

\begin{equation}
(\rho u)\frac{\partial h}{\partial x}    
+
(\rho v)
\frac{d h} { dy}
=
\mu
\frac{\partial  }{\partial y}
\left( 
\frac{d h} { dy}
\right)
+
\frac{d h} { dy}
\frac{\partial  \mu}{\partial y}
+
\mu
\left(
    \frac{\partial u }{\partial y}
\right)^2 
\end{equation}

And using the relation between $h$ and $u$ in the first term of the right hand side,

\begin{equation}
(\rho u)\frac{\partial h}{\partial x}    
+
(\rho v)
\frac{d h} { dy}
=
\mu
\frac{d h} { du}
\frac{\partial^2 u }{\partial y^2}
+
\mu
\frac{d ^2 h} { d u^2}
\left(
\frac{\partial u }{\partial y}
\right)^2
+
\frac{d h} { dy}
\frac{\partial  \mu}{\partial y}
+
\mu
\left(
    \frac{\partial u }{\partial y}
\right)^2 
\end{equation}

\begin{equation}
(\rho u)\frac{d h}{d u}    
 \frac{\partial u}{\partial x}    
+
(\rho v)
\frac{d h} { du}
 \frac{\partial u}{\partial y}    
 -
\mu
\frac{d h} { du}
\frac{\partial^2 u }{\partial y^2}
-
\frac{d h} { du}
\frac{\partial  u}{\partial y}
\frac{\partial  \mu}{\partial y}
=
\left[
    \frac{d ^2 h} { d u^2}
    +
    1
\right]
    \mu
    \left(
        \frac{\partial u }{\partial y}
    \right)^2 
\end{equation}

\begin{equation}
\frac{d h}{d u}    
\left[
    (\rho u)
     \frac{\partial u}{\partial x}    
    +
    (\rho v)
     \frac{\partial u}{\partial y}    
     -
    \frac{\partial }{\partial y}
    \left(
        \mu
        \frac{\partial u }{\partial y}
    \right)
\right]
=
\left[
    \frac{d ^2 h} { d u^2}
    +
    1
\right]
    \mu
    \left(
        \frac{\partial u }{\partial y}
    \right)^2 
\end{equation}

Using the momentum equation 

\begin{equation}
\left[
    \frac{d ^2 h} { d u^2}
    +
    1
\right]
    \mu
    \left(
        \frac{\partial u }{\partial y}
    \right)^2 
    =
    0
\end{equation}

Then 

\begin{equation}
    \frac{d ^2 h} { d u^2}
    =
    -
    1
\end{equation}

Integrating 

\begin{equation}
    h
    =
    -
    \frac{u^2}{2}
    +
    c_1
    u
    +
    c_2
\end{equation}

For mixing layer these constant 
can be found, using the values 
at the boundaries. For the upper  stream: 

\begin{equation}
    h_1
    =
    -
    \frac{u_1^2}{2}
    +
    c_1
    u_1
    +
    c_2
\label{h1}
\end{equation}

Similarly for the lower stream 

\begin{equation}
    h_2
    =
    -
    \frac{u_2^2}{2}
    +
    c_1
    u_2
    +
    c_2
\label{h2}
\end{equation}

There are two unknowns and two 
equations.  
For the equation \ref{h1}

\begin{equation}
    c_2
    =
    h_1
    +
    \frac{u_1^2}{2}
    -
    c_1
    u_1
\end{equation}

Using it in \ref{h2} 

\begin{equation}
    h_2
    =
    -
    \frac{u_2^2}{2}
    +
    c_1
    u_2
    +
    h_1
    +
    \frac{u_1^2}{2}
    -
    c_1
    u_1
\end{equation}

\begin{equation}
    c_1
    \left(
        u_2
        -
        u_1
    \right)
    =
    h_2
    -
    h_1
    +
    \frac{u_2^2}{2}
    -
    \frac{u_1^2}{2}
\end{equation}

\begin{equation}
    c_1
    =
    \frac{(h_2-h_1)}{(u_2-u_1)}
    +
    \frac{1}{2}
    (u_2+u_1)
\end{equation}

and 

\begin{equation}
    c_2
    =
    h_1
    +
    \frac{u_1^2}{2}
    -
    \left[
        \frac{(h_2-h_1)}{(u_2-u_1)}
        +
        \frac{1}{2}
        (u_2+u_1)
    \right]
    u_1
\end{equation}

Therefore 

\begin{equation}
    h
    =
    -
    \frac{u^2}{2}
    +
    c_1
    u
    +
    c_2
\end{equation}

\begin{equation}
    h
    =
    -
    \frac{u^2}{2}
    +
    \left(
        \frac{(h_2-h_1)}{(u_2-u_1)}
        +
        \frac{1}{2}
        (u_2+u_1)
    \right)
    u
    +
    h_1
    +
    \frac{u_1^2}{2}
    -
    \left[
        \frac{(h_2-h_1)}{(u_2-u_1)}
        +
        \frac{1}{2}
        (u_2+u_1)
    \right]
    u_1
\end{equation}


\begin{equation}
    h
    =
    \frac{u_1^2}{2}
    -
    \frac{u^2}{2}
    +
    \left(
        \frac{(h_2-h_1)}{(u_2-u_1)}
        +
        \frac{1}{2}
        (u_2+u_1)
    \right)
    (u-u_1)
    +
    h_1
\end{equation}

\begin{equation}
    h
    =
    \frac{1}{2}
    (u_1+u)
    (u_1-u)
    +
    \left(
        \frac{(h_2-h_1)}{(u_2-u_1)}
        +
        \frac{1}{2}
        (u_2+u_1)
    \right)
    (u-u_1)
    +
    h_1
\end{equation}

\begin{equation}
    h
    =
    (u_1-u)
    \left[
        \frac{1}{2}
        (u_1+u)
        -
        \left(
            \frac{(h_2-h_1)}{(u_2-u_1)}
            +
            \frac{1}{2}
            (u_2+u_1)
        \right)
    \right]
    +
    h_1
\end{equation}

\begin{equation}
    h
    =
    (u_1-u)
    \left[
        \frac{1}{2}
        (u-u_2)
        -
        \left(
            \frac{(h_2-h_1)}{(u_2-u_1)}
        \right)
    \right]
    +
    h_1
\end{equation}

\begin{equation}
    h
    =
    \frac{1}{2}
    (u-u_2)
    (u_1-u)
    -
    h_2
    \frac{(u_1-u)}{(u_2-u_1)}
    +
    h_1
    \left(
        \frac{(u_1-u)}{(u_2-u_1)}
    \right)
    +
    h_1
\end{equation}

\begin{equation}
    h
    =
    \frac{1}{2}
    (u-u_2)
    (u_1-u)
    -
    h_2
    \frac{(u_1-u)}{(u_2-u_1)}
    +
    h_1
    \left( 
        1
        +
        \left(
            \frac{(u_1-u)}{(u_2-u_1)}
        \right)
    \right)
\end{equation}

\begin{equation}
    h
    =
    \frac{1}{2}
    (u-u_2)
    (u_1-u)
    -
    h_2
    \frac{(u_1-u)}{(u_2-u_1)}
    +
    h_1
       \frac{(u_2-u)}{(u_2-u_1)}
\end{equation}

Assuming $c_p=$ constant, the enthalpy can be related 
with the temperatute with 
the relation $h=c_pT$ 

\begin{equation}
    T
    =
    \frac{1}{2}
    \frac{1}{c_p}
    (u-u_2)
    (u_1-u)
    -
    T_2
    \frac{(u_1-u)}{(u_2-u_1)}
    +
    T_1
    \frac{(u_2-u)}{(u_2-u_1)}
\end{equation}

\begin{equation}
    T
    =
    T_1
    \frac{(u-u_2)}{(u_1-u_2)}
    + 
    T_2
    \frac{(u_1-u)}{(u_1-u_2)}
    +
    \frac{1}{2}
    \frac{1}{c_p}
    (u_1-u)
    (u-u_2)
\end{equation}

A non-dimensional form can be obtained using 
$\overline{T}=T/T_\infty=T/T_1$ and $U=u/U_\infty$, 
$T_2=T_{-\infty}$.  

\begin{equation}
    \overline{T}
    T_\infty
    =
    T_\infty
    \frac{(u-u_2)}{(u_1-u_2)}
    + 
    T_{-\infty}
    \frac{(u_1-u)}{(u_1-u_2)}
    +
    \frac{1}{2}
    \frac{U_\infty^2}{c_p}
    (u_\infty-u)
    (u-u_{-\infty})
\end{equation}

\begin{equation}
    T
    =
    \frac{(u-u_{-\infty})}{(u_{\infty}-u_{-\infty})}
    + 
    \frac{T_{-\infty}}{T_{\infty}}
    \frac{(u_{\infty}-u)}{(u_{\infty}-u_{-\infty})}
    +
    \frac{1}{2}
    \frac{U_\infty^2}{T_{\infty}c_p}
    (u_{\infty}-u)
    (u-u_{-\infty})
\end{equation}

\begin{equation}
    T
    =
    \frac{(u-u_{-\infty})}{(u_{\infty}-u_{-\infty})}
    + 
    \frac{T_{-\infty}}{T_{\infty}}
    \frac{(u_{\infty}-u)}{(u_{\infty}-u_{-\infty})}
    +
    \frac{1}{2}
    \frac{\gamma R}{c_p}
    \frac{U_\infty^2}{\gamma R T_{\infty}}
    (u_{\infty}-u)
    (u-u_{-\infty})
\end{equation}

\begin{equation}
    \frac{R}{c_p}
    =
    \frac{c_p-c_v}{c_p}
    =
    1-\frac{1}{\gamma}
    =
    \frac{\gamma-1}{\gamma}
\end{equation}

$\beta_t=T_{-\infty}/T_{\infty}$

\begin{equation}
    T
    =
    \frac{(u-u_{-\infty})}{(u_{\infty}-u_{-\infty})}
    + 
    \beta_t
    \frac{(u_{\infty}-u)}{(u_{\infty}-u_{-\infty})}
    +
    \frac{1}{2}
    M^2(\gamma-1) (u_{\infty}-u)
    (u-u_{-\infty})
\end{equation}

Using $T=T/T_\infty$ and $U=U/a_\infty$.  

\begin{equation}
    T
    =
    \frac{(u-u_2)}{(u_1-u_2)}
    + 
    \beta_t
    \frac{(u_1-u)}{(u_1-u_2)}
    +
    \frac{\gamma-1}{2}
    (u_1-u)
    (u-u_2)
\end{equation}
    

%using T1 and T2
%using T1 and T2
%using T1 and T2
%
%The last term and the temperature are   dimensional  and depend 
%on the non dimensional 
%parameters.
%
%Using $T=T/T_1$ and $U=U/U_1$.  
%
%\begin{equation}
%    T
%    T_1
%    =
%    T_1
%    \frac{(u-u_2)}{(u_1-u_2)}
%    + 
%    T_2
%    \frac{(u_1-u)}{(u_1-u_2)}
%    +
%    \frac{1}{2}
%    \frac{U_1^2}{c_p}
%    (u_1-u)
%    (u-u_2)
%\end{equation}
%
%\begin{equation}
%    T
%    =
%    \frac{(u-u_2)}{(u_1-u_2)}
%    + 
%    \frac{T_2}{T_1}
%    \frac{(u_1-u)}{(u_1-u_2)}
%    +
%    \frac{1}{2}
%    \frac{U_1^2}{T_1c_p}
%    (u_1-u)
%    (u-u_2)
%\end{equation}
%
%\begin{equation}
%    T
%    =
%    \frac{(u-u_2)}{(u_1-u_2)}
%    + 
%    \frac{T_2}{T_1}
%    \frac{(u_1-u)}{(u_1-u_2)}
%    +
%    \frac{1}{2}
%    \frac{U_1^2\gamma_1 R}{\gamma_1RT_1c_p} (u_1-u)
%    (u-u_2)
%\end{equation}
%
%\begin{equation}
%    T
%    =
%    \frac{(u-u_2)}{(u_1-u_2)}
%    + 
%    \frac{T_2}{T_1}
%    \frac{(u_1-u)}{(u_1-u_2)}
%    +
%    \frac{1}{2}
%    \frac{M^2\gamma_1 R}{c_p} (u_1-u)
%    (u-u_2)
%\end{equation}
%
%\begin{equation}
%    \frac{R}{c_p}
%    =
%    \frac{c_p-c_v}{c_p}
%    =
%    1-\frac{1}{\gamma}
%    =
%    \frac{\gamma-1}{\gamma}
%\end{equation}
%
%$\beta_t=T_2/T_1$
%
%\begin{equation}
%    T
%    =
%    \frac{(u-u_2)}{(u_1-u_2)}
%    + 
%    \beta_t
%    \frac{(u_1-u)}{(u_1-u_2)}
%    +
%    \frac{1}{2}
%    M^2(\gamma_1-1) (u_1-u)
%    (u-u_2)
%\end{equation}
%
%Using $T=T/T_1$ and $U=U/a_1$.  
%
%\begin{equation}
%    T
%    =
%    \frac{(u-u_2)}{(u_1-u_2)}
%    + 
%    \beta_t
%    \frac{(u_1-u)}{(u_1-u_2)}
%    +
%    \frac{\gamma-1}{2}
%    (u_1-u)
%    (u-u_2)
%\end{equation}
%
%

\section{\textbf{Self-Similarity solution for Reacting Mixing Layer equations}}
%\section{\textbf{Self-Similarity solution for Reacting Mixing Layer equations}}

To find a similarity solutions for a reacting mixing layer equations, firstly we 
need to use the Howarth-Dorodnitsyn \ref{h-d} transformation to find am 
incompressible form of the compressibles equations. 
\\
\\
Definig the stream function for compressible flow as: 
\\
\\
\begin{equation}
\frac{\partial \Psi}{\partial x}=-\varrho V \qquad
\frac{\partial \Psi}{\partial y}=\varrho U,
\end{equation} 
\\
\\
The mass conservation equation is identically satisfied by the introduction 
of  this stream function. Now defining a vertical stretching as:
\\
\\
\begin{equation}
x=\overline{x}\qquad
y=\int_0^{\overline{y}}\varrho d\overline{y}%\qquad 
%\hat{V}=\varrho V+U\int_o^{\overline{y}}\varrho_{\overline{x}}\qquad
%\label{h-d}
\end{equation} 
\\
\\
and using the chain rule:
\\
\\
\begin{equation}
f(\overline{x},\overline{y})\qquad
\overline{x}=\overline{x}(x,y)\qquad
\overline{y}=\overline{y}(x,y)
\end{equation} 
\\
\\
\begin{equation}
\frac{\partial f}{\partial \overline{x}}=
\frac{\partial f}{\partial x}
\frac{\partial x}{\partial \overline{x}}+
\frac{\partial f}{\partial y}
\frac{\partial y}{\partial \overline{x}}
\end{equation} 
\\
\\
\begin{equation}
\frac{\partial f}{\partial \overline{y}}=
\frac{\partial f}{\partial x}
\frac{\partial x}{\partial \overline{y}}+
\frac{\partial f}{\partial y}
\frac{\partial y}{\partial \overline{y}}
\end{equation} 
\\
\\
\begin{equation}
\frac{\partial f}{\partial \overline{x}}=
\frac{\partial f}{\partial x}+
%\frac{\partial x}{\partial \overline{x}}+
\frac{\partial f}{\partial y}
\int_0^{\overline{y}}\frac{\partial \varrho }{\partial \overline{x}}dy
%\label{fx}
\end{equation} 
\\
\\
\begin{equation}
\frac{\partial f}{\partial \overline{y}}=
%\frac{\partial f}{\partial x}
%\frac{\partial x}{\partial \overline{y}}+
\frac{\partial f}{\partial y}
%\frac{\partial y}{\partial \overline{y}}
\varrho
%\label{fy}
\end{equation} 
\\
\\
so that:
\\
\\
\begin{equation}
U=\frac{\partial \Psi}{\partial y}=\hat{U},
\end{equation} 
\\
\\
that have the same form as incompresible flow, 
in the same way we can defined:
\\
\\
\begin{equation}
\hat{V}=-\frac{\partial \Psi}{\partial x},
\end{equation} 
\\
\\
Using the definition of $V$:
\\
\\
\begin{equation}
\varrho V=-
\frac{\partial \Psi}{\partial \overline{x}}
\end{equation} 
\\
\\
with  \ref{fx} we can write: 
\\
\\
\begin{equation}
  \varrho V=-
  \left(
       \frac{\partial \Psi}{\partial x}+
       \frac{\partial \Psi}{\partial y}
       \frac{\partial y}{\partial \overline{x}}
  \right)
\end{equation} 
\\
\\
\begin{equation}
  \varrho V=
  \left(
       \hat{V}
       -U
       \frac{\partial y}{\partial \overline{x}}
  \right)
\end{equation} 
\\
\\
\begin{equation}
  \hat{V}=
  \varrho V
  +U
  \frac{\partial y}{\partial \overline{x}}
\label {hatv}
\end{equation} 
\\
\\
Now we can used the equation \ref{h-d}, \ref{fx}, \ref{fy} and \ref{hatv} in the 
Mixing layer equations, \ref{mass2} to  
\ref{g2}, obtain:
\\
\\
\textbf{Mass conservation}
\\
\\
\begin{equation} 
\frac{\partial}{\partial \overline{x}}(\varrho U)+
\frac{\partial}{\partial \overline{y}}(\varrho V)=0
\end{equation} 
\\
\\
\begin{equation} 
\frac{\partial}{\partial x}(\varrho U)+
\frac{\partial}{\partial y}(\varrho U)
\frac{\partial y}{\partial \overline{x}}+
\varrho\frac{\partial}{\partial y}(\varrho V)=0
\end{equation} 
\\
\\
\begin{equation} 
\frac{\partial}{\partial x}(\varrho U)+
\frac{\partial}{\partial y}(\varrho U)
\frac{\partial y}{\partial \overline{x}}+
\varrho\frac{\partial}{\partial y}
  \left(
       \hat{V}
       -U
       \frac{\partial y}{\partial \overline{x}}
  \right)
=0
\end{equation} 
\\
\\
\begin{equation} 
U\frac{\partial \varrho }{\partial x}+
\varrho\frac{\partial U}{\partial x}+
U\frac{\partial \varrho }{\partial y}
\frac{\partial y}{\partial \overline{x}}+
\varrho\frac{\partial U}{\partial y}
\frac{\partial y}{\partial \overline{x}}+
\varrho\frac{\partial \hat{V}}{\partial y}
-\varrho\frac{\partial }{\partial \overline{y}}
       \left(
       U\frac{\partial y}{\partial \overline{x}}
       \right)
=0
\end{equation} 
\\
\\
\begin{equation} 
U\frac{\partial \varrho }{\partial x}+
\varrho\frac{\partial U}{\partial x}+
U\frac{\partial \varrho }{\partial y}
\frac{\partial y}{\partial \overline{x}}+
\varrho\frac{\partial U}{\partial y}
\frac{\partial y}{\partial \overline{x}}+
\varrho\frac{\partial \hat{V}}{\partial \overline{y}}
-\varrho\frac{\partial U}{\partial \overline{y}}
       \frac{\partial y}{\partial \overline{x}}
-\varrho U\frac{\partial }{\partial y}
       \left(
       \frac{\partial y}{\partial \overline{x}}
       \right)
=0
\end{equation} 
\\
\\
\begin{equation} 
U\frac{\partial \varrho }{\partial x}+
\varrho\frac{\partial U}{\partial x}+
U\frac{\partial \varrho }{\partial y}
\frac{\partial y}{\partial \overline{x}}+
\cancel{
\varrho\frac{\partial U}{\partial y}
\frac{\partial y}{\partial \overline{x}}}^0+
\varrho\frac{\partial \hat{V}}{\partial y}
-\cancel{
         \varrho\frac{\partial U}{\partial \overline{y}}
         \frac{\partial y}{\partial \overline{x}}
        }
-\varrho U\frac{\partial }{\partial y}
       \left(
       \frac{\partial y}{\partial \overline{x}}
       \right)
=0
\end{equation} 
\\
\\
\begin{equation} 
U\frac{\partial \varrho }{\partial x}+
\varrho\frac{\partial U}{\partial x}+
U\frac{\partial \varrho }{\partial y}
\frac{\partial y}{\partial \overline{x}}+
\varrho\frac{\partial \hat{V}}{\partial y}
-\varrho U\frac{\partial }{\partial y}
       \left(
       \frac{\partial y}{\partial \overline{x}}
       \right)
=0
\end{equation} 
\\
\\
the last term of the above equation  can be write as:
\\
\\
\begin{equation} 
\varrho U\frac{\partial }{\partial y}
       \left(
       \frac{\partial y}{\partial \overline{x}}
       \right)
=
U\varrho \frac{\partial }{\partial y}
\left(
      \frac{\partial y}{\partial \overline{x}}
\right)
=
U\frac{\partial }{\partial \overline{y}}
\left(
      \frac{\partial y}{\partial \overline{x}}
\right)
=
U\frac{\partial \varrho}{\partial \overline{x}}
=
U
\left(
\frac{\partial \varrho}{\partial x}+
\frac{\partial \varrho}{\partial y}
\frac{\partial y}{\partial \overline{x}}
\right)
\end{equation} 
\\
\\
hence:
\\
\\
\begin{equation} 
\varrho\frac{\partial U}{\partial x}+
\varrho\frac{\partial \hat{V}}{\partial y}
=0
\end{equation} 
\\
\\
\begin{equation} 
\frac{\partial \hat{U}}{\partial x}+
\frac{\partial \hat{V}}{\partial y}
=0
\end{equation} 
\\
\\
\textbf{Species Conservation}
\\
\\
Oxidizer conservation
\\
\\
\begin{equation} 
\varrho U\frac{\partial \psi_{o}}{\partial\overline{x}}+
\varrho V\frac{\partial \psi_{o}}{\partial\overline{y}}=
\frac{1}{Le_o}\frac{\partial}{\partial \overline{y}}\left(\rho D^o\frac{\partial \psi_o}{\partial \overline{y}}\right)+
s^{o}\Omega
%\frac{\partial\rho D^m \tfrac{\partial\rho Y^m u_i}{\partial x_i}u_i}{\partial x_i}=
\end{equation} 
\\
\\
\begin{equation} 
\varrho U\frac{\partial \psi_{o}}{\partial x}+
\varrho U\frac{\partial \psi_{o}}{\partial y}
\frac{\partial y}{\partial \overline{x}}+
\varrho^2 V\frac{\partial \psi_{o}}{\partial y}=
\frac{\varrho}{Le_o}\frac{\partial}{\partial y}\left(\varrho^2 D^o\frac{\partial \psi_o}{\partial y}\right)+
s^{o}\Omega
%\frac{\partial\rho D^m \tfrac{\partial\rho Y^m u_i}{\partial x_i}u_i}{\partial x_i}=
\end{equation} 
\\
\\
\begin{equation} 
\varrho U\frac{\partial \psi_{o}}{\partial x}+
\varrho U\frac{\partial \psi_{o}}{\partial y}
\frac{\partial y}{\partial \overline{x}}+
\varrho^2 \frac{\partial \psi_{o}}{\partial y}
\left(
       \frac{1}{\varrho}\hat{V}-
       \frac{1}{\varrho}U\frac{\partial y}{\partial \overline{x}}
\right)
=
\frac{\varrho}{Le_o}\frac{\partial}{\partial y}\left(\varrho^2 D^o\frac{\partial \psi_o}{\partial y}\right)+
s^{o}\Omega
%\frac{\partial\rho D^m \tfrac{\partial\rho Y^m u_i}{\partial x_i}u_i}{\partial x_i}=
\end{equation} 
\\
\\
\begin{equation} 
\varrho U\frac{\partial \psi_{o}}{\partial x}+
\varrho U\frac{\partial \psi_{o}}{\partial y}
\frac{\partial y}{\partial \overline{x}}+
\varrho^2 \frac{\partial \psi_{o}}{\partial y}
\left(
       \frac{1}{\varrho}\hat{V}-
       \frac{1}{\varrho}U\frac{\partial y}{\partial \overline{x}}
\right)
=
\frac{\varrho}{Le_o}\frac{\partial}{\partial y}\left(\varrho^2 D^o\frac{\partial \psi_o}{\partial y}\right)+
s^{o}\Omega
%\frac{\partial\rho D^m \tfrac{\partial\rho Y^m u_i}{\partial x_i}u_i}{\partial x_i}=
\end{equation} 
\\
\\
\begin{equation} 
\varrho U\frac{\partial \psi_{o}}{\partial x}+
\varrho \hat{V}\frac{\partial \psi_{o}}{\partial y}
=
\frac{\varrho}{Le_o}\frac{\partial}{\partial y}\left(\varrho^2 D^o\frac{\partial \psi_o}{\partial y}\right)+
s^{o}\Omega
%\frac{\partial\rho D^m \tfrac{\partial\rho Y^m u_i}{\partial x_i}u_i}{\partial x_i}=
\end{equation} 
\\
\\
\begin{equation} 
 \hat{U}\frac{\partial \psi_{o}}{\partial x}+
 \hat{V}\frac{\partial \psi_{o}}{\partial y}
=
\frac{1}{Le_o}\frac{\partial}{\partial y}\left(\varrho^2 D^o\frac{\partial \psi_o}{\partial y}\right)+
\frac{s^{o}\Omega}{\varrho}
%\frac{\partial\rho D^m \tfrac{\partial\rho Y^m u_i}{\partial x_i}u_i}{\partial x_i}=
\end{equation} 
\\
\\
Fuel conservation
\\
\\
In the same way as in oxidizer conservartion, using the transform coordinates $x, y, \hat{U}$ and $\hat{V}$ in the Fuel
conservations species, we get: 
\\
\\
\begin{equation} 
 \hat{U}\frac{\partial \psi_{F}}{\partial x}+
 \hat{V}\frac{\partial \psi_{F}}{\partial y}
=
\frac{1}{Le_F}\frac{\partial}{\partial y}\left(\varrho^2 D^F\frac{\partial \psi_F}{\partial y}\right)+
\frac{s^{F}\Omega}{\varrho}
%\frac{\partial\rho D^m \tfrac{\partial\rho Y^m u_i}{\partial x_i}u_i}{\partial x_i}=
\end{equation} 
\\
\\
\textbf{Momemtum conservation}
\\
\\
\begin{equation} 
\varrho U\frac{\partial U}{\partial \overline{x}}+
\varrho V\frac{\partial U}{\partial \overline{y}}=
\frac{\partial}{\partial \overline{y}}\left(\mu\frac{\partial U}{\partial  \overline{y}}\right)
\end{equation} 
\\
\\
\begin{equation} 
\varrho U\frac{\partial U}{\partial x}+
\varrho U\frac{\partial U}{\partial y}
\frac{\partial y}{\partial \overline{x}}+
\varrho^2 V\frac{\partial U}{\partial y}=
\varrho\frac{\partial}{\partial y}\left(\mu\varrho\frac{\partial U}{\partial  y}\right)
\end{equation} 
\\
\\
\begin{equation} 
\varrho U\frac{\partial U}{\partial x}+
\varrho U\frac{\partial U}{\partial y}
\frac{\partial y}{\partial \overline{x}}+
\varrho^2 
\left(
       \frac{1}{\varrho}\hat{V}-
       \frac{1}{\varrho}U\frac{\partial y}{\partial \overline{x}}
\right)
\frac{\partial U}{\partial y}=
\varrho\frac{\partial}{\partial y}\left(\mu\varrho\frac{\partial U}{\partial  y}\right)
\end{equation} 
\\
\\
\begin{equation} 
\varrho U\frac{\partial U}{\partial x}+
\varrho \hat{V}\frac{\partial U}{\partial y}=
\varrho\frac{\partial}{\partial y}\left(\mu\varrho\frac{\partial U}{\partial  y}\right)
\end{equation} 
\\
\\
\begin{equation} 
 \hat{U}\frac{\partial \hat{U}}{\partial x}+
 \hat{V}\frac{\partial \hat{U}}{\partial y}=
\frac{\partial}{\partial y}\left(\mu\varrho\frac{\partial \hat{U}}{\partial  y}\right)
\end{equation} 
\\
\\
\textbf{Energy Equation}
\\
\\
\begin{equation}
\varrho U\frac{\partial }{\partial \overline{x}}(c_p \overline{T})+
\varrho V\frac{\partial }{\partial \overline{y}}(c_p \overline{T})=
(\gamma-1)M^2\mu\frac{\partial^2 U}{\partial \overline{y}^2}+
\frac{1}{Pr}\frac{\partial}{\partial \overline{y}}\left(k\frac{\partial \overline{T}}{\partial \overline{y}}\right)+
Q\Omega
\end{equation} 
\\
\\
Using the chain rule to change the second order derivative of $\overline{y}$ in the energy equation. 
\\
\\
\begin{equation}
\frac{\partial^2f}{\partial \overline{y}^2}=
\frac{\partial}{\partial y}
\left(
\frac{\partial f}{\partial y}
\frac{\partial y}{\partial \overline{y}}
\right)
\frac{\partial y}{\partial \overline{y}}
\end{equation} 
\\
\\
\begin{equation}
\frac{\partial^2f}{\partial \overline{y}^2}=
\frac{\partial ^2 f}{\partial y^2}
\left(
\frac{\partial y}{\partial \overline{y}}
\right)^2+
\frac{\partial f}{\partial y}
\frac{\partial ^2 y}{\partial \overline{y}^2}
\end{equation} 
\\
\\
\begin{equation}
\frac{\partial^2f}{\partial \overline{y}^2}=
\varrho^2\frac{\partial ^2 f}{\partial y^2}+
\varrho
\frac{\partial \varrho}{\partial y}
\frac{\partial f}{\partial y}
\end{equation} 
\\
\\
\begin{equation}
\begin{split}
\varrho U\frac{\partial }{\partial x}(c_p \overline{T})+
\varrho U\frac{\partial }{\partial y}(c_p \overline{T})
\frac{\partial y}{\partial \overline{x}}+
\varrho^2 V
\frac{\partial }{\partial y}(c_p \overline{T})=
\\
\varrho^2(\gamma-1)M^2\mu\frac{\partial^2 U}{\partial y^2}+
\varrho
\frac{\partial \varrho}{\partial y}
(\gamma-1)M^2\mu\frac{\partial U}{\partial y}+
\frac{\varrho}{Pr}\frac{\partial}{\partial y}\left(k\varrho\frac{\partial \overline{T}}{\partial y}\right)+
Q\Omega
\end{split}
\end{equation} 
\\
\\
\begin{equation}
\begin{split}
\varrho U\frac{\partial }{\partial x}(c_p \overline{T})+
\varrho U\frac{\partial }{\partial y}(c_p \overline{T})
\frac{\partial y}{\partial \overline{x}}+
\varrho^2 
\left(
       \frac{1}{\varrho}\hat{V}-
       \frac{1}{\varrho}U\frac{\partial y}{\partial \overline{x}}
\right)
\frac{\partial }{\partial y}(c_p \overline{T})=
\\
\varrho^2(\gamma-1)M^2\mu\frac{\partial^2 U}{\partial y^2}+
\varrho
\frac{\partial \varrho}{\partial y}
(\gamma-1)M^2\mu\frac{\partial U}{\partial y}+
\frac{\varrho}{Pr}\frac{\partial}{\partial y}\left(k\varrho\frac{\partial \overline{T}}{\partial y}\right)+
Q\Omega
\end{split}
\end{equation} 
\\
\\
\begin{equation}
\begin{split}
\hat{U}\frac{\partial }{\partial x}(c_p \overline{T})+
\hat{V}
\frac{\partial }{\partial y}(c_p \overline{T})=
\varrho(\gamma-1)M^2\mu\frac{\partial^2 \hat{U}}{\partial y^2}+
\\
\frac{\partial \varrho}{\partial y}
(\gamma-1)M^2\mu\frac{\partial \hat{U}}{\partial y}+
\frac{1}{Pr}\frac{\partial}{\partial y}\left(k\varrho\frac{\partial \overline{T}}{\partial y}\right)+
\frac{Q\Omega}{\varrho}
\end{split}
\end{equation} 
\\
\\
\begin{equation}
\hat{U}\frac{\partial }{\partial x}(c_p \overline{T})+
\hat{V}
\frac{\partial }{\partial y}(c_p \overline{T})=
(\gamma-1)M^2\mu
\frac{\partial}{\partial y}
\left(
\varrho\frac{\partial \hat{U}}{\partial y}
\right)+
\frac{1}{Pr}\frac{\partial}{\partial y}\left(k\varrho\frac{\partial \overline{T}}{\partial y}\right)+
\frac{Q\Omega}{\varrho}
\end{equation} 
\\
\\
To transform completely the compressible equations to incompressible form we need to used the
Chapman's approximate viscosity law:
\\
\\
\begin{equation} 
\mu=C_w T
\end{equation} 
\\
\\
Multiply by $\varrho$ and using the gas perfect equation, we get:
\\
\\
\begin{equation} 
\mu\varrho=C_w
\end{equation} 
\\
\\
where $C_w$  is a function of the temperature $C_w=C_w(T)$ . 
\\
\\
\begin{equation} 
\frac{\partial \hat{U}}{\partial x}+
\frac{\partial \hat{V}}{\partial y}
=0
\end{equation} 
\\
\\
\begin{equation} 
 \hat{U}\frac{\partial \psi_{o}}{\partial x}+
 \hat{V}\frac{\partial \psi_{o}}{\partial y}
=
\frac{1}{Le_o}\frac{\partial}{\partial y}\left(\varrho^2 D^o\frac{\partial \psi_o}{\partial y}\right)+
\frac{s^{o}\Omega}{\varrho}
%\frac{\partial\rho D^m \tfrac{\partial\rho Y^m u_i}{\partial x_i}u_i}{\partial x_i}=
\end{equation} 
\\
\\
\begin{equation} 
 \hat{U}\frac{\partial \psi_{F}}{\partial x}+
 \hat{V}\frac{\partial \psi_{F}}{\partial y}
=
\frac{1}{Le_F}\frac{\partial}{\partial y}\left(\varrho^2 D^F\frac{\partial \psi_F}{\partial y}\right)+
\frac{s^{F}\Omega}{\varrho}
%\frac{\partial\rho D^m \tfrac{\partial\rho Y^m u_i}{\partial x_i}u_i}{\partial x_i}=
\end{equation} 
\\
\\
\begin{equation} 
 \hat{U}\frac{\partial \hat{U}}{\partial x}+
 \hat{V}\frac{\partial \hat{U}}{\partial y}=
\frac{\partial}{\partial y}\left(C_w(T)\frac{\partial \hat{U}}{\partial  y}\right)
\end{equation} 
\\
\\
\begin{equation}
\hat{U}\frac{\partial }{\partial x}(c_p \overline{T})+
\hat{V}
\frac{\partial }{\partial y}(c_p \overline{T})=
(\gamma-1)M^2\mu
\frac{\partial}{\partial y}
\left(
\varrho\frac{\partial \hat{U}}{\partial y}
\right)+
\frac{1}{Pr}\frac{\partial}{\partial y}\left(k\varrho\frac{\partial \overline{T}}{\partial y}\right)+
\frac{Q\Omega}{\varrho}
\end{equation} 

\section{Crocco Busemann}

In this section a special solution to boundary layer equation is found using the definition of the 
total enthalpy.

From the boundary layer equation in two dimensional flow 

\begin{equation}
\frac{\partial }{\partial x}(\rho u)    
+
\frac{\partial }{\partial y}(\rho v)    
=
0
\end{equation}

\begin{equation}
(\rho u)\frac{\partial u}{\partial x}    
+
(\rho v)\frac{\partial u}{\partial y}    
=
-
\frac{\partial  P}{\partial x}
+
\frac{\partial  }{\partial y}\left( \mu\frac{\partial  u }{\partial y}\right)
\end{equation}

\begin{equation}
(\rho u)\frac{\partial h}{\partial x}    
+
(\rho v)\frac{\partial h}{\partial y}    
=
u
\frac{\partial  P}{\partial x}
+
\frac{\partial  }{\partial y}
\left( \frac{\mu}{Pr}\frac{\partial  h }{\partial y}\right)
+
\mu
\left(
    \frac{\partial u }{\partial y}
\right)^2 
\end{equation}

where the Prantdl number is defined as $Pr=\mu c_p/k$.


Since the total enthalpy is defined as 

\begin{equation}
H=h+\frac{1}{2}u^2
\end{equation}

the total enthalpy equation using both 
the energy and momentum equation is  

\begin{equation}
(\rho u)\frac{\partial H}{\partial x}    
+
(\rho v)\frac{\partial H}{\partial y}    
=
\frac{\partial  }{\partial y}
\left( \frac{\mu}{Pr}\frac{\partial  H }{\partial y}\right)
+
\frac{\partial  }{\partial y}
\left[
    \left(
    1-\frac{1}{Pr}
    \right)
 \mu u\frac{\partial  u }{\partial y}
 \right]
\end{equation}

This equation allows a simplest solution  
$H=h+\frac{1}{2}u^2=$ constant as long as $Pr=1$, which leads to  

\begin{equation}
\frac{\partial  H }{\partial y}
=
\frac{\partial  h_w }{\partial y}
=
0
\end{equation}

Representing the not heat transfer at the wall.

$Pr=1$ implies in a perfect balance between viscous 
dissipation and heat conduction so as keep the 
the stagnation enthalpy constant in adiabatic boundary layer and also is a good approximation for gases.

Another solution can be obtained if 
the pressure gradient is neglected
in the boundary layer equation.  
With this, the momentum equation and the energy equation get very similar,
it seems as if u and h could be 
interchanged except for the dissipation term. 


\begin{equation}
(\rho u)\frac{\partial u}{\partial x}    
+
(\rho v)\frac{\partial u}{\partial y}    
=
\frac{\partial  }{\partial y}\left( \mu\frac{\partial  u }{\partial y}\right)
\end{equation}

\begin{equation}
(\rho u)\frac{\partial h}{\partial x}    
+
(\rho v)\frac{\partial h}{\partial y}    
=
\frac{\partial  }{\partial y}
\left( \frac{\mu}{Pr}\frac{\partial  h }{\partial y}\right)
+
\mu
\left(
    \frac{\partial u }{\partial y}
\right)^2 
\end{equation}

Then a solution of the form 

\begin{equation}
\frac{\partial h }{\partial y}
=
\frac{d h} { du}
\frac{\partial u }{\partial y}
\end{equation}


\begin{equation}
\frac{\partial  }{\partial y}
\left(
    \frac{\partial h }{\partial y}
\right)
=
\frac{\partial  }{\partial y}
\left(
\frac{d h} { du}
\frac{\partial u }{\partial y}
\right)
=
\frac{\partial  }{\partial u}
\left(
\frac{d h} { du}
\frac{\partial u }{\partial y}
\right)
\frac{\partial u }{\partial y}
\end{equation}

\begin{equation}
\frac{\partial  }{\partial y}
\left(
    \frac{\partial h }{\partial y}
\right)
=
\frac{\partial  }{\partial u}
\left(
\frac{d h} { du}
\frac{\partial u }{\partial y}
\right)
\frac{\partial u }{\partial y}
=
\frac{d h} { du}
\frac{\partial u^2 }{\partial y^2}
+
\frac{d ^2 h} { d u^2}
\left(
\frac{\partial u }{\partial y}
\right)^2
\end{equation}
 
\begin{equation}
(\rho u)\frac{\partial u}{\partial x}    
+
(\rho v)\frac{\partial u}{\partial y}    
=
\frac{\partial  }{\partial y}\left( \mu\frac{\partial  u }{\partial y}\right)
\end{equation}

\begin{equation}
(\rho u)\frac{\partial h}{\partial x}    
+
(\rho v)
\frac{d h} { du}
\frac{\partial u }{\partial y}
=
\frac{\partial  }{\partial y}
\left( \frac{\mu}{Pr}
\frac{d h} { du}
\frac{\partial u }{\partial y}
\right)
+
\mu
\left(
    \frac{\partial u }{\partial y}
\right)^2 
\end{equation}

Assuming $Pr=1$

\begin{equation}
(\rho u)\frac{\partial h}{\partial x}    
+
(\rho v)
\frac{d h} { dy}
=
\frac{\partial  }{\partial y}
\left( 
\mu
\frac{d h} { dy}
\right)
+
\mu
\left(
    \frac{\partial u }{\partial y}
\right)^2 
\end{equation}

\begin{equation}
(\rho u)\frac{\partial h}{\partial x}    
+
(\rho v)
\frac{d h} { dy}
=
\mu
\frac{\partial  }{\partial y}
\left( 
\frac{d h} { dy}
\right)
+
\frac{d h} { dy}
\frac{\partial  \mu}{\partial y}
+
\mu
\left(
    \frac{\partial u }{\partial y}
\right)^2 
\end{equation}

\begin{equation}
(\rho u)\frac{\partial h}{\partial x}    
+
(\rho v)
\frac{d h} { dy}
=
\mu
\frac{d h} { du}
\frac{\partial u^2 }{\partial y^2}
+
\mu
\frac{d ^2 h} { d u^2}
\left(
\frac{\partial u }{\partial y}
\right)^2
+
\frac{d h} { dy}
\frac{\partial  \mu}{\partial y}
+
\mu
\left(
    \frac{\partial u }{\partial y}
\right)^2 
\end{equation}

\begin{equation}
(\rho u)\frac{d h}{d u}    
 \frac{\partial u}{\partial x}    
+
(\rho v)
\frac{d h} { du}
 \frac{\partial u}{\partial y}    
 -
\mu
\frac{d h} { du}
\frac{\partial u^2 }{\partial y^2}
-
\frac{d h} { du}
\frac{\partial  u}{\partial y}
\frac{\partial  \mu}{\partial y}
=
\left[
    \frac{d ^2 h} { d u^2}
    +
    1
\right]
    \mu
    \left(
        \frac{\partial u }{\partial y}
    \right)^2 
\end{equation}

\begin{equation}
\frac{d h}{d u}    
\left[
    (\rho u)
     \frac{\partial u}{\partial x}    
    +
    (\rho v)
     \frac{\partial u}{\partial y}    
     -
    \frac{\partial }{\partial y}
    \left(
        \mu
        \frac{\partial u }{\partial y}
    \right)
\right]
=
\left[
    \frac{d ^2 h} { d u^2}
    +
    1
\right]
    \mu
    \left(
        \frac{\partial u }{\partial y}
    \right)^2 
\end{equation}

Using the momentum equation 

\begin{equation}
\left[
    \frac{d ^2 h} { d u^2}
    +
    1
\right]
    \mu
    \left(
        \frac{\partial u }{\partial y}
    \right)^2 
    =
    0
\end{equation}

Then 

\begin{equation}
    \frac{d ^2 h} { d u^2}
    =
    -
    1
\end{equation}

Integrating 

\begin{equation}
    h
    =
    -
    \frac{u^2}{2}
    +
    c_1
    u
    +
    c_2
\end{equation}

For mixing layer these constant 
can be found, using the values 
at the boundaries. For the upper  stream: 

\begin{equation}
    h_1
    =
    -
    \frac{u_1^2}{2}
    +
    c_1
    u_1
    +
    c_2
\label{h1}
\end{equation}

Similarly for the lower stream 

\begin{equation}
    h_2
    =
    -
    \frac{u_2^2}{2}
    +
    c_1
    u_2
    +
    c_2
\label{h2}
\end{equation}

This is two unknown and two 
equations.  

For the equation \ref{h1}

\begin{equation}
    c_2
    =
    h_1
    +
    \frac{u_1^2}{2}
    -
    c_1
    u_1
\end{equation}

Using it in \ref{h2} 

\begin{equation}
    h_2
    =
    -
    \frac{u_2^2}{2}
    +
    c_1
    u_2
    +
    h_1
    +
    \frac{u_1^2}{2}
    -
    c_1
    u_1
\end{equation}

\begin{equation}
    c_1
    \left(
        u_2
        -
        u_1
    \right)
    =
    h_2
    -
    h_1
    +
    \frac{u_2^2}{2}
    -
    \frac{u_1^2}{2}
\end{equation}

\begin{equation}
    c_1
    =
    \frac{(h_2-h_1)}{(u_2-u_1)}
    +
    \frac{1}{2}
    (u_2+u_1)
\end{equation}

and 

\begin{equation}
    c_2
    =
    h_1
    +
    \frac{u_1^2}{2}
    -
    \left[
        \frac{(h_2-h_1)}{(u_2-u_1)}
        +
        \frac{1}{2}
        (u_2+u_1)
    \right]
    u_1
\end{equation}

Therefore 

\begin{equation}
    h
    =
    -
    \frac{u^2}{2}
    +
    c_1
    u
    +
    c_2
\end{equation}

\begin{equation}
    h
    =
    -
    \frac{u^2}{2}
    +
    \left(
        \frac{(h_2-h_1)}{(u_2-u_1)}
        +
        \frac{1}{2}
        (u_2+u_1)
    \right)
    u
    +
    h_1
    +
    \frac{u_1^2}{2}
    -
    \left[
        \frac{(h_2-h_1)}{(u_2-u_1)}
        +
        \frac{1}{2}
        (u_2+u_1)
    \right]
    u_1
\end{equation}


\begin{equation}
    h
    =
    \frac{u_1^2}{2}
    -
    \frac{u^2}{2}
    +
    \left(
        \frac{(h_2-h_1)}{(u_2-u_1)}
        +
        \frac{1}{2}
        (u_2+u_1)
    \right)
    (u-u_1)
    +
    h_1
\end{equation}

\begin{equation}
    h
    =
    \frac{1}{2}
    (u_1+u)
    (u_1-u)
    +
    \left(
        \frac{(h_2-h_1)}{(u_2-u_1)}
        +
        \frac{1}{2}
        (u_2+u_1)
    \right)
    (u-u_1)
    +
    h_1
\end{equation}

\begin{equation}
    h
    =
    (u_1-u)
    \left[
        \frac{1}{2}
        (u_1+u)
        -
        \left(
            \frac{(h_2-h_1)}{(u_2-u_1)}
            +
            \frac{1}{2}
            (u_2+u_1)
        \right)
    \right]
    +
    h_1
\end{equation}

\begin{equation}
    h
    =
    (u_1-u)
    \left[
        \frac{1}{2}
        (u-u_2)
        -
        \left(
            \frac{(h_2-h_1)}{(u_2-u_1)}
        \right)
    \right]
    +
    h_1
\end{equation}

\begin{equation}
    h
    =
    \frac{1}{2}
    (u-u_2)
    (u_1-u)
    -
    h_2
    \frac{(u_1-u)}{(u_2-u_1)}
    +
    h_1
    \left(
        \frac{(u_1-u)}{(u_2-u_1)}
    \right)
    +
    h_1
\end{equation}

\begin{equation}
    h
    =
    \frac{1}{2}
    (u-u_2)
    (u_1-u)
    -
    h_2
    \frac{(u_1-u)}{(u_2-u_1)}
    +
    h_1
    \left( 
        1
        +
        \left(
            \frac{(u_1-u)}{(u_2-u_1)}
        \right)
    \right)
\end{equation}

\begin{equation}
    h
    =
    \frac{1}{2}
    (u-u_2)
    (u_1-u)
    -
    h_2
    \frac{(u_1-u)}{(u_2-u_1)}
    +
    h_1
       \frac{(u_2-u)}{(u_2-u_1)}
\end{equation}

Assuming $c_p=$ constant, the enthalpy can be related 
with the temperatute with 
the relation $h=c_pT$ 

\begin{equation}
    T
    =
    \frac{1}{2}
    \frac{1}{c_p}
    (u-u_2)
    (u_1-u)
    -
    T_2
    \frac{(u_1-u)}{(u_2-u_1)}
    +
    T_1
    \frac{(u_2-u)}{(u_2-u_1)}
\end{equation}

\begin{equation}
    T
    =
    T_1
    \frac{(u-u_2)}{(u_1-u_2)}
    + 
    T_2
    \frac{(u_1-u)}{(u_1-u_2)}
    +
    \frac{1}{2}
    \frac{1}{c_p}
    (u_1-u)
    (u-u_2)
\end{equation}

The last term and the temperature are   dimensional  and depend 
on the non dimensional 
parameters.

Using $T=T/T_1$ and $U=U/U_1$.  

\begin{equation}
    T
    T_1
    =
    T_1
    \frac{(u-u_2)}{(u_1-u_2)}
    + 
    T_2
    \frac{(u_1-u)}{(u_1-u_2)}
    +
    \frac{1}{2}
    \frac{U_1^2}{c_p}
    (u_1-u)
    (u-u_2)
\end{equation}

\begin{equation}
    T
    =
    \frac{(u-u_2)}{(u_1-u_2)}
    + 
    \frac{T_2}{T_1}
    \frac{(u_1-u)}{(u_1-u_2)}
    +
    \frac{1}{2}
    \frac{U_1^2}{T_1c_p}
    (u_1-u)
    (u-u_2)
\end{equation}

\begin{equation}
    T
    =
    \frac{(u-u_2)}{(u_1-u_2)}
    + 
    \frac{T_2}{T_1}
    \frac{(u_1-u)}{(u_1-u_2)}
    +
    \frac{1}{2}
    \frac{U_1^2\gamma_1 R}{\gamma_1RT_1c_p} (u_1-u)
    (u-u_2)
\end{equation}

\begin{equation}
    T
    =
    \frac{(u-u_2)}{(u_1-u_2)}
    + 
    \frac{T_2}{T_1}
    \frac{(u_1-u)}{(u_1-u_2)}
    +
    \frac{1}{2}
    \frac{M^2\gamma_1 R}{c_p} (u_1-u)
    (u-u_2)
\end{equation}

\begin{equation}
    \frac{R}{c_p}
    =
    \frac{c_p-c_v}{c_p}
    =
    1-\frac{1}{\gamma}
    =
    \frac{\gamma-1}{\gamma}
\end{equation}

$\beta_t=T_2/T_1$

\begin{equation}
    T
    =
    \frac{(u-u_2)}{(u_1-u_2)}
    + 
    \beta_t
    \frac{(u_1-u)}{(u_1-u_2)}
    +
    \frac{1}{2}
    M^2(\gamma_1-1) (u_1-u)
    (u-u_2)
\end{equation}

Using $T=T/T_1$ and $U=U/a_1$.  

\begin{equation}
    T
    =
    \frac{(u-u_2)}{(u_1-u_2)}
    + 
    \beta_t
    \frac{(u_1-u)}{(u_1-u_2)}
    +
    \frac{\gamma-1}{2}
    (u_1-u)
    (u-u_2)
\end{equation}
%----------------------------------------------------------------------------------------


\bibliographystyle{plainnat}
\bibliography{reference}
\
%----------------------------------------------------------------------------------------

\end{document}
