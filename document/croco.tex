%%\section{Crocco Busemann} %% %%

In this section a special solution to boundary layer equation is found using the definition of the total enthalpy.

From the boundary layer equation in two dimensional flow 

\begin{equation}
\frac{\partial }{\partial x}(\rho u)    
+
\frac{\partial }{\partial y}(\rho v)    
=
0
\end{equation}

\begin{equation}
(\rho u)\frac{\partial u}{\partial x}    
+
(\rho v)\frac{\partial u}{\partial y}    
=
-
\frac{\partial  P}{\partial x}
+
\frac{\partial  }{\partial y}\left( \mu\frac{\partial  u }{\partial y}\right)
\end{equation}

\begin{equation}
(\rho u)\frac{\partial h}{\partial x}    
+
(\rho v)\frac{\partial h}{\partial y}    
=
u
\frac{\partial  P}{\partial x}
+
\frac{\partial  }{\partial y}
\left( \frac{\mu}{Pr}\frac{\partial  h }{\partial y}\right)
+
\mu
\left(
    \frac{\partial u }{\partial y}
\right)^2 
\end{equation}

where the Prantdl number is defined as $Pr=\mu c_p/k$.
Since the total enthalpy is defined as 

\begin{equation}
H=h+\frac{1}{2}u^2
\end{equation}

the total enthalpy equation using both 
the energy and momentum equation. 

Multiplying the momentum equation by $u$ 

\begin{equation}
    u
(\rho u)\frac{\partial u}{\partial x}    
+
u
(\rho v)\frac{\partial u}{\partial y}    
=
-
u
\frac{\partial  P}{\partial x}
+
u
\frac{\partial  }{\partial y}\left( \mu\frac{\partial  u }{\partial y}\right)
\end{equation}

\begin{equation}
\frac{1}{2}  
\rho u\frac{\partial u^2}{\partial x}    
+
\frac{1}{2}  
\rho v\frac{\partial u^2}{\partial y}    
=
-
u
\frac{\partial  P}{\partial x}
+
u
\frac{\partial  }{\partial y}\left( \mu \frac{\partial  u }{\partial y}\right)
\end{equation}

The last term can be written as

\begin{equation}
u
\frac{\partial  }{\partial y}\left( \mu \frac{\partial  u }{\partial y}\right)
=
\frac{\partial  }{\partial y}\left( \mu u \frac{\partial  u }{\partial y}\right)
-
\mu
\left(\frac{\partial u }{\partial y}
\right)^2
,
\end{equation}


and adding it to the energy equation

\begin{equation}
\begin{split}
\frac{1}{2}  
(\rho u)\frac{\partial u^2}{\partial x}    
+
(\rho u)\frac{\partial h}{\partial x}    
+
\frac{1}{2}  
(\rho v)\frac{\partial u^2}{\partial y}    
+
(\rho v)\frac{\partial h}{\partial y}    
=
\\
\frac{\partial  }{\partial y}\left( \mu u \frac{\partial  u }{\partial y}\right)
-
\cancel{
\mu
\left(\frac{\partial u }{\partial y}
\right)^2
}
+
\frac{\partial  }{\partial y}
\left( \frac{\mu}{Pr}\frac{\partial  h }{\partial y}\right)
+
\cancel{
    \mu
\left(
    \frac{\partial u }{\partial y}
\right)^2 
}
\end{split}
\end{equation}

\begin{equation}
(\rho u)\frac{\partial H}{\partial x}    
+
(\rho v)\frac{\partial H}{\partial y}    
=
\frac{\partial  }{\partial y}
\left[
\mu
\left(  
    u
    \frac{\partial  u }{\partial y}
    +
    \frac{1}{Pr}
    \frac{\partial  h }{\partial y}
\right)
\right]
\end{equation}

\begin{equation}
\frac{\partial h}{\partial y}
=
\frac{\partial  H}{\partial y}
-
u
\frac{\partial  u }{\partial y}
\end{equation}

\begin{equation}
(\rho u)\frac{\partial H}{\partial x}    
+
(\rho v)\frac{\partial H}{\partial y}    
=
\frac{\partial  }{\partial y}
\left\{
\mu
    \left[
    u
    \frac{\partial  u }{\partial y}
    +
    \frac{1}{Pr}
    \left(
    \frac{\partial  H}{\partial y}
    -
    u
    \frac{\partial  u }{\partial y}
    \right)
    \right]
\right\}
\end{equation}

\begin{equation}
(\rho u)\frac{\partial H}{\partial x}    
+
(\rho v)\frac{\partial H}{\partial y}    
=
\frac{\partial  }{\partial y}
\left[
\frac{\mu}{Pr}
\left(  
    \frac{\partial  H}{\partial y}
\right)
\right]
+
\frac{\partial  }{\partial y}
\left[
\mu
\left(  
    1-
    \frac{1}{Pr}
\right)
u
\frac{\partial  u }{\partial y}
\right]
\end{equation}


This equation allows a simplest solution  
$H=h+\frac{1}{2}u^2=$ constant as long as $Pr=1$ of a adiabatic wall,
which leads to  

\begin{equation}
\left.\frac{\partial T}{\partial y}\right|_{w}
=
\left.\frac{\partial h}{\partial y}\right|_{w}
=
\left.\frac{\partial H}{\partial y}\right|_{w}
=
0
\end{equation}

Representing the not heat transfer at the wall.

$Pr=1$ implies in a perfect balance between viscous 
dissipation and heat conduction so as keep the 
the stagnation enthalpy constant in adiabatic boundary layer 
and also is a good approximation for gases.
So velocity and temperature are smoothed by the same mechanism.


Another solution can be obtained if 
the pressure gradient is neglected
in the boundary layer equation.  
The momentum equation and the total enthalpy equation 
are identical.  Total enthalpy behaves like a passive scalar.
This implies a direct functional relationship between 
$H$ and $u$, $H=H(u)$.
With this, the momentum equation and the energy equation
get very similar,
it seems as if $u$ and $h$ could be 
interchanged except for the dissipation term. 


\begin{equation}
(\rho u)\frac{\partial u}{\partial x}    
+
(\rho v)\frac{\partial u}{\partial y}    
=
\frac{\partial  }{\partial y}\left( \mu\frac{\partial  u }{\partial y}\right)
\end{equation}

\begin{equation}
(\rho u)\frac{\partial h}{\partial x}    
+
(\rho v)\frac{\partial h}{\partial y}    
=
\frac{\partial  }{\partial y}
\left( \frac{\mu}{Pr}\frac{\partial  h }{\partial y}\right)
+
\mu
\left(
    \frac{\partial u }{\partial y}
\right)^2 
\end{equation}



Then, $h=h(u)$ and $u=u(x,y)$, a solution of the form 

\begin{equation}
\frac{\partial h }{\partial y}
=
\frac{d h} { du}
\frac{\partial u }{\partial y}
\end{equation}


\begin{equation}
\frac{\partial  }{\partial y}
\left(
    \frac{\partial h }{\partial y}
\right)
=
\frac{\partial  }{\partial y}
\left(
\frac{d h} { du}
\frac{\partial u }{\partial y}
\right)
\end{equation}

\begin{equation}
\frac{\partial  }{\partial y}
\left(
    \frac{\partial h }{\partial y}
\right)
=
\frac{\partial  }{\partial y}
\left(
\frac{d h} { du}
\right)
\frac{\partial u }{\partial y}
+
\frac{d h} { du}
\frac{\partial  }{\partial y}
\left(
\frac{\partial u }{\partial y}
\right)
\end{equation}

\begin{equation}
\frac{\partial  }{\partial y}
\left(
    \frac{\partial h }{\partial y}
\right)
=
\frac{\partial  }{\partial u}
\left(
\frac{d h} { du}
\right)
\frac{\partial u}{\partial y}
\frac{\partial u }{\partial y}
+
\frac{d h} { du}
\left(
\frac{\partial^2 u }{\partial y^2}
\right)
\end{equation}

\begin{equation}
\frac{\partial  }{\partial y}
\left(
    \frac{\partial h }{\partial y}
\right)
=
\frac{d^2 h} { d u^2}
\left(
\frac{\partial u}{\partial y}
\right)^2
+
\frac{d h} { du}
\left(
\frac{\partial^2 u }{\partial y^2}
\right)
\end{equation}

Then, in the momemtum and energy equation, neglecting the pressure grandient
and assuming $Pr=1$,  

\begin{equation}
(\rho u)\frac{\partial u}{\partial x}    
+
(\rho v)\frac{\partial u}{\partial y}    
=
\frac{\partial  }{\partial y}\left( \mu\frac{\partial  u }{\partial y}\right)
\end{equation}


\begin{equation}
(\rho u)\frac{\partial h}{\partial x}    
+
(\rho v)
\frac{d h} { dy}
=
\frac{\partial  }{\partial y}
\left( 
\mu
\frac{d h} { dy}
\right)
+
\mu
\left(
    \frac{\partial u }{\partial y}
\right)^2 
\end{equation}

Opening the energy dissipation term

\begin{equation}
(\rho u)\frac{\partial h}{\partial x}    
+
(\rho v)
\frac{d h} { dy}
=
\mu
\frac{\partial  }{\partial y}
\left( 
\frac{d h} { dy}
\right)
+
\frac{d h} { dy}
\frac{\partial  \mu}{\partial y}
+
\mu
\left(
    \frac{\partial u }{\partial y}
\right)^2 
\end{equation}

And using the relation between $h$ and $u$ in the first term of the right hand side,

\begin{equation}
(\rho u)\frac{\partial h}{\partial x}    
+
(\rho v)
\frac{d h} { dy}
=
\mu
\frac{d h} { du}
\frac{\partial^2 u }{\partial y^2}
+
\mu
\frac{d ^2 h} { d u^2}
\left(
\frac{\partial u }{\partial y}
\right)^2
+
\frac{d h} { dy}
\frac{\partial  \mu}{\partial y}
+
\mu
\left(
    \frac{\partial u }{\partial y}
\right)^2 
\end{equation}

\begin{equation}
(\rho u)\frac{d h}{d u}    
 \frac{\partial u}{\partial x}    
+
(\rho v)
\frac{d h} { du}
 \frac{\partial u}{\partial y}    
 -
\mu
\frac{d h} { du}
\frac{\partial^2 u }{\partial y^2}
-
\frac{d h} { du}
\frac{\partial  u}{\partial y}
\frac{\partial  \mu}{\partial y}
=
\left[
    \frac{d ^2 h} { d u^2}
    +
    1
\right]
    \mu
    \left(
        \frac{\partial u }{\partial y}
    \right)^2 
\end{equation}

\begin{equation}
\frac{d h}{d u}    
\left[
    (\rho u)
     \frac{\partial u}{\partial x}    
    +
    (\rho v)
     \frac{\partial u}{\partial y}    
     -
    \frac{\partial }{\partial y}
    \left(
        \mu
        \frac{\partial u }{\partial y}
    \right)
\right]
=
\left[
    \frac{d ^2 h} { d u^2}
    +
    1
\right]
    \mu
    \left(
        \frac{\partial u }{\partial y}
    \right)^2 
\end{equation}

Using the momentum equation 

\begin{equation}
\left[
    \frac{d ^2 h} { d u^2}
    +
    1
\right]
    \mu
    \left(
        \frac{\partial u }{\partial y}
    \right)^2 
    =
    0
\end{equation}

Then 

\begin{equation}
    \frac{d ^2 h} { d u^2}
    =
    -
    1
\end{equation}

Integrating 

\begin{equation}
    h
    =
    -
    \frac{u^2}{2}
    +
    c_1
    u
    +
    c_2
\end{equation}

For mixing layer these constant 
can be found, using the values 
at the boundaries. For the upper  stream: 

\begin{equation}
    h_1
    =
    -
    \frac{u_1^2}{2}
    +
    c_1
    u_1
    +
    c_2
\label{h1}
\end{equation}

Similarly for the lower stream 

\begin{equation}
    h_2
    =
    -
    \frac{u_2^2}{2}
    +
    c_1
    u_2
    +
    c_2
\label{h2}
\end{equation}

There are two unknowns and two 
equations.  
For the equation \ref{h1}

\begin{equation}
    c_2
    =
    h_1
    +
    \frac{u_1^2}{2}
    -
    c_1
    u_1
\end{equation}

Using it in \ref{h2} 

\begin{equation}
    h_2
    =
    -
    \frac{u_2^2}{2}
    +
    c_1
    u_2
    +
    h_1
    +
    \frac{u_1^2}{2}
    -
    c_1
    u_1
\end{equation}

\begin{equation}
    c_1
    \left(
        u_2
        -
        u_1
    \right)
    =
    h_2
    -
    h_1
    +
    \frac{u_2^2}{2}
    -
    \frac{u_1^2}{2}
\end{equation}

\begin{equation}
    c_1
    =
    \frac{(h_2-h_1)}{(u_2-u_1)}
    +
    \frac{1}{2}
    (u_2+u_1)
\end{equation}

and 

\begin{equation}
    c_2
    =
    h_1
    +
    \frac{u_1^2}{2}
    -
    \left[
        \frac{(h_2-h_1)}{(u_2-u_1)}
        +
        \frac{1}{2}
        (u_2+u_1)
    \right]
    u_1
\end{equation}

Therefore 

\begin{equation}
    h
    =
    -
    \frac{u^2}{2}
    +
    c_1
    u
    +
    c_2
\end{equation}

\begin{equation}
    h
    =
    -
    \frac{u^2}{2}
    +
    \left(
        \frac{(h_2-h_1)}{(u_2-u_1)}
        +
        \frac{1}{2}
        (u_2+u_1)
    \right)
    u
    +
    h_1
    +
    \frac{u_1^2}{2}
    -
    \left[
        \frac{(h_2-h_1)}{(u_2-u_1)}
        +
        \frac{1}{2}
        (u_2+u_1)
    \right]
    u_1
\end{equation}


\begin{equation}
    h
    =
    \frac{u_1^2}{2}
    -
    \frac{u^2}{2}
    +
    \left(
        \frac{(h_2-h_1)}{(u_2-u_1)}
        +
        \frac{1}{2}
        (u_2+u_1)
    \right)
    (u-u_1)
    +
    h_1
\end{equation}

\begin{equation}
    h
    =
    \frac{1}{2}
    (u_1+u)
    (u_1-u)
    +
    \left(
        \frac{(h_2-h_1)}{(u_2-u_1)}
        +
        \frac{1}{2}
        (u_2+u_1)
    \right)
    (u-u_1)
    +
    h_1
\end{equation}

\begin{equation}
    h
    =
    (u_1-u)
    \left[
        \frac{1}{2}
        (u_1+u)
        -
        \left(
            \frac{(h_2-h_1)}{(u_2-u_1)}
            +
            \frac{1}{2}
            (u_2+u_1)
        \right)
    \right]
    +
    h_1
\end{equation}

\begin{equation}
    h
    =
    (u_1-u)
    \left[
        \frac{1}{2}
        (u-u_2)
        -
        \left(
            \frac{(h_2-h_1)}{(u_2-u_1)}
        \right)
    \right]
    +
    h_1
\end{equation}

\begin{equation}
    h
    =
    \frac{1}{2}
    (u-u_2)
    (u_1-u)
    -
    h_2
    \frac{(u_1-u)}{(u_2-u_1)}
    +
    h_1
    \left(
        \frac{(u_1-u)}{(u_2-u_1)}
    \right)
    +
    h_1
\end{equation}

\begin{equation}
    h
    =
    \frac{1}{2}
    (u-u_2)
    (u_1-u)
    -
    h_2
    \frac{(u_1-u)}{(u_2-u_1)}
    +
    h_1
    \left( 
        1
        +
        \left(
            \frac{(u_1-u)}{(u_2-u_1)}
        \right)
    \right)
\end{equation}

\begin{equation}
    h
    =
    \frac{1}{2}
    (u-u_2)
    (u_1-u)
    -
    h_2
    \frac{(u_1-u)}{(u_2-u_1)}
    +
    h_1
       \frac{(u_2-u)}{(u_2-u_1)}
\end{equation}

Assuming $c_p=$ constant, the enthalpy can be related 
with the temperatute with 
the relation $h=c_pT$ 

\begin{equation}
    T
    =
    \frac{1}{2}
    \frac{1}{c_p}
    (u-u_2)
    (u_1-u)
    -
    T_2
    \frac{(u_1-u)}{(u_2-u_1)}
    +
    T_1
    \frac{(u_2-u)}{(u_2-u_1)}
\end{equation}

\begin{equation}
    T
    =
    T_1
    \frac{(u-u_2)}{(u_1-u_2)}
    + 
    T_2
    \frac{(u_1-u)}{(u_1-u_2)}
    +
    \frac{1}{2}
    \frac{1}{c_p}
    (u_1-u)
    (u-u_2)
\end{equation}

A non-dimensional form can be obtained using 
$\overline{T}=T/T_\infty=T/T_1$ and $U=u/U_\infty$, 
$T_2=T_{-\infty}$.  

\begin{equation}
    \overline{T}
    T_\infty
    =
    T_\infty
    \frac{(u-u_2)}{(u_1-u_2)}
    + 
    T_{-\infty}
    \frac{(u_1-u)}{(u_1-u_2)}
    +
    \frac{1}{2}
    \frac{U_\infty^2}{c_p}
    (u_\infty-u)
    (u-u_{-\infty})
\end{equation}

\begin{equation}
    T
    =
    \frac{(u-u_{-\infty})}{(u_{\infty}-u_{-\infty})}
    + 
    \frac{T_{-\infty}}{T_{\infty}}
    \frac{(u_{\infty}-u)}{(u_{\infty}-u_{-\infty})}
    +
    \frac{1}{2}
    \frac{U_\infty^2}{T_{\infty}c_p}
    (u_{\infty}-u)
    (u-u_{-\infty})
\end{equation}

\begin{equation}
    T
    =
    \frac{(u-u_{-\infty})}{(u_{\infty}-u_{-\infty})}
    + 
    \frac{T_{-\infty}}{T_{\infty}}
    \frac{(u_{\infty}-u)}{(u_{\infty}-u_{-\infty})}
    +
    \frac{1}{2}
    \frac{\gamma R}{c_p}
    \frac{U_\infty^2}{\gamma R T_{\infty}}
    (u_{\infty}-u)
    (u-u_{-\infty})
\end{equation}

\begin{equation}
    \frac{R}{c_p}
    =
    \frac{c_p-c_v}{c_p}
    =
    1-\frac{1}{\gamma}
    =
    \frac{\gamma-1}{\gamma}
\end{equation}

$\beta_t=T_{-\infty}/T_{\infty}$

\begin{equation}
    T
    =
    \frac{(u-u_{-\infty})}{(u_{\infty}-u_{-\infty})}
    + 
    \beta_t
    \frac{(u_{\infty}-u)}{(u_{\infty}-u_{-\infty})}
    +
    \frac{1}{2}
    M^2(\gamma-1) (u_{\infty}-u)
    (u-u_{-\infty})
\end{equation}

Using $T=T/T_\infty$ and $U=U/a_\infty$.  

\begin{equation}
    T
    =
    \frac{(u-u_2)}{(u_1-u_2)}
    + 
    \beta_t
    \frac{(u_1-u)}{(u_1-u_2)}
    +
    \frac{\gamma-1}{2}
    (u_1-u)
    (u-u_2)
\end{equation}
    

%using T1 and T2
%using T1 and T2
%using T1 and T2
%
%The last term and the temperature are   dimensional  and depend 
%on the non dimensional 
%parameters.
%
%Using $T=T/T_1$ and $U=U/U_1$.  
%
%\begin{equation}
%    T
%    T_1
%    =
%    T_1
%    \frac{(u-u_2)}{(u_1-u_2)}
%    + 
%    T_2
%    \frac{(u_1-u)}{(u_1-u_2)}
%    +
%    \frac{1}{2}
%    \frac{U_1^2}{c_p}
%    (u_1-u)
%    (u-u_2)
%\end{equation}
%
%\begin{equation}
%    T
%    =
%    \frac{(u-u_2)}{(u_1-u_2)}
%    + 
%    \frac{T_2}{T_1}
%    \frac{(u_1-u)}{(u_1-u_2)}
%    +
%    \frac{1}{2}
%    \frac{U_1^2}{T_1c_p}
%    (u_1-u)
%    (u-u_2)
%\end{equation}
%
%\begin{equation}
%    T
%    =
%    \frac{(u-u_2)}{(u_1-u_2)}
%    + 
%    \frac{T_2}{T_1}
%    \frac{(u_1-u)}{(u_1-u_2)}
%    +
%    \frac{1}{2}
%    \frac{U_1^2\gamma_1 R}{\gamma_1RT_1c_p} (u_1-u)
%    (u-u_2)
%\end{equation}
%
%\begin{equation}
%    T
%    =
%    \frac{(u-u_2)}{(u_1-u_2)}
%    + 
%    \frac{T_2}{T_1}
%    \frac{(u_1-u)}{(u_1-u_2)}
%    +
%    \frac{1}{2}
%    \frac{M^2\gamma_1 R}{c_p} (u_1-u)
%    (u-u_2)
%\end{equation}
%
%\begin{equation}
%    \frac{R}{c_p}
%    =
%    \frac{c_p-c_v}{c_p}
%    =
%    1-\frac{1}{\gamma}
%    =
%    \frac{\gamma-1}{\gamma}
%\end{equation}
%
%$\beta_t=T_2/T_1$
%
%\begin{equation}
%    T
%    =
%    \frac{(u-u_2)}{(u_1-u_2)}
%    + 
%    \beta_t
%    \frac{(u_1-u)}{(u_1-u_2)}
%    +
%    \frac{1}{2}
%    M^2(\gamma_1-1) (u_1-u)
%    (u-u_2)
%\end{equation}
%
%Using $T=T/T_1$ and $U=U/a_1$.  
%
%\begin{equation}
%    T
%    =
%    \frac{(u-u_2)}{(u_1-u_2)}
%    + 
%    \beta_t
%    \frac{(u_1-u)}{(u_1-u_2)}
%    +
%    \frac{\gamma-1}{2}
%    (u_1-u)
%    (u-u_2)
%\end{equation}
%
%