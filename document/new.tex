%\section{\textbf{Self-Similarity solution for Reacting Mixing Layer equations}}

To find a similarity solutions for a reacting mixing layer equations, firstly we 
need to use the Howarth-Dorodnitsyn \ref{h-d} transformation to find am 
incompressible form of the compressibles equations. 
\\
\\
Definig the stream function for compressible flow as: 
\\
\\
\begin{equation}
\frac{\partial \Psi}{\partial x}=-\varrho V \qquad
\frac{\partial \Psi}{\partial y}=\varrho U,
\end{equation} 
\\
\\
The mass conservation equation is identically satisfied by the introduction 
of  this stream function. Now defining a vertical stretching as:
\\
\\
\begin{equation}
x=\overline{x}\qquad
y=\int_0^{\overline{y}}\varrho d\overline{y}%\qquad 
%\hat{V}=\varrho V+U\int_o^{\overline{y}}\varrho_{\overline{x}}\qquad
%\label{h-d}
\end{equation} 
\\
\\
and using the chain rule:
\\
\\
\begin{equation}
f(\overline{x},\overline{y})\qquad
\overline{x}=\overline{x}(x,y)\qquad
\overline{y}=\overline{y}(x,y)
\end{equation} 
\\
\\
\begin{equation}
\frac{\partial f}{\partial \overline{x}}=
\frac{\partial f}{\partial x}
\frac{\partial x}{\partial \overline{x}}+
\frac{\partial f}{\partial y}
\frac{\partial y}{\partial \overline{x}}
\end{equation} 
\\
\\
\begin{equation}
\frac{\partial f}{\partial \overline{y}}=
\frac{\partial f}{\partial x}
\frac{\partial x}{\partial \overline{y}}+
\frac{\partial f}{\partial y}
\frac{\partial y}{\partial \overline{y}}
\end{equation} 
\\
\\
\begin{equation}
\frac{\partial f}{\partial \overline{x}}=
\frac{\partial f}{\partial x}+
%\frac{\partial x}{\partial \overline{x}}+
\frac{\partial f}{\partial y}
\int_0^{\overline{y}}\frac{\partial \varrho }{\partial \overline{x}}dy
%\label{fx}
\end{equation} 
\\
\\
\begin{equation}
\frac{\partial f}{\partial \overline{y}}=
%\frac{\partial f}{\partial x}
%\frac{\partial x}{\partial \overline{y}}+
\frac{\partial f}{\partial y}
%\frac{\partial y}{\partial \overline{y}}
\varrho
%\label{fy}
\end{equation} 
\\
\\
so that:
\\
\\
\begin{equation}
U=\frac{\partial \Psi}{\partial y}=\hat{U},
\end{equation} 
\\
\\
that have the same form as incompresible flow, 
in the same way we can defined:
\\
\\
\begin{equation}
\hat{V}=-\frac{\partial \Psi}{\partial x},
\end{equation} 
\\
\\
Using the definition of $V$:
\\
\\
\begin{equation}
\varrho V=-
\frac{\partial \Psi}{\partial \overline{x}}
\end{equation} 
\\
\\
with  \ref{fx} we can write: 
\\
\\
\begin{equation}
  \varrho V=-
  \left(
       \frac{\partial \Psi}{\partial x}+
       \frac{\partial \Psi}{\partial y}
       \frac{\partial y}{\partial \overline{x}}
  \right)
\end{equation} 
\\
\\
\begin{equation}
  \varrho V=
  \left(
       \hat{V}
       -U
       \frac{\partial y}{\partial \overline{x}}
  \right)
\end{equation} 
\\
\\
\begin{equation}
  \hat{V}=
  \varrho V
  +U
  \frac{\partial y}{\partial \overline{x}}
\label {hatv}
\end{equation} 
\\
\\
Now we can used the equation \ref{h-d}, \ref{fx}, \ref{fy} and \ref{hatv} in the 
Mixing layer equations, \ref{mass2} to  
\ref{g2}, obtain:
\\
\\
\textbf{Mass conservation}
\\
\\
\begin{equation} 
\frac{\partial}{\partial \overline{x}}(\varrho U)+
\frac{\partial}{\partial \overline{y}}(\varrho V)=0
\end{equation} 
\\
\\
\begin{equation} 
\frac{\partial}{\partial x}(\varrho U)+
\frac{\partial}{\partial y}(\varrho U)
\frac{\partial y}{\partial \overline{x}}+
\varrho\frac{\partial}{\partial y}(\varrho V)=0
\end{equation} 
\\
\\
\begin{equation} 
\frac{\partial}{\partial x}(\varrho U)+
\frac{\partial}{\partial y}(\varrho U)
\frac{\partial y}{\partial \overline{x}}+
\varrho\frac{\partial}{\partial y}
  \left(
       \hat{V}
       -U
       \frac{\partial y}{\partial \overline{x}}
  \right)
=0
\end{equation} 
\\
\\
\begin{equation} 
U\frac{\partial \varrho }{\partial x}+
\varrho\frac{\partial U}{\partial x}+
U\frac{\partial \varrho }{\partial y}
\frac{\partial y}{\partial \overline{x}}+
\varrho\frac{\partial U}{\partial y}
\frac{\partial y}{\partial \overline{x}}+
\varrho\frac{\partial \hat{V}}{\partial y}
-\varrho\frac{\partial }{\partial \overline{y}}
       \left(
       U\frac{\partial y}{\partial \overline{x}}
       \right)
=0
\end{equation} 
\\
\\
\begin{equation} 
U\frac{\partial \varrho }{\partial x}+
\varrho\frac{\partial U}{\partial x}+
U\frac{\partial \varrho }{\partial y}
\frac{\partial y}{\partial \overline{x}}+
\varrho\frac{\partial U}{\partial y}
\frac{\partial y}{\partial \overline{x}}+
\varrho\frac{\partial \hat{V}}{\partial \overline{y}}
-\varrho\frac{\partial U}{\partial \overline{y}}
       \frac{\partial y}{\partial \overline{x}}
-\varrho U\frac{\partial }{\partial y}
       \left(
       \frac{\partial y}{\partial \overline{x}}
       \right)
=0
\end{equation} 
\\
\\
\begin{equation} 
U\frac{\partial \varrho }{\partial x}+
\varrho\frac{\partial U}{\partial x}+
U\frac{\partial \varrho }{\partial y}
\frac{\partial y}{\partial \overline{x}}+
\cancel{
\varrho\frac{\partial U}{\partial y}
\frac{\partial y}{\partial \overline{x}}}^0+
\varrho\frac{\partial \hat{V}}{\partial y}
-\cancel{
         \varrho\frac{\partial U}{\partial \overline{y}}
         \frac{\partial y}{\partial \overline{x}}
        }
-\varrho U\frac{\partial }{\partial y}
       \left(
       \frac{\partial y}{\partial \overline{x}}
       \right)
=0
\end{equation} 
\\
\\
\begin{equation} 
U\frac{\partial \varrho }{\partial x}+
\varrho\frac{\partial U}{\partial x}+
U\frac{\partial \varrho }{\partial y}
\frac{\partial y}{\partial \overline{x}}+
\varrho\frac{\partial \hat{V}}{\partial y}
-\varrho U\frac{\partial }{\partial y}
       \left(
       \frac{\partial y}{\partial \overline{x}}
       \right)
=0
\end{equation} 
\\
\\
the last term of the above equation  can be write as:
\\
\\
\begin{equation} 
\varrho U\frac{\partial }{\partial y}
       \left(
       \frac{\partial y}{\partial \overline{x}}
       \right)
=
U\varrho \frac{\partial }{\partial y}
\left(
      \frac{\partial y}{\partial \overline{x}}
\right)
=
U\frac{\partial }{\partial \overline{y}}
\left(
      \frac{\partial y}{\partial \overline{x}}
\right)
=
U\frac{\partial \varrho}{\partial \overline{x}}
=
U
\left(
\frac{\partial \varrho}{\partial x}+
\frac{\partial \varrho}{\partial y}
\frac{\partial y}{\partial \overline{x}}
\right)
\end{equation} 
\\
\\
hence:
\\
\\
\begin{equation} 
\varrho\frac{\partial U}{\partial x}+
\varrho\frac{\partial \hat{V}}{\partial y}
=0
\end{equation} 
\\
\\
\begin{equation} 
\frac{\partial \hat{U}}{\partial x}+
\frac{\partial \hat{V}}{\partial y}
=0
\end{equation} 
\\
\\
\textbf{Species Conservation}
\\
\\
Oxidizer conservation
\\
\\
\begin{equation} 
\varrho U\frac{\partial \psi_{o}}{\partial\overline{x}}+
\varrho V\frac{\partial \psi_{o}}{\partial\overline{y}}=
\frac{1}{Le_o}\frac{\partial}{\partial \overline{y}}\left(\rho D^o\frac{\partial \psi_o}{\partial \overline{y}}\right)+
s^{o}\Omega
%\frac{\partial\rho D^m \tfrac{\partial\rho Y^m u_i}{\partial x_i}u_i}{\partial x_i}=
\end{equation} 
\\
\\
\begin{equation} 
\varrho U\frac{\partial \psi_{o}}{\partial x}+
\varrho U\frac{\partial \psi_{o}}{\partial y}
\frac{\partial y}{\partial \overline{x}}+
\varrho^2 V\frac{\partial \psi_{o}}{\partial y}=
\frac{\varrho}{Le_o}\frac{\partial}{\partial y}\left(\varrho^2 D^o\frac{\partial \psi_o}{\partial y}\right)+
s^{o}\Omega
%\frac{\partial\rho D^m \tfrac{\partial\rho Y^m u_i}{\partial x_i}u_i}{\partial x_i}=
\end{equation} 
\\
\\
\begin{equation} 
\varrho U\frac{\partial \psi_{o}}{\partial x}+
\varrho U\frac{\partial \psi_{o}}{\partial y}
\frac{\partial y}{\partial \overline{x}}+
\varrho^2 \frac{\partial \psi_{o}}{\partial y}
\left(
       \frac{1}{\varrho}\hat{V}-
       \frac{1}{\varrho}U\frac{\partial y}{\partial \overline{x}}
\right)
=
\frac{\varrho}{Le_o}\frac{\partial}{\partial y}\left(\varrho^2 D^o\frac{\partial \psi_o}{\partial y}\right)+
s^{o}\Omega
%\frac{\partial\rho D^m \tfrac{\partial\rho Y^m u_i}{\partial x_i}u_i}{\partial x_i}=
\end{equation} 
\\
\\
\begin{equation} 
\varrho U\frac{\partial \psi_{o}}{\partial x}+
\varrho U\frac{\partial \psi_{o}}{\partial y}
\frac{\partial y}{\partial \overline{x}}+
\varrho^2 \frac{\partial \psi_{o}}{\partial y}
\left(
       \frac{1}{\varrho}\hat{V}-
       \frac{1}{\varrho}U\frac{\partial y}{\partial \overline{x}}
\right)
=
\frac{\varrho}{Le_o}\frac{\partial}{\partial y}\left(\varrho^2 D^o\frac{\partial \psi_o}{\partial y}\right)+
s^{o}\Omega
%\frac{\partial\rho D^m \tfrac{\partial\rho Y^m u_i}{\partial x_i}u_i}{\partial x_i}=
\end{equation} 
\\
\\
\begin{equation} 
\varrho U\frac{\partial \psi_{o}}{\partial x}+
\varrho \hat{V}\frac{\partial \psi_{o}}{\partial y}
=
\frac{\varrho}{Le_o}\frac{\partial}{\partial y}\left(\varrho^2 D^o\frac{\partial \psi_o}{\partial y}\right)+
s^{o}\Omega
%\frac{\partial\rho D^m \tfrac{\partial\rho Y^m u_i}{\partial x_i}u_i}{\partial x_i}=
\end{equation} 
\\
\\
\begin{equation} 
 \hat{U}\frac{\partial \psi_{o}}{\partial x}+
 \hat{V}\frac{\partial \psi_{o}}{\partial y}
=
\frac{1}{Le_o}\frac{\partial}{\partial y}\left(\varrho^2 D^o\frac{\partial \psi_o}{\partial y}\right)+
\frac{s^{o}\Omega}{\varrho}
%\frac{\partial\rho D^m \tfrac{\partial\rho Y^m u_i}{\partial x_i}u_i}{\partial x_i}=
\end{equation} 
\\
\\
Fuel conservation
\\
\\
In the same way as in oxidizer conservartion, using the transform coordinates $x, y, \hat{U}$ and $\hat{V}$ in the Fuel
conservations species, we get: 
\\
\\
\begin{equation} 
 \hat{U}\frac{\partial \psi_{F}}{\partial x}+
 \hat{V}\frac{\partial \psi_{F}}{\partial y}
=
\frac{1}{Le_F}\frac{\partial}{\partial y}\left(\varrho^2 D^F\frac{\partial \psi_F}{\partial y}\right)+
\frac{s^{F}\Omega}{\varrho}
%\frac{\partial\rho D^m \tfrac{\partial\rho Y^m u_i}{\partial x_i}u_i}{\partial x_i}=
\end{equation} 
\\
\\
\textbf{Momemtum conservation}
\\
\\
\begin{equation} 
\varrho U\frac{\partial U}{\partial \overline{x}}+
\varrho V\frac{\partial U}{\partial \overline{y}}=
\frac{\partial}{\partial \overline{y}}\left(\mu\frac{\partial U}{\partial  \overline{y}}\right)
\end{equation} 
\\
\\
\begin{equation} 
\varrho U\frac{\partial U}{\partial x}+
\varrho U\frac{\partial U}{\partial y}
\frac{\partial y}{\partial \overline{x}}+
\varrho^2 V\frac{\partial U}{\partial y}=
\varrho\frac{\partial}{\partial y}\left(\mu\varrho\frac{\partial U}{\partial  y}\right)
\end{equation} 
\\
\\
\begin{equation} 
\varrho U\frac{\partial U}{\partial x}+
\varrho U\frac{\partial U}{\partial y}
\frac{\partial y}{\partial \overline{x}}+
\varrho^2 
\left(
       \frac{1}{\varrho}\hat{V}-
       \frac{1}{\varrho}U\frac{\partial y}{\partial \overline{x}}
\right)
\frac{\partial U}{\partial y}=
\varrho\frac{\partial}{\partial y}\left(\mu\varrho\frac{\partial U}{\partial  y}\right)
\end{equation} 
\\
\\
\begin{equation} 
\varrho U\frac{\partial U}{\partial x}+
\varrho \hat{V}\frac{\partial U}{\partial y}=
\varrho\frac{\partial}{\partial y}\left(\mu\varrho\frac{\partial U}{\partial  y}\right)
\end{equation} 
\\
\\
\begin{equation} 
 \hat{U}\frac{\partial \hat{U}}{\partial x}+
 \hat{V}\frac{\partial \hat{U}}{\partial y}=
\frac{\partial}{\partial y}\left(\mu\varrho\frac{\partial \hat{U}}{\partial  y}\right)
\end{equation} 
\\
\\
\textbf{Energy Equation}
\\
\\
\begin{equation}
\varrho U\frac{\partial }{\partial \overline{x}}(c_p \overline{T})+
\varrho V\frac{\partial }{\partial \overline{y}}(c_p \overline{T})=
(\gamma-1)M^2\mu\frac{\partial^2 U}{\partial \overline{y}^2}+
\frac{1}{Pr}\frac{\partial}{\partial \overline{y}}\left(k\frac{\partial \overline{T}}{\partial \overline{y}}\right)+
Q\Omega
\end{equation} 
\\
\\
Using the chain rule to change the second order derivative of $\overline{y}$ in the energy equation. 
\\
\\
\begin{equation}
\frac{\partial^2f}{\partial \overline{y}^2}=
\frac{\partial}{\partial y}
\left(
\frac{\partial f}{\partial y}
\frac{\partial y}{\partial \overline{y}}
\right)
\frac{\partial y}{\partial \overline{y}}
\end{equation} 
\\
\\
\begin{equation}
\frac{\partial^2f}{\partial \overline{y}^2}=
\frac{\partial ^2 f}{\partial y^2}
\left(
\frac{\partial y}{\partial \overline{y}}
\right)^2+
\frac{\partial f}{\partial y}
\frac{\partial ^2 y}{\partial \overline{y}^2}
\end{equation} 
\\
\\
\begin{equation}
\frac{\partial^2f}{\partial \overline{y}^2}=
\varrho^2\frac{\partial ^2 f}{\partial y^2}+
\varrho
\frac{\partial \varrho}{\partial y}
\frac{\partial f}{\partial y}
\end{equation} 
\\
\\
\begin{equation}
\begin{split}
\varrho U\frac{\partial }{\partial x}(c_p \overline{T})+
\varrho U\frac{\partial }{\partial y}(c_p \overline{T})
\frac{\partial y}{\partial \overline{x}}+
\varrho^2 V
\frac{\partial }{\partial y}(c_p \overline{T})=
\\
\varrho^2(\gamma-1)M^2\mu\frac{\partial^2 U}{\partial y^2}+
\varrho
\frac{\partial \varrho}{\partial y}
(\gamma-1)M^2\mu\frac{\partial U}{\partial y}+
\frac{\varrho}{Pr}\frac{\partial}{\partial y}\left(k\varrho\frac{\partial \overline{T}}{\partial y}\right)+
Q\Omega
\end{split}
\end{equation} 
\\
\\
\begin{equation}
\begin{split}
\varrho U\frac{\partial }{\partial x}(c_p \overline{T})+
\varrho U\frac{\partial }{\partial y}(c_p \overline{T})
\frac{\partial y}{\partial \overline{x}}+
\varrho^2 
\left(
       \frac{1}{\varrho}\hat{V}-
       \frac{1}{\varrho}U\frac{\partial y}{\partial \overline{x}}
\right)
\frac{\partial }{\partial y}(c_p \overline{T})=
\\
\varrho^2(\gamma-1)M^2\mu\frac{\partial^2 U}{\partial y^2}+
\varrho
\frac{\partial \varrho}{\partial y}
(\gamma-1)M^2\mu\frac{\partial U}{\partial y}+
\frac{\varrho}{Pr}\frac{\partial}{\partial y}\left(k\varrho\frac{\partial \overline{T}}{\partial y}\right)+
Q\Omega
\end{split}
\end{equation} 
\\
\\
\begin{equation}
\begin{split}
\hat{U}\frac{\partial }{\partial x}(c_p \overline{T})+
\hat{V}
\frac{\partial }{\partial y}(c_p \overline{T})=
\varrho(\gamma-1)M^2\mu\frac{\partial^2 \hat{U}}{\partial y^2}+
\\
\frac{\partial \varrho}{\partial y}
(\gamma-1)M^2\mu\frac{\partial \hat{U}}{\partial y}+
\frac{1}{Pr}\frac{\partial}{\partial y}\left(k\varrho\frac{\partial \overline{T}}{\partial y}\right)+
\frac{Q\Omega}{\varrho}
\end{split}
\end{equation} 
\\
\\
\begin{equation}
\hat{U}\frac{\partial }{\partial x}(c_p \overline{T})+
\hat{V}
\frac{\partial }{\partial y}(c_p \overline{T})=
(\gamma-1)M^2\mu
\frac{\partial}{\partial y}
\left(
\varrho\frac{\partial \hat{U}}{\partial y}
\right)+
\frac{1}{Pr}\frac{\partial}{\partial y}\left(k\varrho\frac{\partial \overline{T}}{\partial y}\right)+
\frac{Q\Omega}{\varrho}
\end{equation} 
\\
\\
To transform completely the compressible equations to incompressible form we need to used the
Chapman's approximate viscosity law:
\\
\\
\begin{equation} 
\mu=C_w T
\end{equation} 
\\
\\
Multiply by $\varrho$ and using the gas perfect equation, we get:
\\
\\
\begin{equation} 
\mu\varrho=C_w
\end{equation} 
\\
\\
where $C_w$  is a function of the temperature $C_w=C_w(T)$ . 
\\
\\
\begin{equation} 
\frac{\partial \hat{U}}{\partial x}+
\frac{\partial \hat{V}}{\partial y}
=0
\end{equation} 
\\
\\
\begin{equation} 
 \hat{U}\frac{\partial \psi_{o}}{\partial x}+
 \hat{V}\frac{\partial \psi_{o}}{\partial y}
=
\frac{1}{Le_o}\frac{\partial}{\partial y}\left(\varrho^2 D^o\frac{\partial \psi_o}{\partial y}\right)+
\frac{s^{o}\Omega}{\varrho}
%\frac{\partial\rho D^m \tfrac{\partial\rho Y^m u_i}{\partial x_i}u_i}{\partial x_i}=
\end{equation} 
\\
\\
\begin{equation} 
 \hat{U}\frac{\partial \psi_{F}}{\partial x}+
 \hat{V}\frac{\partial \psi_{F}}{\partial y}
=
\frac{1}{Le_F}\frac{\partial}{\partial y}\left(\varrho^2 D^F\frac{\partial \psi_F}{\partial y}\right)+
\frac{s^{F}\Omega}{\varrho}
%\frac{\partial\rho D^m \tfrac{\partial\rho Y^m u_i}{\partial x_i}u_i}{\partial x_i}=
\end{equation} 
\\
\\
\begin{equation} 
 \hat{U}\frac{\partial \hat{U}}{\partial x}+
 \hat{V}\frac{\partial \hat{U}}{\partial y}=
\frac{\partial}{\partial y}\left(C_w(T)\frac{\partial \hat{U}}{\partial  y}\right)
\end{equation} 
\\
\\
\begin{equation}
\hat{U}\frac{\partial }{\partial x}(c_p \overline{T})+
\hat{V}
\frac{\partial }{\partial y}(c_p \overline{T})=
(\gamma-1)M^2\mu
\frac{\partial}{\partial y}
\left(
\varrho\frac{\partial \hat{U}}{\partial y}
\right)+
\frac{1}{Pr}\frac{\partial}{\partial y}\left(k\varrho\frac{\partial \overline{T}}{\partial y}\right)+
\frac{Q\Omega}{\varrho}
\end{equation} 

\section{Crocco Busemann}

In this section a special solution to boundary layer equation is found using the definition of the 
total enthalpy.

From the boundary layer equation in two dimensional flow 

\begin{equation}
\frac{\partial }{\partial x}(\rho u)    
+
\frac{\partial }{\partial y}(\rho v)    
=
0
\end{equation}

\begin{equation}
(\rho u)\frac{\partial u}{\partial x}    
+
(\rho v)\frac{\partial u}{\partial y}    
=
-
\frac{\partial  P}{\partial x}
+
\frac{\partial  }{\partial y}\left( \mu\frac{\partial  u }{\partial y}\right)
\end{equation}

\begin{equation}
(\rho u)\frac{\partial h}{\partial x}    
+
(\rho v)\frac{\partial h}{\partial y}    
=
u
\frac{\partial  P}{\partial x}
+
\frac{\partial  }{\partial y}
\left( \frac{\mu}{Pr}\frac{\partial  h }{\partial y}\right)
+
\mu
\left(
    \frac{\partial u }{\partial y}
\right)^2 
\end{equation}

where the Prantdl number is defined as $Pr=\mu c_p/k$.


Since the total enthalpy is defined as 

\begin{equation}
H=h+\frac{1}{2}u^2
\end{equation}

the total enthalpy equation using both 
the energy and momentum equation is  

\begin{equation}
(\rho u)\frac{\partial H}{\partial x}    
+
(\rho v)\frac{\partial H}{\partial y}    
=
\frac{\partial  }{\partial y}
\left( \frac{\mu}{Pr}\frac{\partial  H }{\partial y}\right)
+
\frac{\partial  }{\partial y}
\left[
    \left(
    1-\frac{1}{Pr}
    \right)
 \mu u\frac{\partial  u }{\partial y}
 \right]
\end{equation}

This equation allows a simplest solution  
$H=h+\frac{1}{2}u^2=$ constant as long as $Pr=1$, which leads to  

\begin{equation}
\frac{\partial  H }{\partial y}
=
\frac{\partial  h_w }{\partial y}
=
0
\end{equation}

Representing the not heat transfer at the wall.

$Pr=1$ implies in a perfect balance between viscous 
dissipation and heat conduction so as keep the 
the stagnation enthalpy constant in adiabatic boundary layer and also is a good approximation for gases.

Another solution can be obtained if 
the pressure gradient is neglected
in the boundary layer equation.  
With this, the momentum equation and the energy equation get very similar,
it seems as if u and h could be 
interchanged except for the dissipation term. 


\begin{equation}
(\rho u)\frac{\partial u}{\partial x}    
+
(\rho v)\frac{\partial u}{\partial y}    
=
\frac{\partial  }{\partial y}\left( \mu\frac{\partial  u }{\partial y}\right)
\end{equation}

\begin{equation}
(\rho u)\frac{\partial h}{\partial x}    
+
(\rho v)\frac{\partial h}{\partial y}    
=
\frac{\partial  }{\partial y}
\left( \frac{\mu}{Pr}\frac{\partial  h }{\partial y}\right)
+
\mu
\left(
    \frac{\partial u }{\partial y}
\right)^2 
\end{equation}

Then a solution of the form 

\begin{equation}
\frac{\partial h }{\partial y}
=
\frac{d h} { du}
\frac{\partial u }{\partial y}
\end{equation}


\begin{equation}
\frac{\partial  }{\partial y}
\left(
    \frac{\partial h }{\partial y}
\right)
=
\frac{\partial  }{\partial y}
\left(
\frac{d h} { du}
\frac{\partial u }{\partial y}
\right)
=
\frac{\partial  }{\partial u}
\left(
\frac{d h} { du}
\frac{\partial u }{\partial y}
\right)
\frac{\partial u }{\partial y}
\end{equation}

\begin{equation}
\frac{\partial  }{\partial y}
\left(
    \frac{\partial h }{\partial y}
\right)
=
\frac{\partial  }{\partial u}
\left(
\frac{d h} { du}
\frac{\partial u }{\partial y}
\right)
\frac{\partial u }{\partial y}
=
\frac{d h} { du}
\frac{\partial u^2 }{\partial y^2}
+
\frac{d ^2 h} { d u^2}
\left(
\frac{\partial u }{\partial y}
\right)^2
\end{equation}
 
\begin{equation}
(\rho u)\frac{\partial u}{\partial x}    
+
(\rho v)\frac{\partial u}{\partial y}    
=
\frac{\partial  }{\partial y}\left( \mu\frac{\partial  u }{\partial y}\right)
\end{equation}

\begin{equation}
(\rho u)\frac{\partial h}{\partial x}    
+
(\rho v)
\frac{d h} { du}
\frac{\partial u }{\partial y}
=
\frac{\partial  }{\partial y}
\left( \frac{\mu}{Pr}
\frac{d h} { du}
\frac{\partial u }{\partial y}
\right)
+
\mu
\left(
    \frac{\partial u }{\partial y}
\right)^2 
\end{equation}

Assuming $Pr=1$

\begin{equation}
(\rho u)\frac{\partial h}{\partial x}    
+
(\rho v)
\frac{d h} { dy}
=
\frac{\partial  }{\partial y}
\left( 
\mu
\frac{d h} { dy}
\right)
+
\mu
\left(
    \frac{\partial u }{\partial y}
\right)^2 
\end{equation}

\begin{equation}
(\rho u)\frac{\partial h}{\partial x}    
+
(\rho v)
\frac{d h} { dy}
=
\mu
\frac{\partial  }{\partial y}
\left( 
\frac{d h} { dy}
\right)
+
\frac{d h} { dy}
\frac{\partial  \mu}{\partial y}
+
\mu
\left(
    \frac{\partial u }{\partial y}
\right)^2 
\end{equation}

\begin{equation}
(\rho u)\frac{\partial h}{\partial x}    
+
(\rho v)
\frac{d h} { dy}
=
\mu
\frac{d h} { du}
\frac{\partial u^2 }{\partial y^2}
+
\mu
\frac{d ^2 h} { d u^2}
\left(
\frac{\partial u }{\partial y}
\right)^2
+
\frac{d h} { dy}
\frac{\partial  \mu}{\partial y}
+
\mu
\left(
    \frac{\partial u }{\partial y}
\right)^2 
\end{equation}

\begin{equation}
(\rho u)\frac{d h}{d u}    
 \frac{\partial u}{\partial x}    
+
(\rho v)
\frac{d h} { du}
 \frac{\partial u}{\partial y}    
 -
\mu
\frac{d h} { du}
\frac{\partial u^2 }{\partial y^2}
-
\frac{d h} { du}
\frac{\partial  u}{\partial y}
\frac{\partial  \mu}{\partial y}
=
\left[
    \frac{d ^2 h} { d u^2}
    +
    1
\right]
    \mu
    \left(
        \frac{\partial u }{\partial y}
    \right)^2 
\end{equation}

\begin{equation}
\frac{d h}{d u}    
\left[
    (\rho u)
     \frac{\partial u}{\partial x}    
    +
    (\rho v)
     \frac{\partial u}{\partial y}    
     -
    \frac{\partial }{\partial y}
    \left(
        \mu
        \frac{\partial u }{\partial y}
    \right)
\right]
=
\left[
    \frac{d ^2 h} { d u^2}
    +
    1
\right]
    \mu
    \left(
        \frac{\partial u }{\partial y}
    \right)^2 
\end{equation}

Using the momentum equation 

\begin{equation}
\left[
    \frac{d ^2 h} { d u^2}
    +
    1
\right]
    \mu
    \left(
        \frac{\partial u }{\partial y}
    \right)^2 
    =
    0
\end{equation}

Then 

\begin{equation}
    \frac{d ^2 h} { d u^2}
    =
    -
    1
\end{equation}

Integrating 

\begin{equation}
    h
    =
    -
    \frac{u^2}{2}
    +
    c_1
    u
    +
    c_2
\end{equation}

For mixing layer these constant 
can be found, using the values 
at the boundaries. For the upper  stream: 

\begin{equation}
    h_1
    =
    -
    \frac{u_1^2}{2}
    +
    c_1
    u_1
    +
    c_2
\label{h1}
\end{equation}

Similarly for the lower stream 

\begin{equation}
    h_2
    =
    -
    \frac{u_2^2}{2}
    +
    c_1
    u_2
    +
    c_2
\label{h2}
\end{equation}

This is two unknown and two 
equations.  

For the equation \ref{h1}

\begin{equation}
    c_2
    =
    h_1
    +
    \frac{u_1^2}{2}
    -
    c_1
    u_1
\end{equation}

Using it in \ref{h2} 

\begin{equation}
    h_2
    =
    -
    \frac{u_2^2}{2}
    +
    c_1
    u_2
    +
    h_1
    +
    \frac{u_1^2}{2}
    -
    c_1
    u_1
\end{equation}

\begin{equation}
    c_1
    \left(
        u_2
        -
        u_1
    \right)
    =
    h_2
    -
    h_1
    +
    \frac{u_2^2}{2}
    -
    \frac{u_1^2}{2}
\end{equation}

\begin{equation}
    c_1
    =
    \frac{(h_2-h_1)}{(u_2-u_1)}
    +
    \frac{1}{2}
    (u_2+u_1)
\end{equation}

and 

\begin{equation}
    c_2
    =
    h_1
    +
    \frac{u_1^2}{2}
    -
    \left[
        \frac{(h_2-h_1)}{(u_2-u_1)}
        +
        \frac{1}{2}
        (u_2+u_1)
    \right]
    u_1
\end{equation}

Therefore 

\begin{equation}
    h
    =
    -
    \frac{u^2}{2}
    +
    c_1
    u
    +
    c_2
\end{equation}

\begin{equation}
    h
    =
    -
    \frac{u^2}{2}
    +
    \left(
        \frac{(h_2-h_1)}{(u_2-u_1)}
        +
        \frac{1}{2}
        (u_2+u_1)
    \right)
    u
    +
    h_1
    +
    \frac{u_1^2}{2}
    -
    \left[
        \frac{(h_2-h_1)}{(u_2-u_1)}
        +
        \frac{1}{2}
        (u_2+u_1)
    \right]
    u_1
\end{equation}


\begin{equation}
    h
    =
    \frac{u_1^2}{2}
    -
    \frac{u^2}{2}
    +
    \left(
        \frac{(h_2-h_1)}{(u_2-u_1)}
        +
        \frac{1}{2}
        (u_2+u_1)
    \right)
    (u-u_1)
    +
    h_1
\end{equation}

\begin{equation}
    h
    =
    \frac{1}{2}
    (u_1+u)
    (u_1-u)
    +
    \left(
        \frac{(h_2-h_1)}{(u_2-u_1)}
        +
        \frac{1}{2}
        (u_2+u_1)
    \right)
    (u-u_1)
    +
    h_1
\end{equation}

\begin{equation}
    h
    =
    (u_1-u)
    \left[
        \frac{1}{2}
        (u_1+u)
        -
        \left(
            \frac{(h_2-h_1)}{(u_2-u_1)}
            +
            \frac{1}{2}
            (u_2+u_1)
        \right)
    \right]
    +
    h_1
\end{equation}

\begin{equation}
    h
    =
    (u_1-u)
    \left[
        \frac{1}{2}
        (u-u_2)
        -
        \left(
            \frac{(h_2-h_1)}{(u_2-u_1)}
        \right)
    \right]
    +
    h_1
\end{equation}

\begin{equation}
    h
    =
    \frac{1}{2}
    (u-u_2)
    (u_1-u)
    -
    h_2
    \frac{(u_1-u)}{(u_2-u_1)}
    +
    h_1
    \left(
        \frac{(u_1-u)}{(u_2-u_1)}
    \right)
    +
    h_1
\end{equation}

\begin{equation}
    h
    =
    \frac{1}{2}
    (u-u_2)
    (u_1-u)
    -
    h_2
    \frac{(u_1-u)}{(u_2-u_1)}
    +
    h_1
    \left( 
        1
        +
        \left(
            \frac{(u_1-u)}{(u_2-u_1)}
        \right)
    \right)
\end{equation}

\begin{equation}
    h
    =
    \frac{1}{2}
    (u-u_2)
    (u_1-u)
    -
    h_2
    \frac{(u_1-u)}{(u_2-u_1)}
    +
    h_1
       \frac{(u_2-u)}{(u_2-u_1)}
\end{equation}

Assuming $c_p=$ constant, the enthalpy can be related 
with the temperatute with 
the relation $h=c_pT$ 

\begin{equation}
    T
    =
    \frac{1}{2}
    \frac{1}{c_p}
    (u-u_2)
    (u_1-u)
    -
    T_2
    \frac{(u_1-u)}{(u_2-u_1)}
    +
    T_1
    \frac{(u_2-u)}{(u_2-u_1)}
\end{equation}

\begin{equation}
    T
    =
    T_1
    \frac{(u-u_2)}{(u_1-u_2)}
    + 
    T_2
    \frac{(u_1-u)}{(u_1-u_2)}
    +
    \frac{1}{2}
    \frac{1}{c_p}
    (u_1-u)
    (u-u_2)
\end{equation}

The last term and the temperature are   dimensional  and depend 
on the non dimensional 
parameters.

Using $T=T/T_1$ and $U=U/U_1$.  

\begin{equation}
    T
    T_1
    =
    T_1
    \frac{(u-u_2)}{(u_1-u_2)}
    + 
    T_2
    \frac{(u_1-u)}{(u_1-u_2)}
    +
    \frac{1}{2}
    \frac{U_1^2}{c_p}
    (u_1-u)
    (u-u_2)
\end{equation}

\begin{equation}
    T
    =
    \frac{(u-u_2)}{(u_1-u_2)}
    + 
    \frac{T_2}{T_1}
    \frac{(u_1-u)}{(u_1-u_2)}
    +
    \frac{1}{2}
    \frac{U_1^2}{T_1c_p}
    (u_1-u)
    (u-u_2)
\end{equation}

\begin{equation}
    T
    =
    \frac{(u-u_2)}{(u_1-u_2)}
    + 
    \frac{T_2}{T_1}
    \frac{(u_1-u)}{(u_1-u_2)}
    +
    \frac{1}{2}
    \frac{U_1^2\gamma_1 R}{\gamma_1RT_1c_p} (u_1-u)
    (u-u_2)
\end{equation}

\begin{equation}
    T
    =
    \frac{(u-u_2)}{(u_1-u_2)}
    + 
    \frac{T_2}{T_1}
    \frac{(u_1-u)}{(u_1-u_2)}
    +
    \frac{1}{2}
    \frac{M^2\gamma_1 R}{c_p} (u_1-u)
    (u-u_2)
\end{equation}

\begin{equation}
    \frac{R}{c_p}
    =
    \frac{c_p-c_v}{c_p}
    =
    1-\frac{1}{\gamma}
    =
    \frac{\gamma-1}{\gamma}
\end{equation}

$\beta_t=T_2/T_1$

\begin{equation}
    T
    =
    \frac{(u-u_2)}{(u_1-u_2)}
    + 
    \beta_t
    \frac{(u_1-u)}{(u_1-u_2)}
    +
    \frac{1}{2}
    M^2(\gamma_1-1) (u_1-u)
    (u-u_2)
\end{equation}

Using $T=T/T_1$ and $U=U/a_1$.  

\begin{equation}
    T
    =
    \frac{(u-u_2)}{(u_1-u_2)}
    + 
    \beta_t
    \frac{(u_1-u)}{(u_1-u_2)}
    +
    \frac{\gamma-1}{2}
    (u_1-u)
    (u-u_2)
\end{equation}
%----------------------------------------------------------------------------------------